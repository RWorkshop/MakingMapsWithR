% A measure of confidence in Bland-Altman analysis for the interchangeability of two methods of measurement.
% Preiss D1, Fisher J.
%
% Abstract: 
% Bland-Altman (B-A) analysis has largely replaced the correlation coefficient as the predominant tool for evaluating the 
% interchangeability of two methods of clinical measurement. However, we contend that B-A analysis might lead to erroneous 
% conclusions when the data range is small. 
% We provide an example to illustrate this and explore a possible analysis technique to address this limitation.

\section*{Preiss (2000) : Limitations of BA }

Correlation is a measure of association, but not agreement. However It still persists in literature.

The threshold for exchangability is a decision for tha analyst

Preiss and Fisher explore the limitations of the BA approach.

Summary : the outcome of BA analysis is dependent on the range of measurements such that narrow ranges will necessarily produce good results.
This pitfall can be avoided by determining the probability thatvsuch observed agreements occur in pairing of unassociated measurements.

Standard BA analysis can then proceed without concerns about finding good agreement from unassociated measurements.

%-----------------------------------------------------------%
% Stat Med. 1990 Apr;9(4):351-62.
% The assessment of methods of measurement.
% Chinn S.
% Criteria are given for the choice of scale prior to estimation of repeatability. 
% Recommendations of Bland and Altman should then be used for expressing repeatability and agreement of methods 
% of measurement on the same scale. 
% Repeatability of measurements on different scales should be compared using the appropriate ratio of variances, or intraclass correlation coefficient. A reference range for diagnosis requires a high ratio of between-subject variation to total variation. The index of separation between diseased and healthy subjects should be used whenever possible. 
% Changes within patients should be compared with reference change ranges, and not against the diagnostic range.


\section*{Chinn (1990) : Repeatability}
