2012 June 14th ( Barnhart et Al)
Accuracy and Precision (Barnhart et al, Pg 5)
Definition of Accuracy: Accuracy is the converse of inter-method bias.
Definition of Precision: Precision is defined by the ISO and FDA . The ISO define it as the closeness of agreement.
Repeatability (Barnhart pg 7)
ISOs definition: Closeness of agreement between measures under the same conditions. (i.e. True Replicates)
Types of agreement:  absolute agreement, relational agreement (Stine 1989), linear agreement.
Agreement (Barnhart Pg 13)
Barnhart et al remarks that the concepts of reliability and agreement may appear different.
Agreement assesses the degree of closeness between readings within a subject, while reliability assess the degree of differentiation between subjects.
MSD, reproducibility standard deviation, reproducibility variances and reproducibility coefficient are measures of agreement between observers.
Bland-Altman method (Barnhart pg 19/20)
The LOA method was extended to consider the case of replicate and repeated measures by Bland-Altman (1999).
Agreement (Barnhart Pg 13)
Due to the simplicity and “intuitive appeal”, limits of agreement are widely used for assessing agreement between two measurement methods in medical literature.
Bland and Altman describe a “method of moments” approach for estimating inter-method bias and precision in the presence of replicate measurements. (1999,2007).
Coverage Probability and TDI
This approach considers the distribution of the absolute values of the case-wise differences.
The boundary is known as the “Total deviation index”.
This probability is known as the “Coverage Probability”. TDI and CP have a relational correspondence.
As remarked by Barnhart et al, a tolerance interval derived using TDI(0.95) and 95% LOA have related interpretations.
Lin 2007 extended the approach to account for the case of replicate measuements.
Assumption of the existence of the instrument variable that may be difficult to verify in practice (Barnhart et al pg. 45).
Structural Equation Model (Latent variable Models) were proposed by Dunn(1989,2004) and Kelly (1985).

2012 May 24
The matter of how well two methods of measurement are said to be “in agreement” is a frequently posed question in statistical literature. A useful, and broadly consistent, set of definitions of what this “agreement” entail is put forth by Barnhart et al and Roy (2009). 
As pointed out by earlier contributors to the subject ( commonly referred to as “Method Comparison Studies”)
Shared with previous contributions (Bland and Altman, Carstensen) is the condition that there should no systematic  tendency for one of the methods to consistently provide a value higher that than of the other method. If such a tendency did exist, we would refer to it as an inter-method bias.
In earlier literature, the emphasis was placed up on single measurements simultaneously by each of the methods of measurement. Several different approaches, such as the Bland-Altman plot, and Orthogonal Regression (a special case of Deming Regression where the residual variances are assumed to be equal) have been proposed. Arguably, for the single replicate case, the established methodologies are sufficient for assessing agreement between two methods.
In subsequent contributions, the matter of assessing agreement in the presence  of replicate measurements was addressed. Some approaches extended already established approaches (Bland-Altam 1999).  Other contributions were based on methodologies not seen previously in Method comparison Study Literature  (for example, Carstensen et al 2008 and Roy 2009, using LME models). 
A review of recent literature demonstrates how useful and effective the use of LME models are.
