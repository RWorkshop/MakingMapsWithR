\documentclass[Chap1bmain.tex]{subfiles}
\begin{document}

\section{Other Types of Studies}
\textbf{lewis} categorize method comparison studies into three
different types.  The key difference between the first two is
whether or not a `gold standard' method is used. In situations
where one instrument or method is known to be `accurate and
precise', it is considered as the`gold standard' \textbf{lewis}. A
method that is not considered to be a gold standard is referred to
as an `approximate method'. In calibration studies they are
referred to a criterion methods and test methods respectively.


\textbf{1. Calibration problems}. The purpose is to establish a
relationship between methods, one of which is an approximate
method, the other a gold standard. The results of the approximate
method can be mapped to a known probability distribution of the
results of the gold standard \textbf{lewis}. (In such studies, the
gold standard method and corresponding approximate method are
generally referred to a criterion method and test method
respectively.) \textbf{BA83} make clear that their methodology is
not intended for calibration problems.

\bigskip \textbf{2. Comparison problems}. When two approximate
methods, that use the same units of measurement, are to be
compared. This is the case which the Bland-Altman methodology is
specfically intended for, and therefore it is the most relevant of
the three.

\bigskip \textbf{3. Conversion problems}. When two approximate
methods, that use different units of measurement, are to be
compared. This situation would arise when the measurement methods
use 'different proxies', i.e different mechanisms of measurement.
\textbf{lewis} deals specifically with this issue. In the context
of this study, it is the least relevant of the three.



\end{document}
