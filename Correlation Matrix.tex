\documentclass[12pt, a4paper]{article}
\usepackage{natbib}
\usepackage{vmargin}
\usepackage{graphicx}
\usepackage{epsfig}
\usepackage{subfigure}
%\usepackage{amscd}
\usepackage{amssymb}
\usepackage{subfigure}
\usepackage{amsbsy}
\usepackage{amsthm, amsmath}
%\usepackage[dvips]{graphicx}
\bibliographystyle{chicago}
\renewcommand{\baselinestretch}{1.4}

% left top textwidth textheight headheight % headsep footheight footskip
\setmargins{3.0cm}{2.5cm}{15.5 cm}{23.5cm}{0.5cm}{0cm}{1cm}{1cm}

\pagenumbering{arabic}


\begin{document}
\author{Kevin O'Brien}
\title{New Work }

\subsection{Extended LME model}
% Pinheiro Bates Page 202
The extended single level LME model relaxes the independence assumption, allowing heteroscedastic and correlated within group errors.


\begin{equation}
\epsilon_{i} = \mathcal{N}(0, \sigma^2 \Lambda_{i})
\end{equation}

$\Lambda_{i}$ are positive definite matrices. $\sigma^2$ is factored out of the matrix for computational reasons.


\section{Variance functions}

Variance functions are applied to LME models through the `weights' argument. $R$ supports several variance functions.

`varIdent' cosntructs a model with different variances per stratum.

\subsection{Diagnostic plots}
% Pinheiro Bates Page 391
Diagnostic plots for identifying within-group heteroscedascity and assessing the adequacy of a variance function can also be used with `nlme' objects.

\newpage
\bibliography{transferbib}
\end{document}



