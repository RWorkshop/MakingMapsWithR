\documentclass[12pt, a4paper]{report}
\usepackage{natbib}
\usepackage{vmargin}
\usepackage{graphicx}
\usepackage{epsfig}
\usepackage{subfigure}
\usepackage{subfiles}
%\usepackage{amscd}
\usepackage{amssymb}
\usepackage{amsbsy}
\usepackage{amsthm, amsmath}
%\usepackage[dvips]{graphicx}
\bibliographystyle{chicago}
\renewcommand{\baselinestretch}{1.8}

% left top textwidth textheight headheight % headsep footheight footskip
\setmargins{3.0cm}{2.5cm}{15.5 cm}{23.5cm}{0.5cm}{0cm}{1cm}{1cm}

\pagenumbering{arabic}


\begin{document}
\author{Kevin O'Brien}
\title{Chapter 2b - Method Comparison and Evaluation}
\date{\today}
\maketitle

\tableofcontents \setcounter{tocdepth}{2}

%%%%%%%%%%%%%
%1 Method Comparison Studies            %%%%%%%%%%%%%%%%%%%%%%%%%%%%%%%%%
%%%%%%%%%%%%%%%%%%%%%%%%%%%%%%%%%%%%%%%%%%%%%%%%%%%%%%%%%%%%%%%%%%%%%%%%%
% - FORMAL METHODS HERE
\subfile{ConversionProblems}
\subfile{GoldStandard}
\subfile{MeasurementErrorModels}
\subfile{OtherProblems}

\newpage
Rest of Document Here

\section{Methods of assessing agreement}

\begin{enumerate}
\item Pearson's Correlation Coefficient\item Intraclass
correlation coefficient \item Bland Altman Plot \item Bartko's
Ellipse (1994) \item Blackwood Bradley Test \item Lin's
Reproducibility Index \item Luiz Step function
\end{enumerate}

Bland and Altman attend to the issue of repeated measures in
$1996$.
\\
Repeated measurements on several subjects can be used to quantify
measurement error, the variation between measurements of the same
quantity on the same individual.
\\
Bland and Altman discuss two metrics for measurement error; the
within-subject standard deviation ,and the correlation
coefficient.

The above plot incorporates both the conventional limits of
agreement ( the inner pair of dashed lines), the `t' limits of
agreement ( the outer pair of dashed lines) centred around the
inter-method bias (indicated by the full line). This plot is
intended for expository purposes only, as the sample size is
small.





\subsection{Equivalence and Interchangeability}
Limits of agreement are intended to analyse equivalence. How this
is assessed is the considered judgement of the practitioner. In
\citet{BA86} an example of good agreement is cited. For two
methods of measuring `oxygen saturation', the limits of agreement
are calculated as (-2.0,2.8).A practitioner would ostensibly find
this to be sufficiently narrow.

If the limits of agreement are not clinically important, which is
to say that the differences tend not to be substantial, the two
methods may be used interchangeably. \citet{DunnSEME} takes issue
with the notion of `equivalence', remarking that while agreement
indicated equivalence, equivalence does not reflect agreement.




\section{Bland Altman Plots In Literature}
\citet{mantha} contains a study the use of Bland Altman plots of
44 articles in several named journals over a two year period. 42
articles used Bland Altman's limits of agreement, wit the other
two used correlation and regression analyses. \citet{mantha}
remarks that 3 papers, from 42 mention predefined maximum width
for limits of agreement which would not impair medical care.

The conclusion of \citet{mantha} is that there are several
inadequacies and inconsistencies in the reporting of results ,and
that more standardization in the use of Bland Altman plots is
required. The authors recommend the prior determination of limits
of agreement before the study is carried out. This contention is
endorsed by \citet{lin}, which makes a similar recommendation for
the sample size, noting that\emph{sample sizes required either was
not mentioned or no rationale for its choice was given}.

\begin{quote}
In order to avoid the appearance of "data dredging", both the
sample size and the (limits of agreement) should be specified and
justified before the actual conduct of the trial. \citep{lin}
\end{quote}

\citet{Dewitte} remarks that the limits of agreement should be
compared to a clinically acceptable difference in measurements.
%%%%%%%%%%%%%%%%%%%%%%%%%%%%%%%%%%%%%%%%%%%%%%%%%%%%%%%%%%%%%%%%%%%%%%%%%
%4 Inappropriate assessment of Agreement       %%%%%%%%%%%%%%%%%%%%%%%%%%
%%%%%%%%%%%%%%%%%%%%%%%%%%%%%%%%%%%%%%%%%%%%%%%%%%%%%%%%%%%%%%%%%%%%%%%%%


\subsection{Gold Standard} This is considered to be the most
accurate measurement of a particular parameter.
\section{Discussion on Method Comparison Studies}

The need to compare the results of two different measurement
techniques is common in medical statistics.
\\
\\
In particular, in medicine, new methods or devices that are
cheaper, easier to use, or less invasive, are routinely developed.
Agreement between a new method and a traditional reference or gold
standard must be evaluated before the new one is put into
practice. Various methodologies have been proposed for this
purpose in recent years.

Indications on how to deal with outliers in Bland Altman plots
\\
We wish to determine how outliers should be treated in a Bland
Altman Plot
\\
In their 1983 paper they merely state that the plot can be used to
'spot outliers'.
\\
In  their 1986 paper, Bland and Altman give an example of an
outlier. They state that it could be omitted in practice, but make
no further comments on the matter.
\\
In Bland and Altmans 1999 paper, we get the clearest indication of
what Bland and Altman suggest on how to react to the presence of
outliers. Their recommendation is to recalculate the limits
without them, in order to test the difference with the calculation
where outliers are retained.\\

The span has reduced from 77 to 59 mmHg, a noticeable but not
particularly large reduction.
\\
However, they do not recommend removing outliers. Furthermore,
they say:
\\
We usually find that this method of analysis is not too sensitive
to one or two large outlying differences.
\\
We ask if this would be so in all cases. Given that the limits of
agreement may or may not be disregarded, depending on their
perceived suitability, we examine whether it would possible that
the deletion of an outlier may lead to a calculation of limits of
agreement that are usable in all cases?
\\
Should an Outlying Observation be omitted from a data set? In
general, this is not considered prudent.
\\
Also, it may be required that the outliers are worthy of
particular attention themselves.
\\
Classifying outliers and recalculating We opted to examine this
matter in more detail. The following points have to be considered
\\how to suitably identify an outlier (in a generalized sense)
\\Would a recalculation of the limits of agreement generally
results in  a compacted range between the upper and lower limits
of agreement?
\subsection{Agreement} Bland and Altman (1986) define Perfect
agreement as 'The case where all of the pairs of rater data lie
along the line of equality'. The Line of Equality is defined as
the 45 degree line passing through the origin, or X=Y on a XY
plane.

\subsection{Lack Of Agreement}
\begin{enumerate}
\item Constant Bias\item Proportional Bias
\end{enumerate}

\subsubsection*{Constant Bias} This is a form of systematic
deviations estimated as the average difference between the test
and the reference method


\subsubsection*{Proportional Bias} Two methods may agree on
average, but they may exhibit differences over a range of
measurements\section{Bland Altman Plot} Bland Altman have
recommended the use of graphical techniques to assess agreement.
Principally their method is calculating , for each pair of
corresponding two methods of measurement of some underlying
quantity, with no replicate measurements, the difference and mean.
Differences are then plotted against the mean.
\\
Hopkins argued that the bias in a subsequent Bland-Altman plot was
due, in part, to using least-squares regression at the calibration
phase.

\subsection{Bland Altman plots using 'Gold Standard' raters}
According to Bland and Altman, one should use the methodology
previous outlined, even when one of the raters is a Gold Standard.


\subsection{Bias Detection}
further to this method, the presence of constant bias may be
indicated if the average value differences is not equal to zero.
Bland and Altman does, however, indicate the indication of absence
of bias does not provide sufficient information to allow a
judgement as to whether or not one method can be substituted for
another.










\addcontentsline{toc}{section}{Bibliography}

\bibliography{newbib}
\end{document}
