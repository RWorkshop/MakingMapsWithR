\subsection{Using LME models to create Prediction Intervals}

%
\textbf{Carstensen et al - Mixed Models}
\begin{itemize}
\item Carstensen et al [4] also advocates the use of linear mixed models in
the study of method comparisons. The model is constructed to
describe the relationship between a value of measurement and its
real value. 

\item The non-replicate case is considered first, as it is
the context of the Bland-Altman plots. 
\item This model assumes that
\textit{inter-method bias} is the only difference between the two methods.
A measurement $y_{mi}$ by method $m$ on individual $i$ is
formulated as follows;
\end{itemize}

\begin{equation}
y_{mi}  = \alpha_{m} + \mu_{i} + e_{mi} \qquad ( e_{mi} \sim
N(0,\sigma^{2}_{m}))
\end{equation}

%%%%%%%%%%%%%%%%%%%%%%%%%%%%%%%%%%%%%%%%%%%%%%%%%%%%%%%%%%%%%%%%%%%%%%%%%%%%%%%%%%%%%%

%%%%%%%%%%%%%%%%%%%%%%%%%%%%%%%%%%%%%%%%%%%%%%%%%%%%%%%%%%%%%%%%%%%%%%%%%%%%%%%%%%%%%%

%
% \frametitle{Carstensen's Mixed Models}
\begin{itemize}
\item Carstensen et al [5] sets out a methodology of computing the limits of
agreement based upon variance component estimates derived using
linear mixed effects models. 
\item Measures of repeatability, a
characteristic of individual methods of measurements, are also
derived using this method.
\end{itemize}


%%%%%%%%%%%%%%%%%%%%%%%%%%%%%%%%%%%%%%%%%%%%%%%%%%%%%%%%%%%%%%%%%%%%%%%%%%%%%%%%%%%%%%
%
% \frametitle{Carstensen's Mixed Models}
\large
\begin{itemize}
\item The differences are expressed as $d_{i} = y_{1i} - y_{2i}$.
\item For the
replicate case, an interaction term $c$ is added to the model,
with an associated variance component. 
\item All the random effects are
assumed independent, and that all replicate measurements are
assumed to be exchangeable within each method.
\end{itemize}


\begin{equation}
y_{mir}  = \alpha_{m} + \mu_{i} + c_{mi} + e_{mir} \qquad ( e_{mi}
\sim N(0,\sigma^{2}_{m}), c_{mi} \sim N(0,\tau^{2}_{m}))
\end{equation}
%%%%%%%%%%%%%%%%%%%%%%%%%%%%%%%%%%%%%%%%%%%%%%%%%%%%%%%%%%%%%%%%%%%%%%%%%%%%%%%%%%%%%%


%%%%%%%%%%%%%%%%%%%%%%%%%%%%%%%%%%%%%%%%%%%%%%%%%%%%%%%%%%%%%%%%%%%%%%%%%%%%%%%%%%%%%%
%
\large
\begin{itemize}
\item Carstensen \textit{et al} \cite{BXC2004} also advocates the use of linear mixed models in
the study of method comparisons. 
\item The model is constructed to
describe the relationship between a value of measurement and its
real value.
\item  The non-replicate case is considered first, as it is
the context of the Bland Altman plots. This model assumes that
inter-method bias is the only difference between the two methods.
A measurement $y_{mi}$ by method $m$ on individual $i$ is
formulated as follows;
\end{itemize}
\begin{equation}
y_{mi}  = \alpha_{m} + \mu_{i} + e_{mi} \qquad ( e_{mi} \sim
N(0,\sigma^{2}_{m}))
\end{equation}


%
\Large
\begin{itemize}
\item The differences are expressed as $d_{i} = y_{1i} - y_{2i}$ For the
replicate case, an interaction term $c$ is added to the model,
with an associated variance component. 
\item All the random effects are
assumed independent, and that all replicate measurements are
assumed to be exchangeable within each method.
\end{itemize}
\begin{eqnarray}
y_{mir}  = \alpha_{m} + \mu_{i} + c_{mi} + e_{mir} 
\end{eqnarray}

%------------------------------------------------------------------------------ %
%
% \frametitle{Carstensen's Mixed Models}
\large
\begin{itemize}
\item The following model (in the authors own notation) is
formulated as follows, where $y_{mir}$ is the $r$th replicate
measurement on subject $i$ with method $m$.
\end{itemize}
{
\LARGE
\begin{equation}
y_{mir}  = \alpha_{m} + \mu_{i} + c_{mi} + e_{mir} \qquad ( e_{mi}
\sim N(0,\sigma^{2}_{m}), c_{mi} \sim N(0,\tau^{2}_{m}))
\end{equation}


\begin{equation}
y_{mir}  = \alpha_{m} + \beta_{m}\mu_{i} + c_{mi} + e_{mir} 
\end{equation}
}
\vspace{0.3cm}
{
\normalsize
\[ e_{mi} \sim N(0,\sigma^{2}_{m}), c_{mi} \sim N(0,\tau^{2}_{m})\]
}

%------------------------------------------------------- %
 %SLIDE 3
%[fragile]
% \frametitle{Carstensen's Mixed Models}
\begin{itemize}
\item The intercept term $\alpha$ and the $\beta_{m}\mu_{i}$ term follow
from \textit{Dunn} \cite{DunnSEME}, expressing constant and proportional bias
respectively , in the presence of a real value $\mu_{i}.$
\item $c_{mi}$ is a interaction term to account for replicate, and
 $e_{mir}$ is the residual associated with each observation.
\item Since variances are specific to each method, this model can be
fitted separately for each method.
\end{itemize}


%---------------------------------------------------------------- %
%
% \frametitle{Carstensen's Mixed Models}
\begin{itemize}
\item The above formulation doesn't require the data set to be balanced.
However, it does require a sufficient large number of replicates
and measurements to overcome the problem of identifiability. 
\item The
import of which is that more than two methods of measurement may
be required to carry out the analysis. 
\end{itemize}


%---------------------------------------------------------------- %
%
% \frametitle{Carstensen's Mixed Models}
\begin{itemize}
\item There is also the
assumptions that observations of measurements by particular
methods are exchangeable within subjects. \item \textbf{\textit{Exchangeability}} means
that future samples from a population behaves like earlier
samples).
\end{itemize}

%---------------------------------------------------------------- %

%-----------------------%
%
% \frametitle{Computing LoAs from LME models}
\emph{
One important feature of replicate observations is that they should be independent
of each other. In essence, this is achieved by ensuring that the observer makes each
measurement independent of knowledge of the previous value(s). This may be difficult
to achieve in practice.}


\subsection*{Using LME models to create Prediction Intervals}


%\[  e_{mi} \sim N(0,\sigma^{2}_{m}) \c_{mi} \sim N(0,\tau^{2}_{m}) \]
%\end{eqnarray}
%

%
\Large
\begin{itemize}
\item Carstensen \textit{et al} \cite{BXC2008} proposes a methodology to calculate prediction
intervals in the presence of replicate measurements, overcoming
problems associated with Bland-Altman methodology in this regard.
\item It is not possible to estimate the interaction variance components
$\tau^{2}_{1}$ and $\tau^{2}_{2}$ separately. Therefore it must be
assumed that they are equal. The variance of the difference can be
estimated as follows:
\begin{equation}
var(y_{1j}-y_{2j})
\end{equation}
\end{itemize}

%-------------------------------------------------------- %
\subsection{Carstensen's Mixed Models}


%
% Cut This Slide?
Carstensen \textit{et al}[4] proposes linear mixed effects models for deriving
conversion calculations similar to Deming's regression, and for
estimating variance components for measurements by different
methods. The following model ( in the authors own notation) is
formulated as follows, where $y_{mir}$ is the $r$th replicate
measurement on subject $i$ with method $m$.

\begin{equation}
y_{mir}  = \alpha_{m} + \beta_{m}\mu_{i} + c_{mi} + e_{mir} \qquad
( e_{mi} \sim N(0,\sigma^{2}_{m}), c_{mi} \sim N(0,\tau^{2}_{m}))
\end{equation}

%%%%%%%%%%%%%%%%%%%%%%%%%%%%%%%%%%%%%%%%%%%%%%%%%%%%%%%%%%%%%%%%%%%%%%%%%%%%%%%%%%%%%%
%
The intercept term $\alpha$ and the $\beta_{m}\mu_{i}$ term follow
from Dunn[7], expressing constant and proportional bias
respectively , in the presence of a real value $\mu_{i}.$
 $c_{mi}$ is a interaction term to account for replicate, and
 $e_{mir}$ is the residual associated with each observation.
Since variances are specific to each method, this model can be
fitted separately for each method.

%-----------------------------------------------------------------------%
%
% \frametitle{Carstensen's Mixed Models}
\begin{itemize}
\item This model includes a method by item interaction term.\\

\item Carstensen presents two models. One for the case where the replicates, and a second for when they are linked.\\
\item Carstensen's model does not take into account either between-item or within-item covariance between methods.\\
\item In the presented example, it is shown that Roy's LoAs are lower than those of Carstensen.
\end{itemize}



\[\left(\begin{array}{cc}
                \omega^1_2  & 0 \\
              0 & \omega^2_2 \\
            \end{array}  \right)
            =  \left(
            \begin{array}{cc}
              \tau^2  & 0 \\
              0 & \tau^2 \\
            \end{array} \right)+
            \left(
            \begin{array}{cc}
              \sigma^2_1  & 0 \\
              0 & \sigma^2_2 \\
            \end{array}\right)
\]






%-----------------------------------------------------------------------------------%
%
% \frametitle{Carstensen model in the single measurement case}
\begin{itemize}
\item Carstensen \textit{et al}[4] presents a model to describe the relationship between a value of measurement and its real value.
\item The non-replicate case is considered first, as it is the context of the Bland-Altman plots.
\item This model assumes that inter-method bias is the only difference between the two methods.
\end{itemize}


%-----------------------------------------------------------------------------------%
%
% \frametitle{Carstensen model in the single measurement case}

\begin{equation}
y_{mi}  = \alpha_{m} + \mu_{i} + e_{mi} \qquad  e_{mi} \sim \mathcal{N}(0,\sigma^{2}_{m})
\end{equation}

The differences are expressed as $d_{i} = y_{1i} - y_{2i}$.

For the replicate case, an interaction term $c$ is added to the model, with an associated variance component.

\subsection{Computing LoAs from LME models}
%--------------------------%

%
\textbf{Section 8 computing LoAs from LME models}

\begin{itemize}
\item
\end{itemize}

%--------------------------%

%
\emph{
One important feature of replicate observations is that they should be independent
of each other. In essence, this is achieved by ensuring that the observer makes each
measurement independent of knowledge of the previous value(s). This may be difficult
to achieve in practice.}


%-------------------------------------------------------------------------------------%
\subsection{Carstensen's LOAs}
%
Carstensen presents a model where the variation between items for
method $m$ is captured by $\sigma_m$ and the within item variation
by $\tau_m$.

Further to his model, Carstensen computes the limits of agreement
as

\[
\hat{\alpha}_1 - \hat{\alpha}_2 \pm \sqrt{2 \hat{\tau}^2 +
\hat{\sigma}^2_1 + \hat{\sigma}^2_2}
\]

%-------------------------------------------------------------------------------------%
%
% \frametitle{Carstensen's LOAs}
\large
\begin{itemize}
\item The respective estimates computed by both methods are tabulated as follows. Evidently there is close correspondence between both sets of estimates.

\item \alert{bxc2008} formulates an LME model, both in the absence and the presence of an interaction term.\alert{bxc} uses both to demonstrate the importance of using an interaction term. Failure to take the replication structure into
account results in over-estimation of the limits of agreement. 
\item For the Carstensen estimates below, an interaction term was included when computed.
\end{itemize}


\end{document}
