\documentclass[12pt, a4paper]{report}
\usepackage{natbib}
\usepackage{vmargin}
\usepackage{epsfig}
\usepackage{subfigure}
%\usepackage{amscd}
\usepackage{amssymb}
\usepackage{amsbsy}
\usepackage{amsthm,amsmath}
\usepackage{setspace}
%\usepackage[dvips]{graphicx}
\renewcommand{\baselinestretch}{1.2}
% left top textwidth textheight headheight
% headsep footheight footskip
\setmargins{3.0cm}{2.5cm}{15.5 cm}{23.5cm}{0.5cm}{0cm}{1cm}{1cm}
\pagenumbering{arabic}

\doublespacing
\title{Transfer}
\author{Kevin O'Brien}
\date{Summer 2009}
\begin{document}
\chapter{LME in MCS}
\section{Carstensen's Model}

\section{Carstensen's Limits of agreement}
\citet{bxc2008} presents a methodology to compute the limits of
agreement based on LME models. Importantly, Carstensen's underlying model differs from Roy's model in some key respects, and therefore a prior discussion of Carstensen's model is required.

\subsection{Carstensen's Model}

\citet{BXC2004} presents a model to describe the relationship between a value of measurement and its
real value. The non-replicate case is considered first, as it is the context of the Bland Altman plots. This model assumes that inter-method bias is the only difference between the two methods.

A measurement $y_{mi}$ by method $m$ on individual $i$ is formulated as follows;
\begin{equation}
y_{mi}  = \alpha_{m} + \mu_{i} + e_{mi} \qquad  e_{mi} \sim
\mathcal{N}(0,\sigma^{2}_{m})
\end{equation}
The differences are expressed as $d_{i} = y_{1i} - y_{2i}$. For the replicate case, an interaction term $c$ is added to the model, with an associated variance component. All the random effects are assumed independent, and that all replicate measurements are assumed to be exchangeable within each method.

\begin{equation}
y_{mir}  = \alpha_{m} + \mu_{i} + c_{mi} + e_{mir}, \qquad  e_{mi}
\sim \mathcal{N}(0,\sigma^{2}_{m}), \quad c_{mi} \sim \mathcal{N}(0,\tau^{2}_{m}).
\end{equation}
%----
\citet{BXC2008} uses an approach based on linear mixed effects (LME) models for the purpose of computing the limits of agreement for two methods of measurement, where replicate measurements are taken on items. As the emphasis of this methodology lies on the inter-method bias and the limits of agreement, the two key elements of the Bland-Altman methodology, other formal tests are not described.

Using Carstensen's notation, a measurement $y_{mi}$ by method $m$ on individual $i$ the measurement $y_{mir} $ is the $r$th replicate measurement on the $i$th item by the $m$th method, where $m=1,2,$ $i=1,\ldots,N,$ and $r = 1,\ldots,n_i$ is formulated as follows;

\begin{equation}
y_{mir}  = \alpha_{m} + \mu_{i} + c_{mi} + \epsilon_{mir}, \qquad  e_{mi}
\sim \mathcal{N}(0,\sigma^{2}_{m}), \quad c_{mi} \sim \mathcal{N}(0,\tau^{2}_{m}).
\end{equation}

Here the terms $\alpha_{m}$ and $\mu_{i}$ represent the fixed effect for method $m$ and a true value for item $i$ respectively. The random effect terms comprise an interaction term $c_{mi}$ and the residuals $\epsilon_{mir}$.
The $c_{mi}$ term represent random effect parameters corresponding to the two methods, having $\mathrm{E}(c_{mi})=0$ with $\mathrm{Var}(c_{mi})=\tau^2_m$. Carstensen specifies the variance of the interaction terms as being univariate normally distributed. As such, $\mathrm{Cov}(c_{mi}, c_{m^\prime i})= 0.$ All the random effects are assumed independent, and that all replicate measurements are assumed to be exchangeable within each method.

With regards to specifying the variance terms, Carstensen remarks that using his approach is common, remarking that \emph{
The only slightly non-standard (meaning "not often used") feature is the differing residual variances between methods }\citep{bxc2010}.

In contrast to Roy's model, Carstensen's model requires that commonly used assumptions be applied, specifically that the off-diagonal elements of the between-item and within-item variability matrices are zero. By
extension the overall variability off-diagonal elements are also zero. Also, implementation requires that the between-item variances are estimated as the same value: $g^2_1 = g^2_2 = g^2$.
As a consequence, Carstensen's method does not allow for a formal test of the between-item variability.

\[\left(\begin{array}{cc}
                \omega^1_2  & 0 \\
              0 & \omega^2_2 \\
            \end{array}  \right)
            =  \left(
            \begin{array}{cc}
              \tau^2  & 0 \\
              0 & \tau^2 \\
            \end{array} \right)+
            \left(
            \begin{array}{cc}
              \sigma^2_1  & 0 \\
              0 & \sigma^2_2 \\
            \end{array}\right)
\]


%---Key difference 1---The True Value
%---Colollary -- Difference in model types
The presence of the true value term $\mu_i$ gives rise to an important difference between Carstensen's and Roy's models. The fixed effect of Roy's model comprise of an intercept term and fixed effect terms for both methods, with no reference to the true value of any individual item. In other words, Roy considers the group of items being measured as a sample taken from a population. Therefore a distinction can be made between the two models: Roy's model is a standard LME model, whereas Carstensen's model is a more complex additive model.

%---Carstensen's limits of agreement
%---The between item variances are not individually computed. An estimate for their sum is used.
%---The within item variances are indivdually specified.
%---Carstensen remarks upon this in his book (page 61), saying that it is "not often used".
%---The Carstensen model does not include covariance terms for either VC matrices.
%---Some of Carstensens estimates are presented, but not extractable, from R code, so calculations have to be done by %---hand.
%---All of Roys stimates are  extractable from R code, so automatic compuation can be implemented
%---When there is negligible covariance between the two methods, Roys LoA and Carstensen's LoA are roughly the same.
%---When there is covariance between the two methods, Roy's LoA and Carstensen's LoA differ, Roys usually narrower.


%-----------------------------------------------------------------------------------%

\section{Carstensen's Limits of agreement}
\citet{bxc2008} presents a methodology to compute the limits of agreement based on LME models. The method of computation is the
same as Roy's model, but with the covariance estimates set to zero.

In cases where there is negligible covariance between methods, the limits of agreement computed using Roy's model accord with those computed using Carstensen's model. In cases where some degree of
covariance is present between the two methods, the limits of agreement computed using models will differ. In the presented
example, it is shown that Roy's LoAs are lower than those of Carstensen, when covariance is present.

Importantly, estimates required to calculate the limits of agreement are not extractable, and therefore the calculation must
be done by hand.
%-----------------------------------------------------------------------------------------------------%
\section{Repeated Measurements}

In cases where there are repeated measurements by each of the two
methods on the same subjects , Bland Altman suggest calculating
the mean for each method on each subject and use these pairs of
means to compare the two methods.
The estimate of bias will be unaffected using this approach, but
the estimate of the standard deviation of the differences will be
too small, because of the reduction of the effect of repeated
measurement error. Bland Altman propose a correction for this.
Carstensen attends to this issue also, adding that another
approach would be to treat each repeated measurement separately.
\section{Carstensen's Mixed Models}

\citet{BXC2004} proposes linear mixed effects models for deriving
conversion calculations similar to Deming's regression, and for
estimating variance components for measurements by different
methods. The following model ( in the authors own notation) is
formulated as follows, where $y_{mir}$ is the $r$th replicate
measurement on subject $i$ with method $m$.

\begin{equation}
y_{mir}  = \alpha_{m} + \beta_{m}\mu_{i} + c_{mi} + e_{mir} \qquad
( e_{mi} \sim N(0,\sigma^{2}_{m}), c_{mi} \sim N(0,\tau^{2}_{m}))
\end{equation}
The intercept term $\alpha$ and the $\beta_{m}\mu_{i}$ term follow
from \citet{DunnSEME}, expressing constant and proportional bias
respectively , in the presence of a real value $\mu_{i}.$
 $c_{mi}$ is a interaction term to account for replicate, and
 $e_{mir}$ is the residual associated with each observation.
Since variances are specific to each method, this model can be
fitted separately for each method.

The above formulation doesn't require the data set to be balanced.
However, it does require a sufficient large number of replicates
and measurements to overcome the problem of identifiability. The
import of which is that more than two methods of measurement may
be required to carry out the analysis. There is also the
assumptions that mobservations of measurements by particular
methods are exchangeable within subjects. (Exchangeability means
that future samples from a population behaves like earlier
samples).

%\citet{BXC2004} describes the above model as a `functional model',
%similar to models described by \citet{Kimura}, but without any
%assumptions on variance ratios. A functional model is . An
%alternative to functional models is structural modelling

\citet{BXC2004} uses the above formula to predict observations for
a specific individual $i$ by method $m$;

\begin{equation}BLUP_{mir} = \hat{\alpha_{m}} + \hat{\beta_{m}}\mu_{i} +
c_{mi} \end{equation}. Under the assumption that the $\mu$s are
the true item values, this would be sufficient to estimate
parameters. When that assumption doesn't hold, regression techniques (known as updating techniques)
can be used additionally to determine the estimates.
The assumption of exchangeability can be unrealistic in certain situations.
\citet{BXC2004} provides an amended formulation which includes an extra interaction
term ($d_{mr} d_{mr} \sim N(0,\omega^{2}_{m}$)to account for this.

\citet{BXC2008} sets out a methodology of computing the limits of
agreement based upon variance component estimates derived using
linear mixed effects models. Measures of repeatability, a
characteristic of individual methods of measurements, are also
derived using this method.

\subsection{Using LME models to create Prediction Intervals}
\citet{BXC2004} also advocates the use of linear mixed models in
the study of method comparisons. The model is constructed to
describe the relationship between a value of measurement and its
real value. The non-replicate case is considered first, as it is
the context of the Bland Altman plots. This model assumes that
inter-method bias is the only difference between the two methods.
A measurement $y_{mi}$ by method $m$ on individual $i$ is
formulated as follows;
\begin{equation}
y_{mi}  = \alpha_{m} + \mu_{i} + e_{mi} \qquad ( e_{mi} \sim
N(0,\sigma^{2}_{m}))
\end{equation}
The differences are expressed as $d_{i} = y_{1i} - y_{2i}$ For the
replicate case, an interaction term $c$ is added to the model,
with an associated variance component. All the random effects are
assumed independent, and that all replicate measurements are
assumed to be exchangeable within each method.

\begin{equation}
y_{mir}  = \alpha_{m} + \mu_{i} + c_{mi} + e_{mir} \qquad ( e_{mi}
\sim N(0,\sigma^{2}_{m}), c_{mi} \sim N(0,\tau^{2}_{m}))
\end{equation}

\citet{BXC2008} proposes a methodology to calculate prediction
intervals in the presence of replicate measurements, overcoming
problems associated with Bland-Altman methodology in this regard.
It is not possible to estimate the interaction variance components
$\tau^{2}_{1}$ and $\tau^{2}_{2}$ separately. Therefore it must be
assumed that they are equal. The variance of the difference can be
estimated as follows:
\begin{equation}
var(y_{1j}-y_{2j})
\end{equation}

\subsection{Computation} Modern software
packages can be used to fit models accordingly. The best linear
unbiased predictor (BLUP) for a specific subject $i$ measured with
method $m$ has the form $BLUP_{mir} = \hat{\alpha_{m}} +
\hat{\beta_{m}}\mu_{i} + c_{mi}$, under the assumption that the
$\mu$s are the true item values.




%%%%%%%%%%%%%%%%%%%%%%%%%%%%%%%%%%%%%%%%%%%%%%%%%%%%%%%%%%%%%%%%%%%%%%%%%%%%%%%%%%%%%%%%%%%%%%%%%%%%%%%%% Other Approaches


\newpage



\citet{pkcng} generalize this approach to account for situations
where the distributions are not identical, which is commonly the
case. The TDI is not consistent and may not preserve its
asymptotic nominal level, and that the coverage probability
approach of \citet{lin2002} is overly conservative for moderate
sample sizes. This methodology proposed by \citet{pkcng} is a
regression based approach that models the mean and the variance of
differences as functions of observed values of the average of the
paired measurements.
%%%%%%%%%%%%%%%%%%%%%%%%%%%%%%%%%%%%%%%%%%%%%%%%%%%%%%%%%%%%%%%%%%%%%%%%%%%%%%%%%%%%%%%%%%%%%%%%%%%%%%%%%5

Maximum likelihood estimation is used to estimate the parameters.
The REML estimation is not considered since it does not lead to a
joint distribution of the estimates of fixed effects and random
effects parameters, upon which the assessment of agreement is
based.



\section{LME}
Consistent with the conventions of mixed models, \citet{pkc}
formulates the measurement $y_{ij} $from method $i$ on individual
$j$ as follows;
\begin{equation}
y_{ij} =P_{ij}\theta + W_{ij}v_{i} + X_{ij}b_{j} + Z_{ij}u_{j} +
\epsilon_{ij},     (j=1,2, i=1,2....n)
\end{equation}
The design matrix $P_{ij}$ , with its associated column vector
$\theta$, specifies the fixed effects common to both methods. The
fixed effect specific to the $j$th method is articulated by the
design matrix $W_{ij}$ and its column vector $v_{i}$. The random
effects common to both methods is specified in the design matrix
$X_{ij}$, with vector $b_{j}$ whereas the random effects specific
to the $i$th subject by the $j$th method is expressed by $Z_{ij}$,
and vector $u_{j}$. Noticeably this notation is not consistent
with that described previously.  The design matrices are specified
so as to includes a fixed intercept for each method, and a random
intercept for each individual. Additional assumptions must also be
specified;
\begin{equation}
v_{ij} \sim N(0,\Sigma),
\end{equation}
These vectors are assumed to be independent for different $i$s,
and are also mutually independent. All Covariance matrices are
positive definite.  In the above model effects can be classed as
those common to both methods, and those that vary with method.
When considering differences, the effects common to both
effectively cancel each other out. The differences of each pair of
measurements can be specified as following;
\begin{equation}
d_{ij} = X_{ij}b_{j} + Z_{ij}u_{j} + \epsilon_{ij},     (j=1,2,
i=1,2....n)
\end{equation}
This formulation has seperate distributional assumption from the
model stated previously.

This agreement covariate $x$ is the key step in how this
methodology assesses agreement.
%%%%%%%%%%%%%%%%%%%%%%%%%%%%%%%%%%%%%%%%%%%%%%%%%%%%%%%%%%%%%%%%%%%%%%%%%%%%%%%%%%%%%%%%%%%%%%%%%%%%%%%5


\bibliographystyle{chicago}
\bibliography{transferbib}
\end{document}
