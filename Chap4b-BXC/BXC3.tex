\documentclass[Chap4bmain.tex]{subfiles}
\begin{document}

\documentclass[Chap4bmain.tex]{subfiles}

%Carstensen’s LME model
%LoA as computed by Carstensen’s LME model Papers
%Carstensen et Al 2006
%Carstensen et al 2008
%Bendix Carstensen 2010
% Section 5.3 Models for replicate measurements
% Section 7 A general model for method comparisons.
% Section 7.2 Interpretation of Random effects

\begin{document}


\section{Carstensen's Model}
\citet{BXC2008} proposes an approach for comparing two or more methods of measurement based on linear mixed effects models. This approach extends the well established Bland-Altman methodology for the case of replicate measurements on each item. Carstensen considers the matter of computing an appropriate estimate for the standard deviation of case-wise differences, so as to determine the limits of agreement. As the interest lies in extending the Bland-Altman methodology, other formal tests are not described.

Using Carstensen's notation, a measurement $y_{mi}$ by method $m$ on individual $i$ the measurement $y_{mir} $ is the $r$th replicate measurement on the $i$th item by the $m$th method, where $m=1,2,\ldots,M$ $i=1,\ldots,N,$ and $r = 1,\ldots,n_i$ is formulated as follows;
\begin{equation}
y_{mir}  = \alpha_{m} + \mu_{i} + c_{mi} + a_{ir} + \epsilon_{mir}, \qquad \quad c_{mi} \sim \mathcal{N}(0,\tau^{2}_{m}) , a_{ir} \sim \mathcal{N}(0,\varsigma^{2}),  \varepsilon_{mi} \sim \mathcal{N}(0,\varphi^{2}_{m}) .
\end{equation}

Here the terms $\alpha_{m}$ and $\mu_{i}$ represent the fixed effect for method $m$ and a true value for item $i$ respectively. The random effect terms comprise an interaction term $c_{mi}$ and the residuals $\varepsilon_{mir}$.
The $c_{mi}$ term represent random effect parameters corresponding to the two methods, having $\mathrm{E}(c_{mi})= 0$ with $\mathrm{Var}(c_{mi})=\tau^2_m$.  

%%%%Stuff about extra interaction term

The random error term for each response is denoted $\varepsilon_{mir}$ having $\mathrm{E}(\varepsilon_{mir})=0$, $\mathrm{Var}(\varepsilon_{mir})=\varphi^2_m$. All the random effects are assumed independent, and that all replicate measurements are assumed to be exchangeable within each method.

%Carstensen specifies the variance of the interaction terms as being univariate normally distributed. As such, $\mathrm{Cov}(c_{mi}, c_{m^\prime i})= 0.$

When only two methods are to be compared, separate estimates of $\tau^2_m$ can not be obtained. Instead the average value $\tau^2$ is obtained and used.


Carstensen's approach is that of a standard two-way mixed effects ANOVA with replicate measurements. With regards to the specification of the variance terms, Carstensen remarks that using his approach is common, remarking that \emph{
The only slightly non-standard (meaning "not often used") feature is the differing residual variances between methods }\citep{bxc2010}.

In contrast to Roy's model, Carstensen's model requires that commonly used assumptions be applied, specifically that the off-diagonal elements of the between-item and within-item variability matrices are zero. By
extension the overall variability off-diagonal elements are also zero. Also, implementation requires that the between-item variances are estimated as the same value: $\tau^2_1 = \tau^2_2 = \tau^2$.


\[\left(\begin{array}{cc}
                \omega^1_2  & 0 \\
              0 & \omega^2_2 \\
            \end{array}  \right)
            =  \left(
            \begin{array}{cc}
              \tau^2  & 0 \\
              0 & \tau^2 \\
            \end{array} \right)+
            \left(
            \begin{array}{cc}
              \sigma^2_1  & 0 \\
              0 & \sigma^2_2 \\
            \end{array}\right)
\]


%---Key difference 1---The True Value
%---Colollary -- Difference in model types
The presence of the true value term $\mu_i$ gives rise to an important difference between Carstensen's and Roy's models. The fixed effect of Roy's model comprise of an intercept term and fixed effect terms for both methods, with no reference to the true value of any individual item. In other words, Roy considers the group of items being measured as a sample taken from a population. Therefore a distinction can be made between the two models: Roy's model is a standard LME model, whereas Carstensen's model is a more complex additive model.

%---Carstensen's limits of agreement
%---The between item variances are not individually computed. An estimate for their sum is used.
%---The within item variances are indivdually specified.
%---Carstensen remarks upon this in his book (page 61), saying that it is "not often used".
%---The Carstensen model does not include covariance terms for either VC matrices.
%---Some of Carstensens estimates are presented, but not extractable, from R code, so calculations have to be done by %---hand.
%---All of Roys stimates are  extractable from R code, so automatic compuation can be implemented
%---When there is negligible covariance between the two methods, Roys LoA and Carstensen's LoA are roughly the same.
%---When there is covariance between the two methods, Roy's LoA and Carstensen's LoA differ, Roys usually narrower.


\section{Carstensen's Mixed Models}

\citet{BXC2004} proposes linear mixed effects models for deriving
conversion calculations similar to Deming's regression, and for
estimating variance components for measurements by different
methods. The following model ( in the authors own notation) is
formulated as follows, where $y_{mir}$ is the $r$th replicate
measurement on subject $i$ with method $m$.

\begin{equation}
y_{mir}  = \alpha_{m} + \beta_{m}\mu_{i} + c_{mi} + e_{mir} \qquad
( e_{mi} \sim N(0,\sigma^{2}_{m}), c_{mi} \sim N(0,\tau^{2}_{m}))
\end{equation}
The intercept term $\alpha$ and the $\beta_{m}\mu_{i}$ term follow
from \citet{DunnSEME}, expressing constant and proportional bias
respectively , in the presence of a real value $\mu_{i}.$
 $c_{mi}$ is a interaction term to account for replicate, and
 $e_{mir}$ is the residual associated with each observation.
Since variances are specific to each method, this model can be
fitted separately for each method.

The above formulation doesn't require the data set to be balanced.
However, it does require a sufficient large number of replicates
and measurements to overcome the problem of identifiability. The
import of which is that more than two methods of measurement may
be required to carry out the analysis. There is also the
assumptions that observations of measurements by particular
methods are exchangeable within subjects. (Exchangeability means
that future samples from a population behaves like earlier
samples).

%\citet{BXC2004} describes the above model as a `functional model',
%similar to models described by \citet{Kimura}, but without any
%assumptions on variance ratios. A functional model is . An
%alternative to functional models is structural modelling

\citet{BXC2004} uses the above formula to predict observations for
a specific individual $i$ by method $m$;

\begin{equation}BLUP_{mir} = \hat{\alpha_{m}} + \hat{\beta_{m}}\mu_{i} +
c_{mi} \end{equation}. Under the assumption that the $\mu$s are
the true item values, this would be sufficient to estimate
parameters. When that assumption doesn't hold, regression techniques (known as updating techniques)
can be used additionally to determine the estimates.
The assumption of exchangeability can be unrealistic in certain situations.
\citet{BXC2004} provides an amended formulation which includes an extra interaction
term ($d_{mr} d_{mr} \sim N(0,\omega^{2}_{m}$)to account for this.

\citet{BXC2008} sets out a methodology of computing the limits of
agreement based upon variance component estimates derived using
linear mixed effects models. Measures of repeatability, a
characteristic of individual methods of measurements, are also
derived using this method.

\newpage


\citet{BXC2004} uses the above formula to predict observations for
a specific individual $i$ by method $m$;

\begin{equation}BLUP_{mir} = \hat{\alpha_{m}} + \hat{\beta_{m}}\mu_{i} +
c_{mi} \end{equation}. Under the assumption that the $\mu$s are
the true item values, this would be sufficient to estimate
parameters. When that assumption doesn't hold, regression
techniques (known as updating techniques) can be used additionally
to determine the estimates. The assumption of exchangeability can
be unrealistic in certain situations. \citet{BXC2004} provides an
amended formulation which includes an extra interaction term ($
d_{mr} \sim N(0,\omega^{2}_{m}$)to account for this.


\newpage
\citet{BXC2008} sets out a methodology of computing the limits of
agreement based upon variance component estimates derived using
linear mixed effects models. Measures of repeatability, a
characteristic of individual methods of measurements, are also
derived using this method.

\citet{BXC2004} also advocates the use of linear mixed models in
the study of method comparisons. The model is constructed to
describe the relationship between a value of measurement and its
real value. The non-replicate case is considered first, as it is
the context of the Bland-Altman plots. This model assumes that
inter-method bias is the only difference between the two methods.
A measurement $y_{mi}$ by method $m$ on individual $i$ is
formulated as follows;
\begin{equation}
y_{mi}  = \alpha_{m} + \mu_{i} + e_{mi} \qquad ( e_{mi} \sim
N(0,\sigma^{2}_{m}))
\end{equation}
The differences are expressed as $d_{i} = y_{1i} - y_{2i}$ For the
replicate case, an interaction term $c$ is added to the model,
with an associated variance component. All the random effects are
assumed independent, and that all replicate measurements are
assumed to be exchangeable within each method.

\begin{equation}
y_{mir}  = \alpha_{m} + \mu_{i} + c_{mi} + e_{mir} \qquad ( e_{mi}
\sim N(0,\sigma^{2}_{m}), c_{mi} \sim N(0,\tau^{2}_{m}))
\end{equation}

\citet{BXC2008} proposes a methodology to calculate prediction
intervals in the presence of replicate measurements, overcoming
problems associated with Bland-Altman methodology in this regard.
It is not possible to estimate the interaction variance components
$\tau^{2}_{1}$ and $\tau^{2}_{2}$ separately. Therefore it must be
assumed that they are equal. The variance of the difference can be
estimated as follows:
\begin{equation}
var(y_{1j}-y_{2j})
\end{equation}


%-----------------------------------------------------------------------------------------------------%
\newpage

\subsection{Bendix Carstensen's data sets}
\citet{bxc2008}describes the sampling method when discussing of a motivating example.Diabetes patients attending an outpatient clinic in Denmark have their $HbA_{1c}$ levels routinely measured at every visit.Venous and Capillary blood samples were obtained from all patients appearing at the clinic over two days.

Samples were measured on four consecutive days on each machines, hence there are five analysis days.Carstensen notes that every machine was calibrated every day to  the manufacturers guidelines.

\subsection{Carstensen Methods}
Bendix Carstensen et al. proposed the use of LME models to allow for a more statistically rigourous approach to computing Limits of Agreement.  The respective papers also discuss several shortcoming for techniques for dealing with replicate measurements, as proposed by Bland-Altman 1999.

%---------------------------------------------------------------%
Components

\begin{verbatim}



Section 5.3 Models for replicate measurements
Section 5 Replicate measurements.

Carstensen page 56
%----------------------------------------------------------------%
air extra random effect that does not depend on method.
It is treated as an extension of i.
The variance of air represents the variation between replication condition (common for all methods), within items, .
\end{verbatim}
\[ymir=m+i+cmi+emir\]

\[cmi=N(0,m2)\]

\[emir=N(0,m2)\]

\begin{verbatim}
Carstensen page 58

var(y10-y20) =12+22+12+22

1-2222+12+22

Roy further to Carstensen

ymir=m+i+cmi+emir

\end{verbatim}
%-----------------------------------------------------------------%


Section 7 A general model for method comparisons.

Carstensen discusses the model and its use as if all parameter estimates are available.

In this model, intermethod bias is assumed to be constant at all measurement levels.

i : True value for item i

The parameter i can be thought of as the underlying, but unobtainable, true measurement for item i.

m: Fixed effect for method m

%-----------------------------------------------------------------%

\subsection{7.2 Interpretation of Random effects}

\begin{itemize}
\item method by item
\item item by replicate
\item method by item by replicate
\end{itemize}

\end{document}
