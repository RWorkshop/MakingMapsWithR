\documentclass[Chap4bmain.tex]{subfiles}
\begin{document}

\subsection{Carstensen Methods}
Bendix Carstensen et al. proposed the use of LME models to allow for a more statistically rigourous approach to computing Limits of Agreement.  The respective papers also discuss several shortcoming for techniques for dealing with replicate measurements, as proposed by Bland-Altman 1999.

%---------------------------------------------------------------%
Components

\begin{verbatim}
Carstensen’s LME model
LoA as computed by Carstensen’s LME model
Papers

Carstensen et Al 2006
Carstensen et al 2008

Bendix Carstensen 2010

Bendix Carstensen 2010
Section 5.3 Models for replicate measurements
Section 7 A general model for method comparisons.
Section 7.2 Interpretation of Random effects


Section 5.3 Models for replicate measurements
Section 5 Replicate measurements.

Carstensen page 56
%----------------------------------------------------------------%
air extra random effect that does not depend on method.
It is treated as an extension of i.
The variance of air represents the variation between replication condition (common for all methods), within items, .
\end{verbatim}
\[ymir=m+i+cmi+emir\]

\[cmi=N(0,m2)\]

\[emir=N(0,m2)\]

\begin{verbatim}
Carstensen page 58

var(y10-y20) =12+22+12+22

1-2222+12+22

Roy further to Carstensen

ymir=m+i+cmi+emir

\end{verbatim}
%-----------------------------------------------------------------%


Section 7 A general model for method comparisons.

Carstensen discusses the model and its use as if all parameter estimates are available.

In this model, intermethod bias is assumed to be constant at all measurement levels.

i : True value for item i

The parameter i can be thought of as the underlying, but unobtainable, true measurement for item i.

m: Fixed effect for method m

%-----------------------------------------------------------------%

\subsection{7.2 Interpretation of Random effects}

\begin{itemize}
\item method by item
\item item by replicate
\item method by item by replicate
\end{itemize}
%-----------------------------------------------------------------%

\end{document}
