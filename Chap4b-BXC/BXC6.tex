\newpage
\section{Carstensen's Mixed Models}

\citet{BXC2004} proposes linear mixed effects models for deriving
conversion calculations similar to Deming's regression, and for
estimating variance components for measurements by different
methods. The following model ( in the authors own notation) is
formulated as follows, where $y_{mir}$ is the $r$th replicate
measurement on subject $i$ with method $m$.

\begin{equation}
y_{mir}  = \alpha_{m} + \beta_{m}\mu_{i} + c_{mi} + e_{mir} \qquad
( e_{mi} \sim N(0,\sigma^{2}_{m}), c_{mi} \sim N(0,\tau^{2}_{m}))
\end{equation}
The intercept term $\alpha$ and the $\beta_{m}\mu_{i}$ term follow
from \citet{DunnSEME}, expressing constant and proportional bias
respectively , in the presence of a real value $\mu_{i}.$
 $c_{mi}$ is a interaction term to account for replicate, and
 $e_{mir}$ is the residual associated with each observation.
Since variances are specific to each method, this model can be
fitted separately for each method.

The above formulation doesn't require the data set to be balanced.
However, it does require a sufficient large number of replicates
and measurements to overcome the problem of identifiability. The
import of which is that more than two methods of measurement may
be required to carry out the analysis. There is also the
assumptions that observations of measurements by particular
methods are exchangeable within subjects. (Exchangeability means
that future samples from a population behaves like earlier
samples).

%\citet{BXC2004} describes the above model as a `functional model',
%similar to models described by \citet{Kimura}, but without any
%assumptions on variance ratios. A functional model is . An
%alternative to functional models is structural modelling

\citet{BXC2004} uses the above formula to predict observations for
a specific individual $i$ by method $m$;

\begin{equation}BLUP_{mir} = \hat{\alpha_{m}} + \hat{\beta_{m}}\mu_{i} +
c_{mi} \end{equation}. Under the assumption that the $\mu$s are
the true item values, this would be sufficient to estimate
parameters. When that assumption doesn't hold, regression
techniques (known as updating techniques) can be used additionally
to determine the estimates. The assumption of exchangeability can
be unrealistic in certain situations. \citet{BXC2004} provides an
amended formulation which includes an extra interaction term ($
d_{mr} \sim N(0,\omega^{2}_{m}$)to account for this.


\newpage
\citet{BXC2008} sets out a methodology of computing the limits of
agreement based upon variance component estimates derived using
linear mixed effects models. Measures of repeatability, a
characteristic of individual methods of measurements, are also
derived using this method.

\subsection{Using LME models to create Prediction Intervals}
\citet{BXC2004} also advocates the use of linear mixed models in
the study of method comparisons. The model is constructed to
describe the relationship between a value of measurement and its
real value. The non-replicate case is considered first, as it is
the context of the Bland-Altman plots. This model assumes that
inter-method bias is the only difference between the two methods.
A measurement $y_{mi}$ by method $m$ on individual $i$ is
formulated as follows;
\begin{equation}
y_{mi}  = \alpha_{m} + \mu_{i} + e_{mi} \qquad ( e_{mi} \sim
N(0,\sigma^{2}_{m}))
\end{equation}
The differences are expressed as $d_{i} = y_{1i} - y_{2i}$ For the
replicate case, an interaction term $c$ is added to the model,
with an associated variance component. All the random effects are
assumed independent, and that all replicate measurements are
assumed to be exchangeable within each method.

\begin{equation}
y_{mir}  = \alpha_{m} + \mu_{i} + c_{mi} + e_{mir} \qquad ( e_{mi}
\sim N(0,\sigma^{2}_{m}), c_{mi} \sim N(0,\tau^{2}_{m}))
\end{equation}
