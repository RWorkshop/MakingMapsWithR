\documentclass[Chap1bmain.tex]{subfiles}
\begin{document}
\citet[p.47]{DunnSEME} cautions that`gold standards' should not be
assumed to be error free. `It is of necessity a subjective
decision when we come to decide that a particular method or
instrument can be treated as if it was a gold standard'. The
clinician gold standard , the sphygmomanometer, is used as an
example thereof.  The sphygmomanometer `leaves considerable room
for improvement' \citep{DunnSEME}. \citet{pizzi} similarly
addresses the issue of glod standards, `well-established gold
standard may itself be imprecise or even unreliable'.


The NIST F1 Caesium fountain atomic clock is considered to be the
gold standard when measuring time, and is the primary time and
frequency standard for the United States. The NIST F1 is accurate
to within one second per 60 million years \citep{NIST}.

Measurements of the interior of the human body are, by definition,
invasive medical procedures. The design of method must balance the
need for accuracy of measurement with the well-being of the
patient. This will inevitably lead to the measurement error as
described by \citet{DunnSEME}. The magnetic resonance angiogram,
used to measure internal anatomy,  is considered to the gold
standard for measuring aortic dissection. Medical test based upon
the angiogram is reported to have a false positive reporting rate
of 5\% and a false negative reporting rate of 8\%. This is
reported as sensitivity of 95\% and a specificity of 92\%
\citep{ACR}.

In literature they are, perhaps more accurately, referred to as
`fuzzy gold standards' \citep{phelps}. Consequently when one of the methods is
essentially a fuzzy gold standard, as opposed to a `true' gold
standard, the comparison of the criterion and test methods should
be consider in the context of a comparison study, as well as of a
calibration study.


\end{document}
