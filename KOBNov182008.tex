\documentclass{article}
\usepackage{amsmath}
\usepackage{amssymb}
%\usepackage{amsfont}
\usepackage{graphicx}
\begin{document}

Kevin O'Brien 18 November 2008

\section*{Multiple Comparison Procedures.}




\subsection*{Classification of Hypotheses }

\begin{tabular}{ c | c | c | c}
\hline
  & declared non-significant&  declared significant & Total   \\\hline

  true null hypotheses & U & V & m.o \\ \hline

  non-true null hypotheses & T & S &  m-m.o \\ \hline

  Total & m-R & R & m \\\hline

\hline\end{tabular}
\subsubsection*{Legend }
 \begin{list}{}
 \item m.0 is the number of true null hypotheses
 \item m - m.0 is the number of false null hypotheses
 \item V is the number of Type I errors (hypotheses declared significant
when they are actually from the null distribution).
 \item T is the number of Type II errors (hypotheses declared not significant when
they are actually from the alternative distribution).
 \item  R is an observable random variable.
 \item  S, T , U, and V are unobservable random variables.
 \begin{list}{}
 \item U is the number of true negatives
 \item V is the number of false positives
 \item T is the number of false negatives
 \item S is the number of true positives
 \end{list}.

  \end{list}.

\subsection*{Key Definitions of Hypotheses }
\subsubsection*{Familywise Error Rate(FWER) }
Familywise error rate (FWER) is the probability of making one or
more type I errors among all the hypotheses when performing
multiple pairwise tests. Further to the classification table
previously we define it as follows:

\[ FWER = 1- P(V = 0 ) \]

\subsubsection*{False Discovery Rate(FDR) }
The False Discovery Rate (FDR) of a set of hypotheses is the
expected proportions of false positives in the set of hypotheses.
(Controls based on other methods are based on controlling the
chance of any false positives). FDR is more sensitive than
previous approaches on account of the fact it takes a more lenient
approach to false positives. Benjamini Hochberg define a variable
Q which is the proportion of errors committed by falsely rejected
null hypotheses, and furthermore they define the FDR as the
expected value of Q.
\[ FDR = E(Q)= E(V/R ) \]


\subsection*{Difficulties with Classical MCPs }
Benjamini and Hochberg(1995) Set out several flaws with the
classical MCP approaches. One in particular is that often control
of the FWER is not particularly needed. Another issue is that
classical procedures that control the FWER in the strong sense, at
levels convention to single-comparsion, tend to have substantially
less power than procedures based on each comparison individually,
at the same level of significance.


\subsection*{Some Approaches in Literature}

We look at Multiple comparison procedures to ascertain whether or
not it is possible to examine the methodologies described in the
relevant academic papers.


 \begin{list}{}
 \item Hochberg (1998)
 \item Simes (1986)
 \item Hommel (1988)
 \end{list}

Benjamini Hochberg (1995) state that these procedures overcome the
difficulties mentioned previously, while maintaining use of FWER
control. This paper proposes the FDR as an alternative to the
FWER.

\subsection*{Bonferroni Procedure }
In several of the papers, the Bonferroni correction is cited as
'very conservative'.It is also referred to as the classical
approach.Benjamini remarks  hat the Bonferroni procedure requires
control over the FWER in the strong sense, a conservative type I
error rate control against any configuration of the hypotheses
tested.


\subsection*{Simes(1986) }
Simes proposed a modified Bonferroni procedure for a test of an
overall hypothesis, which a composite of several component
hypotheses.He bases his method on ordered p-values of individual
tests (P(1)=...=P(N)). \\Following the closure principle, however,
it may be used as a step-down procedure as a multiple testing
procedureIt is a less conservative procedure than the classical
Bonferroni procedure. It is suitable for testing multiple
hypotheses testing where the hypotheses are strongly correlated.
it is comparatively simpler to apply.

\subsection*{Hommel(1988) - Multiple Test Procedure}
Further to Simes (1986), Hommel proposes a multiple test procedure
based on the individual hypotheses. He remarks that it is not
strictly less powerful than Holm's procedure , and that in many
cases it is more powerful.

\subsection*{Shaffer(1986)}
Hommel remarks that Shaffer (1976) makes improvements on Holms
General procedure.

\subsection*{Principle of Closed Test Procedures}.
This principles , which was postulated by Marcus et al (1976)

\subsection*{Coherence}
Coherence is an logical property for multiple tests postulated by
Gabriel (1969). It states that if a hypothesis is retained, all of
its implications also have to be retained.

\section*{Other Matters.}
\subsection*{Bayesian computation with R - Jim Alberts}
Further to your study group proposal , I have begun working on the
Alberts book , concentrating on the first five chapter.
\subsection*{Statistical events in 2009}
further to the conversation in the Castletroy hotel, I have made
some provisional arrangements for the conference in Vancouver in
May 2009. Is there any other events you suggest? I am considering
going to the R users Conference in Rennes , France in July 2009.

\subsection*{Literary Review}
Would you suggest I also start writing a formal literary review.


\end{document}
