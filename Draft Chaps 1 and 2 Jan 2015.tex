\documentclass[12pt, a4paper]{report}
\usepackage{natbib}
\usepackage{vmargin}
\usepackage{graphicx}
\usepackage{epsfig}
\usepackage{subfigure}
%\usepackage{amscd}
\usepackage{amssymb}
\usepackage{subfigure}
\usepackage{amsbsy}
\usepackage{amsthm, amsmath}
%\usepackage[dvips]{graphicx}
\bibliographystyle{chicago}
\renewcommand{\baselinestretch}{1.8}

% left top textwidth textheight headheight % headsep footheight footskip
\setmargins{3.0cm}{2.5cm}{15.5 cm}{23.5cm}{0.5cm}{0cm}{1cm}{1cm}

\pagenumbering{arabic}


\begin{document}
	\author{Kevin O'Brien}
	\title{Print Ready Draft for Chaps 1 and 2}
	\date{\today}
	\maketitle
	
	\tableofcontents \setcounter{tocdepth}{2}
	
	\newpage

	\begin{itemize}
	\item \textbf{Working Title:} Linear Mixed Effects Model Diagnostics applied to Method Comparison Studies.\\
	 Fundamental Areas introduced in order
	\begin{itemize}
		\item[(1)] Method Comparison Studies
		\item[(2)] Linear Mixed Effects Models
		\item[(3)] Model Diagnostics for Lnear and LME models
	\end{itemize}
	\item \textbf{Chapter 1} : Introduction to Method Comparison Studies (10 pages approx)
    \item \textbf{Chapter 2} : Overview of existing methdologies	
    	\begin{itemize}
    		\item Grubb's Formulation
    		\item Formal Methods (e.g. Pitman-Morgan)
    		\item Bland Altman Methods
    		\item Regressions Models (incl Deming Regression and Bartko's Method)
    		\item LME models (Carstensen)
    		\item Roy's Tests for MCS
    		\item R implementation
	\end{itemize}
\end{itemize}
\newpage
	\chapter{Introduction to Method Comparison Studies}
		\section{Introduction}
	The problem of assessing the agreement between two or more methods
	of measurement is ubiquitous in scientific research, and is
	commonly referred to as a `method comparison study'. Published
	examples of method comparison studies can be found in disciplines
	as diverse as pharmacology \citep{ludbrook97}, anaesthesia
	\citep{Myles}, and cardiac imaging methods \citep{Krumm}.
	\smallskip
	
	To illustrate the characteristics of a typical method comparison
	study consider the data in Table I \citep{Grubbs73}. In each of
	twelve experimental trials, a single round of ammunition was fired
	from a 155mm gun and its velocity was measured simultaneously (and
	independently) by three chronographs devices, identified here by
	the labels `Fotobalk', `Counter' and `Terma'.
	\smallskip
	
	
	\newpage
	
	\begin{table}[ht]
		\begin{center}
			\begin{tabular}{rrrr}
				\hline
				Round& Fotobalk [F] & Counter [C]& Terma [T]\\
				\hline
				1 & 793.8 & 794.6 & 793.2 \\
				2 & 793.1 & 793.9 & 793.3 \\
				3 & 792.4 & 793.2 & 792.6 \\
				4 & 794.0 & 794.0 & 793.8 \\
				5 & 791.4 & 792.2 & 791.6 \\
				6 & 792.4 & 793.1 & 791.6 \\
				7 & 791.7 & 792.4 & 791.6 \\
				8 & 792.3 & 792.8 & 792.4 \\
				9 & 789.6 & 790.2 & 788.5 \\
				10 & 794.4 & 795.0 & 794.7 \\
				11 & 790.9 & 791.6 & 791.3 \\
				12 & 793.5 & 793.8 & 793.5 \\
				\hline
			\end{tabular}
			\caption{Velocity measurement from the three chronographs (Grubbs
				1973).}
		\end{center}
	\end{table}
	
	An important aspect of the these data is that all three methods of
	measurement are assumed to have an attended measurement error, and
	the velocities reported in Table 1.1 can not be assumed to be
	`true values' in any absolute sense.
	
	%While lack of
	%agreement between two methods is inevitable, the question , as
	%posed by \citet{BA83}, is 'do the two methods of measurement agree
	%sufficiently closely?'
	
	A method of measurement should ideally be both accurate and precise. \citet{Barnhart} describes agreement as being a broader term that contains both of those qualities. An accurate measurement method will give results close to the unknown `true value'. The precision of a method is indicated by how tightly measurements obtained under identical conditions are distributed around their mean measurement value. A precise and accurate method
	will yield results consistently close to the true value. Of course 	a method may be accurate, but not precise, if the average of its measurements is close to the true value, but those measurements
	are highly dispersed. Conversely a method that is not accurate may be quite precise, as it consistently indicates the same level of inaccuracy. The tendency of a method of measurement to consistently give results above or below the true value is a source of systematic bias. The smaller the systematic bias, the
	greater the accuracy of the method.
	
	% The FDA define precision as the closeness of agreement (degree of
	% scatter) between a series of measurements obtained from multiple
	% sampling of the same homogeneous sample under prescribed
	% conditions. \citet{Barnhart} describes precision as being further
	% subdivided as either within-run, intra-batch precision or
	% repeatability (which assesses precision during a single analytical
	% run), or between-run, inter-batch precision or repeatability
	%(which measures precision over time).
	
	In the context of the agreement of two methods, there is also a tendency of one measurement method to consistently give results above or below the other method. Lack of agreement is a consequence of the existence of `inter-method bias'. For two methods to be considered in good agreement, the inter-method bias should be in the region of zero. A simple estimation of the inter-method bias can be calculated using the differences of the
	paired measurements. The data in Table 1.2 are a good example of
	possible inter-method bias; the `Fotobalk' consistently recording
	smaller velocities than the `Counter' method. Consequently one
	would conclude that there is lack of agreement between the two
	methods.
	
	The absence of inter-method bias by itself is not sufficient to
	establish whether two measurement methods agree. The two methods
	must also have equivalent levels of precision. Should one method
	yield results considerably more variable than those of the other,
	they can not be considered to be in agreement. With this in mind a
	methodology is required that allows an analyst to estimate the
	inter-method bias, and to compare the precision of both methods of
	measurement.
	\newpage
	% latex table generated in R 2.6.0 by xtable 1.5-5 package
	% Wed Aug 26 15:22:41 2009
	\begin{table}[h!]
		
		\begin{center}
			
			\begin{tabular}{rrrr}
				\hline
				Round& Fotobalk (F) & Counter (C) & F-C \\
				\hline
				1 & 793.8& 794.6 & -0.8 \\
				2 & 793.1 & 793.9 & -0.8 \\
				3 & 792.4 & 793.2 & -0.8 \\
				4 & 794.0 & 794.0 & 0.0 \\
				5 & 791.4 & 792.2 & -0.8 \\
				6 & 792.4 & 793.1 & -0.7 \\
				7 & 791.7 & 792.4 & -0.7 \\
				8 & 792.3 & 792.8 & -0.5 \\
				9 & 789.6 & 790.2 & -0.6 \\
				10 & 794.4 & 795.0 & -0.6 \\
				11 & 790.9 & 791.6 & -0.7 \\
				12 & 793.5 & 793.8 & -0.3 \\
				\hline
			\end{tabular}
			\caption{Difference between Fotobalk and Counter measurements.}
		\end{center}
	\end{table}
	
	\bigskip
	
	\newpage
	
	\section{Bland-Altman methodology}
	The issue of whether two measurement methods comparable to the
	extent that they can be used interchangeably with sufficient
	accuracy is encountered frequently in scientific research.
	Historically comparison of two methods of measurement was carried
	out by use of paired sample $t-$test, correlation coefficients or
	simple linear regression. Simple linear regression is unsuitable for method comparison studies because of the required assumption that one variable is measured without error. In comparing two methods, both methods are assume to have attendant random error.
	
	Statisticians Martin Bland and Douglas Altman recognized the inadequacies of these analyzes and
	articulated quite thoroughly the basis on which of which they are unsuitable for comparing two methods of measurement \citep*{BA83}. Furthermore they proposed their simple methodology specifically
	constructed for method comparison studies. They acknowledge the opportunity to apply other valid, but complex, methodologies, but argue that a simple approach is preferable, especially when the
	results must be `explained to non-statisticians'.
	
	Notwithstanding previous remarks about linear regression, the first step recommended, which the authors argue should be mandatory, is construction of a simple scatter plot of the data. The line of equality should also be shown, as it is necessary to give the correct interpretation of how both methods compare. In the case of good agreement, the observations would be distributed closely along the line of equality. A scatter plot of the Grubbs data is shown in Figure 1.1. Visual inspection confirms the previous conclusion that there is an inter-method bias present, i.e. Fotobalk device has a tendency to record a lower velocity.
	
	\begin{figure}[h!]
		\begin{center}
			\includegraphics[width=125mm]{GrubbsScatter.jpeg}
			\caption{Scatter plot For Fotobalk and Counter Methods.}\label{GrubbsScatter}
		\end{center}
	\end{figure}
	
	\citet{Dewitte} notes that scatter plots were very seldom
	presented in the Annals of Clinical Biochemistry. This apparently
	results from the fact that the `Instructions for Authors' dissuade
	the use of regression analysis, which conventionally is
	accompanied by a scatter plot.
	
	\newpage
	\subsection{Bland-Altman plots}
	
	In light of shortcomings associated with scatterplots,
	\citet*{BA83} recommend a further analysis of the data. Firstly
	case-wise differences of measurements of two methods $d_{i} =
	y_{1i}-y_{2i} \mbox{ for }i=1,2,\dots,n$ on the same subject
	should be calculated, and then the average of those measurements
	($a_{i} = (y_{1i} + y_{2i})/2 \mbox{ for }i=1,2,\dots, n$).
	
	\citet{BA83} proposes a scatterplot of the case-wise averages and differences of two methods of measurement. This scatterplot has since become widely known as the Bland-Altman plot. \citet*{BA83} express the
	motivation for this plot thusly:
	\begin{quote}
		``From this type of plot it is much easier to assess the magnitude
		of disagreement (both error and bias), spot outliers, and see
		whether there is any trend, for example an increase in (difference) for high values. This way of plotting the data is a very powerful way of displaying the results of a method comparison study."
	\end{quote}
	
	The case wise-averages capture several aspects of the data, such as expressing the range over which the values were taken, and assessing whether the assumptions of constant variance holds.
	Case-wise averages also allow the case-wise differences to be presented on a two-dimensional plot, with better data visualization qualities than a one dimensional plot. \citet{BA86}
	cautions that it would be the difference against either measurement value instead of their average, as the difference relates to both value. This methodology has proved very popular, and the Bland-Altman plots is widely regarded as powerful graphical methodology for making a visual assessment of the data.
	
	The magnitude of the inter-method bias between the two methods is simply the average of the differences $\bar{d}$. This inter-method bias is represented with a line on the Bland-Altman plot. As the objective of the Bland-Altman plot is to advise on the agreement of two methods, it is the case-wise differences that are also particularly relevant. The variances around this bias is estimated by the standard deviation of these differences $S_{d}$.
	
	\subsection{Bland-Altman plots for the Grubbs data}
	
	In the case of the Grubbs data the inter-method bias is $-0.61$ metres per second, and is indicated by the dashed line on Figure 1.2. By inspection of the plot, it is also possible to compare the precision of each method. Noticeably the differences tend to increase as the averages increase.
	
	
	The Bland-Altman plot for comparing the `Fotobalk' and `Counter'
	methods, which shall henceforth be referred to as the `F vs C'
	comparison,  is depicted in Figure 1.2, using data from Table 1.3.
	The presence and magnitude of the inter-method bias is indicated
	by the dashed line.
	\newpage
	
	%Later it will be shown that case-wise differences are the sole
	%component of the next part of the methodology, the limits of
	%agreement.
	
	
	\begin{table}[h!]
		\renewcommand\arraystretch{0.7}%
		\begin{center}
			\begin{tabular}{|c||c|c||c|c|}
				\hline
				Round & Fotobalk  & Counter  & Differences  & Averages  \\
				&  [F] & [C] & [F-C] &  [(F+C)/2] \\
				\hline
				1 & 793.8 & 794.6 & -0.8 & 794.2 \\
				2 & 793.1 & 793.9 & -0.8 & 793.5 \\
				3 & 792.4 & 793.2 & -0.8 & 792.8 \\
				4 & 794.0 & 794.0 & 0.0 & 794.0 \\
				5 & 791.4 & 792.2 & -0.8 & 791.8 \\
				6 & 792.4 & 793.1 & -0.7 & 792.8 \\
				7 & 791.7 & 792.4 & -0.7 & 792.0 \\
				8 & 792.3 & 792.8 & -0.5 & 792.5 \\
				9 & 789.6 & 790.2 & -0.6 & 789.9 \\
				10 & 794.4 & 795.0 & -0.6 & 794.7 \\
				11 & 790.9 & 791.6 & -0.7 & 791.2 \\
				12 & 793.5 & 793.8 & -0.3 & 793.6 \\
				\hline
			\end{tabular}
			\caption{Fotobalk and Counter methods: differences and averages.}
		\end{center}
	\end{table}
	
	\begin{table}[h!]
		\renewcommand\arraystretch{0.7}%
		\begin{center}
			\begin{tabular}{|c||c|c||c|c|}
				\hline
				Round & Fotobalk  & Terma  & Differences  & Averages  \\
				&  [F] & [T] & [F-T] &  [(F+T)/2] \\
				\hline
				1 & 793.8 & 793.2 & 0.6 & 793.5 \\
				2 & 793.1 & 793.3 & -0.2 & 793.2 \\
				3 & 792.4 & 792.6 & -0.2 & 792.5 \\
				4 & 794.0 & 793.8 & 0.2 & 793.9 \\
				5 & 791.4 & 791.6 & -0.2 & 791.5 \\
				6 & 792.4& 791.6 & 0.8 & 792.0 \\
				7 & 791.7 & 791.6 & 0.1 & 791.6 \\
				8 & 792.3 & 792.4 & -0.1 & 792.3 \\
				9 & 789.6 & 788.5 & 1.1 & 789.0 \\
				10 & 794.4 & 794.7 & -0.3 & 794.5 \\
				11 & 790.9 & 791.3 & -0.4 & 791.1 \\
				12 & 793.5 & 793.5 & 0.0 & 793.5 \\
				
				\hline
			\end{tabular}
			\caption{Fotobalk and Terma methods: differences and averages.}
		\end{center}
	\end{table}
	
	\newpage
	
	\begin{figure}[h!]
		\begin{center}
			\includegraphics[width=120mm]{GrubbsBAplot-noLOA.jpeg}
			\caption{Bland-Altman plot For Fotobalk and Counter methods.}\label{GrubbsBA-noLOA}
		\end{center}
	\end{figure}
	
	
	
	In Figure 1.3 Bland-Altman plots for the `F vs C' and `F vs T'
	comparisons are shown, where `F vs T' refers to the comparison of
	the `Fotobalk' and `Terma' methods. Usage of the Bland-Altman plot
	can be demonstrate in the contrast between these comparisons. By inspection, there exists a larger inter-method bias in the `F vs C' comparison than in the `F vs T' comparison. Conversely there
	appears to be less precision in `F vs T' comparison, as indicated
	by the greater dispersion of covariates.
	
	\begin{figure}[h!]
		\begin{center}
			\includegraphics[height=90mm]{GrubbsDataTwoBAplots.jpeg}
			\caption{Bland-Altman plots for Grubbs' F vs C and F vs T comparisons.}\label{GrubbsDataTwoBAplots}
		\end{center}
	\end{figure}
	
	\newpage
	
	\subsection{Prevalence of the Bland-Altman plot}
	\citet*{BA86}, which further develops the Bland-Altman methodology,
	was found to be the sixth most cited paper of all time by the
	\citet{BAcite}. \cite{Dewitte} describes the rate at which
	prevalence of the Bland-Altman plot has developed in scientific
	literature. \citet{Dewitte} reviewed the use of Bland-Altman plots
	by examining all articles in the journal `Clinical Chemistry'
	between 1995 and 2001. This study concluded that use of the
	Bland�Altman plot increased over the years, from 8\% in 1995 to
	14\% in 1996, and 31�36\% in 2002.
	
	The Bland-Altman Plot has since become expected, and
	often obligatory, approach for presenting method comparison
	studies in many scientific journals \citep{hollis}. Furthermore
	\citet{BritHypSoc} recommend its use in papers pertaining to
	method comparison studies for the journal of the British
	Hypertension Society.
	
	\subsection{Adverse features}
	
	Estimates for inter-method bias and variance of differences are only meaningful if there is uniform inter-bias and variability throughout the range of measurements. Fulfilment of these assumptions can be checked by visual inspection of the plot.The prototype Bland-Altman plots depicted in Figures 1.4, 1.5 and 1.6 are derived from simulated data, for the purpose of demonstrating how the plot would inform an analyst of features that would adversely affect use of the recommended methodology.
	
	Figure 1.4 demonstrates how the Bland-Altman plot would indicate
	increasing variance of differences over the measurement range.
	Fitted regression lines, for both the upper and lower half of the
	plot, has been added to indicate the trend. Figure 1.5 is an
	example of cases where the inter-method bias changes over the
	measurement range. This is known as proportional bias, and is
	defined by \citet{ludbrook97} as meaning that `one method gives
	values that are higher (or lower) than those from the other by an
	amount that is proportional to the level of the measured
	variable'. In both Figures 1.4 and 1.5, the assumptions necessary
	for further analysis using the limits of agreement are violated.
	
	Application of regression techniques to the Bland-Altman plot, and
	subsequent formal testing for the constant variability of
	differences is informative. The data set may be divided into two
	subsets, containing the observations wherein the difference values
	are less than and greater than the inter-method bias respectively.
	For both of these fits, hypothesis tests for the respective slopes
	can be performed. While both tests can be considered separately,
	multiple comparison procedures, such as the Benjamini-Hochberg
	\citep{BH} test, should be also be used.
	
	\begin{figure}[h!]
		\begin{center}
			\includegraphics[height=90mm]{BAFanEffect.jpeg}
			\caption{Bland-Altman plot demonstrating the increase of variance over the range.}\label{BAFanEffect}
		\end{center}
	\end{figure}
	
	\begin{figure}[h!]
		\begin{center}
			\includegraphics[height=90mm]{PropBias.jpeg}
			\caption{Bland-Altman plot indicating the presence of proportional bias.}\label{PropBias}
		\end{center}
	\end{figure}
	
	\begin{figure}[h!]
		\begin{center}
			\includegraphics[width=125mm]{BAOutliers.jpeg}
			\caption{Bland-Altman plot indicating the presence of potential outliers.}\label{Outliers}
		\end{center}
	\end{figure}
	
	\newpage
	
	
	The Bland-Altman plot also can be used to identify outliers. An
	outlier is an observation that is conspicuously different from the
	rest of the data that it arouses suspicion that it occurs due to a
	mechanism, or conditions, different to that of the rest of the
	observations. \citet*{BA99} do not recommend excluding outliers from analyzes,
	but remark that recalculation of the inter-method bias estimate,
	and further calculations based upon that estimate, are useful for
	assessing the influence of outliers. The authors remark that `we
	usually find that this method of analysis is not too sensitive to
	one or two large outlying differences'. Figure 1.6 demonstrates how the Bland-Altman
	plot can be used to visually inspect the presence of potential
	outliers.
	
	As a complement to the Bland-Altman plot, \citet{Bartko} proposes
	the use of a bivariate confidence ellipse, constructed for a
	predetermined level. \citet{AltmanEllipse} provides the relevant calculations for the
	ellipse. This ellipse is intended as a visual
	guidelines for the scatter plot, for detecting outliers and to
	assess the within- and between-subject variances.
	
	The minor axis relates to the between subject variability, whereas
	the major axis relates to the error mean square, with the ellipse
	depicting the size of both relative to each other.
	Consequently Bartko's ellipse provides a visual aid to determining the
	relationship between variances. If $\mbox{var}(a)$ is greater than $\mbox{var}(d)$, the orientation of the ellipse is horizontal. Conversely if $\mbox{var}(a)$ is less than $\mbox{var}(d)$, the orientation of the ellipse is vertical.
	
	
	%(Furthermore \citet{Bartko}
	%proposes formal testing procedures, that shall be discussed in due
	%course.)
	
	The Bland-Altman plot for the Grubbs data, complemented by Bartko's ellipse, is depicted in Figure 1.7.
	The fourth observation is shown to be outside the bounds of the ellipse, indicating that it is a potential outlier.
	
	
	\begin{figure}[h!]
		% Requires \usepackage{graphicx}
		\includegraphics[width=130mm]{GrubbsBartko.jpeg}
		\caption{Bartko's Ellipse For Grubbs' Data.}\label{GrubbsBartko}
	\end{figure}
	
	The limitations of using bivariate approaches to outlier detection
	in the Bland-Altman plot can demonstrated using Bartko's ellipse.
	A covariate is added to the `F vs C' comparison that has a
	difference value equal to the inter-method bias, and an average
	value that markedly deviates from the rest of the average values
	in the comparison, i.e. 786. Table 1.8 depicts a $95\%$ confidence
	ellipse for this manipulated data set. By inspection of the
	confidence interval, a conclusion would be reached that this extra
	covariate is an outlier, in spite of the fact that this
	observation is wholly consistent with the conclusion of the
	Bland-Altman plot.
	
	\begin{figure}[h!]
		% Requires \usepackage{graphicx}
		\includegraphics[width=130mm]{GrubbsBartko2.jpeg}
		\caption{Bartko's Ellipse For Grubbs' Data, with an extra covariate.}\label{GrubbsBartko2}
	\end{figure}
	
	
	Importantly, outlier classification must be informed by the logic of the
	data's formulation. In the Bland-Altman plot, the horizontal displacement of any
	observation is supported by two independent measurements. Any
	observation should not be considered an outlier on the basis of a
	noticeable horizontal displacement from the main cluster, as in
	the case with the extra covariate. Conversely, the fourth
	observation, from the original data set, should be considered an
	outlier, as it has a noticeable vertical displacement from the
	rest of the observations.
	
	%Grubbs' test is a statistical test used for detecting outliers in a
	%univariate data set that is assumed to be normally distributed.
	
	%\citet{Grubbs} defined an outlier as a co-variate that appears to
	%deviate markedly from other members of the sample in which it
	%occurs.
	
	In classifying whether a observation from a univariate data set is
	an outlier, many formal tests are available, such as the Grubbs test for outliers. In assessing
	whether a covariate in a Bland-Altman plot is an outlier, this
	test is useful when applied to the case-wise difference values treated as a
	univariate data set. The null hypothesis of the Grubbs test procedure is the absence
	of any outliers in the data set. Conversely, the alternative hypotheses is that there is at least one outlier
	present.
	
	The test statistic for the Grubbs test ($G$) is the largest
	absolute deviation from the sample mean divided by the standard
	deviation of the differences,
	\[
	G =  \displaystyle\max_{i=1,\ldots, n}\frac{\left \vert d_i -
		\bar{d}\right\vert}{S_{d}}.
	\]
	
	For the `F vs C' comparison it is the fourth observation gives
	rise to the test statistic, $G = 3.64$. The critical value is
	calculated using Student's $t$ distribution and the sample size,
	\[
	U = \frac{n-1}{\sqrt{n}} \sqrt{\frac{t_{\alpha/(2n),n-2}^2}{n - 2
			+ t_{\alpha/(2n),n-2}^2}}.
	\]
	For this test $U = 0.75$. The conclusion of this test is that the fourth observation in the `F vs C' comparison is an outlier, with $p-$value = 0.003, according with the previous result using Bartko's ellipse.
	
	\newpage
	
	
	
	\subsection{Inferences on Bland-Altman estimates}
	\citet*{BA99}advises on how to calculate confidence intervals for the inter-method bias and limits of agreement.
	For the inter-method bias, the confidence interval is a simply that of a mean: $\bar{d} \pm t_{(0.5\alpha,n-1)} S_{d}/\sqrt{n}$.
	The confidence
	intervals and standard error for the limits of agreement follow from the variance of the limits of agreement, which is shown to be
	
	\[
	\mbox{Var}(LoA) = (\frac{1}{n}+\frac{1.96^{2}}{2(n-1)})s_{d}^{2}.
	\]
	
	If $n$ is sufficiently large this can be following approximation
	can be used
	\[
	\mbox{Var}(LoA) \approx 1.71^{2}\frac{s_{d}^{2}}{n}.
	\]
	Consequently the standard errors of both limits can be
	approximated as $1.71$ times the standard error of the
	differences.
	
	A $95\%$ confidence interval can be determined, by means of the
	\emph{t} distribution with $n-1$ degrees of freedom. However \citet*{BA99} comment that such calculations  may be `somewhat optimistic' on account of the associated assumptions not being realized.
	
	%\subsubsection{Small Sample Sizes} The limits of agreement are
	%estimates derived from the sample studied, and will differ from
	%values relevant to the whole population, hence the importance of a
	%suitably large sample size. A different sample would give
	%different limits of agreement. Student's t-distribution is a well
	%known probability distribution used in statistical inference for
	%normally distributed populations when the sample size is small
	%\citep{student,Fisher3}. Consequently, using 't' quantiles , as
	%opposed to standard normal quantiles, may give a more appropriate
	%calculation for limits of agreement when the sample size is small.
	%For sample size $n=12$ the `t' quantile is 2.2 and the limits of
	%agreement are (-0.074,-1.143).
	
	
	\subsection{Formal definition of limits of agreement}
	\citet{BA99} note the similarity of limits of agreement to
	confidence intervals, but are clear that they are not the same
	thing. Interestingly, they describe the limits as `being like a
	reference interval'.
	
	Limits of agreement have very similar construction to Shewhart
	control limits. The Shewhart chart is a well known graphical
	methodology used in statistical process control. Consequently
	there is potential for misinterpreting the limits of agreement as
	if equivalent to Shewhart control limits. Importantly the
	parameters used to determine the Shewhart limits are not based on any sample used for an analysis, but
	on the process's historical values, a key difference with
	Bland-Altman limits of agreement.
	
	\citet{BXC2008} regards the limits of agreement as a prediction
	interval for the difference between future measurements with the
	two methods on a new individual, but states that it does not fit
	the formal definition of a prediction interval, since the
	definition does not consider the errors in estimation of the
	parameters. Prediction intervals, which are often used in
	regression analysis, are estimates of an interval in which future
	observations will fall, with a certain probability, given what has
	already been observed. \citet{BXC2008} offers an alternative
	formulation, a $95\%$ prediction interval for the difference
	\[
	\bar{d} \pm t_{(0.975, n-1)}s_{d} \sqrt{1+\frac{1}{n}}
	\]
	
	\noindent where $n$ is the number of subjects. Carstensen is
	careful to consider the effect of the sample size on the interval
	width, adding that only for 61 or more subjects is there a
	quantile less than 2.
	
	\citet{luiz} offers an alternative description of limits of
	agreement, this time as tolerance limits. A tolerance interval for
	a measured quantity is the interval in which a specified fraction
	of the population's values lie, with a specified level of
	confidence. \citet{Barnhart} describes them as a probability
	interval, and offers a clear description of how they should be
	used;`if the absolute limit is less than an acceptable difference
	$d_{0}$, then the agreement between the two methods is deemed
	satisfactory'.
	
	The prevalence of contradictory definitions of what limits of agreement strictly are will inevitably attenuate the poor standard of reporting using limits of agreement, as mentioned by \citet{mantha}.
	
	%At least 100 historical
	%values must be used to determine the acceptable value (i.e the
	%process mean) and the process standard deviation. The principle
	%that the mean and variance of a large sample of a homogeneous
	%population is a close approximation of the population's mean and
	%variance justifies this.
	
	%\begin{figure}[h!]
	%\begin{center}
	%  \includegraphics[width=125mm]{GrubbsLOAwCIs.jpeg}
	%  \caption{Limits of agreement with confidence intervals}\label{LOAwCIs}
	%\end{center}
	%\end{figure}
	
	%\newpage
	%\section{Agreement Indices}
	%\citet{Barnhart} provided an overview of several agreement
	%indices, including the limits of agreement. Other approaches, such
	%as mean squared deviation, the tolerance deviation index and
	%coverage probability are also discussed.
	
	
	
	
	\subsection{Replicate Measurements}
	
	Thus far, the formulation for comparison of two measurement
	methods is one where one measurement by each method is taken on	each subject. Should there be two or more measurements by each methods, these measurement are known as `replicate measurements'.
	\citet{BXC2008} recommends the use of replicate measurements, but acknowledges the additional computational complexity.
	
	\citet*{BA86} address this problem by offering two different
	approaches. The premise of the first approach is that replicate
	measurements can be treated as independent measurements. The
	second approach is based upon using the mean of the each group of
	replicates as a representative value of that group. Using either
	of these approaches will allow an analyst to estimate the inter
	method bias.
	
	%\subsubsection{Mean of Replicates Limits of Agreement}
	
	However, because of the removal of the effects of the replicate
	measurements error, this would cause the estimation of the
	standard deviation of the differences to be unduly small.
	\citet*{BA86} propose a correction for this.
	
	\citet{BXC2008} takes issue with the limits of agreement based on
	mean values of replicate measurements, in that they can only be interpreted as prediction
	limits for difference between means of repeated measurements by
	both methods, as opposed to the difference of all measurements.
	Incorrect conclusions would be caused by such a misinterpretation.
	\citet{BXC2008} demonstrates how the limits of agreement
	calculated using the mean of replicates are `much too narrow as
	prediction limits for differences between future single
	measurements'. This paper also comments that, while treating the
	replicate measurements as independent will cause a downward bias
	on the limits of agreement calculation, this method is preferable
	to the `mean of replicates' approach.
	
	


	\section{Outline of Thesis}
	Thus the study of method comparison is introduced. The intention of this thesis is to progress the
	study of method comparison studies, using a statistical method known as Linear mixed effects models.
	Chapter two shall describe linear mixed effects models, and how the use of the linear mixed
	effects models have so far extended to method comparison studies. Implementations of important existing work shall be presented, using the \texttt{R} programming language.
	
	Model diagnostics are an integral component of a complete statistical analysis.
	In chapter three model diagnostics shall be described in depth, with particular
	emphasis on linear mixed effects models, further to chapter two.
	
	For the fourth chapter, important linear mixed effects model diagnostic methods shall be extended to method comparison studies, and proposed methods shall be demonstrated on data sets that have become well known in literature on method comparison. The purpose is to both calibrate these methods and to demonstrate applications for them.
	The last chapter shall focus on robust measures of important parameters such as agreement.
	
\end{document}