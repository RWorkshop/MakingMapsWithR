\documentclass[12pt, a4paper]{article}
\usepackage{epsfig}
\usepackage{subfigure}
%\usepackage{amscd}
\usepackage{amssymb}
\usepackage{amsbsy}
\usepackage{amsthm, amsmath}
%\usepackage[dvips]{graphicx}
\usepackage{natbib}
\bibliographystyle{chicago}
\usepackage{vmargin}
% left top textwidth textheight headheight
% headsep footheight footskip
\setmargins{3.0cm}{2.5cm}{15.5 cm}{22cm}{0.5cm}{0cm}{1cm}{1cm}
\renewcommand{\baselinestretch}{1.5}
\pagenumbering{arabic}
\theoremstyle{plain}
\newtheorem{theorem}{Theorem}[section]
\newtheorem{corollary}[theorem]{Corollary}
\newtheorem{ill}[theorem]{Example}
\newtheorem{lemma}[theorem]{Lemma}
\newtheorem{proposition}[theorem]{Proposition}
\newtheorem{conjecture}[theorem]{Conjecture}
\newtheorem{axiom}{Axiom}
\theoremstyle{definition}
\newtheorem{definition}{Definition}[section]
\newtheorem{notation}{Notation}
\theoremstyle{remark}
\newtheorem{remark}{Remark}[section]
\newtheorem{example}{Example}[section]
\renewcommand{\thenotation}{}
\renewcommand{\thetable}{\thesection.\arabic{table}}
\renewcommand{\thefigure}{\thesection.\arabic{figure}}
\title{Research notes: linear mixed effects models}
\author{ } \date{ }


\begin{document}
\author{Kevin O'Brien}
\title{Updating techniques for LME models}

\addcontentsline{toc}{section}{Bibliography}

%-----------------------------------------------------------------------------------------%


Let $y_{mir} $ be the $r$th replicate measurement on the $i$th item by the $m$th method, where $m=1,2,$ $i=1,\ldots,N,$ and $r = 1,\ldots,n_i.$ When the design is balanced and there is no ambiguity we can set $n_i=n.$ The LME model underpinning Roy's approach can be written
\begin{equation}
y_{mir} = \beta_{0} + \beta_{m} + b_{mi} + \epsilon_{mir}.
\end{equation}
Here $\beta_0$ and $\beta_m$ are fixed-effect terms representing, respectively, a model intercept and an overall effect for method $m.$
The $\beta$ terms can be gathered together into (fixed effect) intercept terms $\alpha_m=\beta_0+\beta_m.$ The $b_{1i}$ and $b_{2i}$ terms are correlated random effect parameters having $\mathrm{E}(b_{mi})=0$ with $\mathrm{Var}(b_{mi})=g^2_m$ and $\mathrm{Cov}(b_{mi}, b_{m^\prime i})=g_{12}.$ The random error term for each response is denoted $\epsilon_{mir}$ having $\mathrm{E}(\epsilon_{mir})=0$, $\mathrm{Var}(\epsilon_{mir})=\sigma^2_m$, $\mathrm{Cov}(b_{mir}, b_{m^\prime ir})=\sigma_{12}$, $\mathrm{Cov}(\epsilon_{mir}, \epsilon_{mir^\prime})= 0$ and $\mathrm{Cov}(\epsilon_{mir}, \epsilon_{m^\prime ir^\prime})= 0.$ Two methods of measurement are in complete agreement if the null hypotheses $\mathrm{H}_1\colon \beta_1 = \beta_2$ and $\mathrm{H}_2\colon \sigma^2_1 = \sigma^2_2 $ and $\mathrm{H}_3\colon g^2_1= g^2_2$ hold simultaneously. \citet{roy} proposes a Bonferroni correction to control the familywise error rate for tests of $\{\mathrm{H}_1, \mathrm{H}_2, \mathrm{H}_3\}$ and account for difficulties arising due to multiple testing. Let $\omega^2_m = \sigma^2_m + g^2_m$ represent the overall variability of method $m.$  Roy also integrates $\mathrm{H}_2$ and $\mathrm{H}_3$ into a single testable hypothesis $\mathrm{H}_4\colon \omega^2_1=\omega^2_2.$ CONCERNS?

\bigskip

% Complete paragraph by specifying variances and covariances for epsilons.
% I thing that these are your sigmas?
% Also, state equality of the parameters in this model when each of the three hypotheses above are true.
\citet{Roy} demonstrates how to implement a method comparison study further to model (1) using the SAS proc mixed package.
%------------------------------------------------------------------------------------------------%
\citet{BXC2008} demonstrates how to construct limits of agreement using SAS, STATA and R. In the case of SAS, the PROC MIXED procedure is used.
Implementation in R is performed using the nlme package \citep{pb2000}.

\citet{BXC2008} remarks that the implementation using R is quite ``arcane".

As R is freely available, this paper demonstrates an implementation of Roy's model using R.

The R statistical software package is freely available.

%------------------------------------------------------------------------------------------------%
The LME model is very easy to implement using PROC MIXED of SAS and the results are also easy to interpret.
The SAS proc mixed procedure has very simple syntax.

As the required code to fit the models is complex, R code necessary to fit the models is provided. 

A demonstration is provided on how to use the output to perform the tests, and to compute limits of agreement.



We assume the data are formatted as a dataset with four columns named:

	meth, method of measurement, the number of methods being M,
	item, items (persons, samples) measured by each method, of which there are I,
	repl, replicate indicating repeated measurement of the same item by the same method, and
	y, the measurement.





\newpage
\section{Regression Of Differences On Averages}
Further to Carstensen, we can formulate the two measurements
$y_{1}$ and $y_{2}$ as follows:
\\
$y_{1} = \alpha + \beta\mu + \epsilon_{1}$
\\
$y_{2} = \alpha + \beta\mu + \epsilon_{2}$









\newpage
\subsection{Remarks on the Multivariate Normal Distribution}

Diligence is required when considering the models. Carstensen specifies his models in terms of the univariate normal distribution. Roy's model is specified using the bivariate normal distribution.
This gives rises to a key difference between the two model, in that a bivariate model accounts for covariance between the variables of interest.
The multivariate normal distribution of a $k$-dimensional random vector $X = [X_1, X_2, \ldots, X_k]$
can be written in the following notation:
\[
    X\ \sim\ \mathcal{N}(\mu,\, \Sigma),
\]
or to make it explicitly known that $X$ is $k$-dimensional,
\[
    X\ \sim\ \mathcal{N}_k(\mu,\, \Sigma).
\]
with $k$-dimensional mean vector
\[ \mu = [ \operatorname{E}[X_1], \operatorname{E}[X_2], \ldots, \operatorname{E}[X_k]] \]
and $k \times k$ covariance matrix
\[ \Sigma = [\operatorname{Cov}[X_i, X_j]], \; i=1,2,\ldots,k; \; j=1,2,\ldots,k \]

\bigskip

\begin{enumerate}
\item Univariate Normal Distribution

\[
    X\ \sim\ \mathcal{N}(\mu,\, \sigma^2),
\]

\item Bivariate Normal Distribution

\begin{itemize}
\item[(a)] \[  X\ \sim\ \mathcal{N}_2(\mu,\, \Sigma), \vspace{1cm}\]
\item[(b)] \[    \mu = \begin{pmatrix} \mu_x \\ \mu_y \end{pmatrix}, \quad
    \Sigma = \begin{pmatrix} \sigma_x^2 & \rho \sigma_x \sigma_y \\
                             \rho \sigma_x \sigma_y  & \sigma_y^2 \end{pmatrix}.\]
\end{itemize}
\end{enumerate}
\newpage

\subsection{Note 1: Coefficient of Repeatability}
The coefficient of repeatability is a measure of how well a
measurement method agrees with itself over replicate measurements
\citep{BA99}. Once the within-item variability is known, the
computation of the coefficients of repeatability for both methods
is straightforward.

\subsection{Note 2: Carstensen model in the single measurement case}
\citet{BXC2004} presents a model to describe the relationship between a value of measurement and its real value.
The non-replicate case is considered first, as it is the context of the Bland-Altman plots.
This model assumes that inter-method bias is the only difference between the two methods.


\begin{equation}
y_{mi}  = \alpha_{m} + \mu_{i} + e_{mi} \qquad  e_{mi} \sim \mathcal{N}(0,\sigma^{2}_{m})
\end{equation}

The differences are expressed as $d_{i} = y_{1i} - y_{2i}$.

For the replicate case, an interaction term $c$ is added to the model, with an associated variance component.




\subsection{Note 3: Model terms}
It is important to note the following characteristics of this model.
\begin{itemize}
\item Let the number of replicate measurements on each item $i$ for both methods be $n_i$, hence $2 \times n_i$ responses. However, it is assumed that there may be a different number of replicates made for different items. Let the maximum number of replicates be $p$. An item will have up to $2p$ measurements, i.e. $\max(n_{i}) = 2p$.

% \item $\boldsymbol{y}_i$ is the $2n_i \times 1$ response vector for measurements on the $i-$th item.
% \item $\boldsymbol{X}_i$ is the $2n_i \times  3$ model matrix for the fixed effects for observations on item $i$.
% \item $\boldsymbol{\beta}$ is the $3 \times  1$ vector of fixed-effect coefficients, one for the true value for item $i$, and one effect each for both methods.

\item Later on $\boldsymbol{X}_i$ will be reduced to a $2 \times 1$ matrix, to allow estimation of terms. This is due to a shortage of rank. The fixed effects vector can be modified accordingly.
\item $\boldsymbol{Z}_i$ is the $2n_i \times  2$ model matrix for the random effects for measurement methods on item $i$.
\item $\boldsymbol{b}_i$ is the $2 \times  1$ vector of random-effect coefficients on item $i$, one for each method.
\item $\boldsymbol{\epsilon}$  is the $2n_i \times  1$ vector of residuals for measurements on item $i$.
\item $\boldsymbol{G}$ is the $2 \times  2$ covariance matrix for the random effects.
\item $\boldsymbol{R}_i$ is the $2n_i \times  2n_i$ covariance matrix for the residuals on item $i$.
\item The expected value is given as $\mbox{E}(\boldsymbol{y}_i) = \boldsymbol{X}_i\boldsymbol{\beta}.$ \citep{hamlett}
\item The variance of the response vector is given by $\mbox{Var}(\boldsymbol{y}_i)  = \boldsymbol{Z}_i \boldsymbol{G} \boldsymbol{Z}_i^{\prime} + \boldsymbol{R}_i$ \citep{hamlett}.
\end{itemize}
\newpage

%\chapter{Limits of Agreement}

%\section{Modelling Agreement with LME Models}

% Carstensen pages 22-23


Roys uses and LME model approach to provide a set of formal tests for method comparison studies.\\

Four candidates models are fitted to the data.\\

These models are similar to one another, but for the imposition of equality constraints.\\

These tests are the pairwise comparison of candidate models, one formulated without constraints, the other with a constraint.\\


Roy's model uses fixed effects $\beta_0 + \beta_1$ and $\beta_0 + \beta_1$ to specify the mean of all observationsby \\ methods 1 and 2 respectuively.





Roy adheres to Random Effect ideas in ANOVA

Roy treats items as a sample from a population.\\

Allocation of fixed effects and random effects are very different in each model\\

Carstensen's interest lies in the difference between the population from which they were drawn.\\

Carstensen's model is a mixed effects ANOVA.\\

\[
Y_{mir}  =  \alpha_m + \mu_i + c_{mi} + e_{mir}, \qquad c_{mi} \sim \mathcal{\tau^2_m}, \qquad e_{mir} \sim \mathcal{\sigma^2_m},
\]

This model includes a method by item iteration term.\\

Carstensen presents two models. One for the case where the replicates, and a second for when they are linked.\\

Carstensen's model does not take into account either between-item or within-item covariance between methods.\\


In the presented example, it is shown that Roy's LoAs are lower than those of Carstensen.
Carstensen makes some interesting remarks in this regard.

\begin{quote}
The only slightly non-standard (meaning "not often used") feature is the differing residual variances between methods.
\end{quote}



\makeindex
\begin{document}
\author{Kevin O'Brien}
\title{November 2011 Version A}

\addcontentsline{toc}{section}{Bibliography}


%---------------------------------------------------------------------------%
% - 1. Model Diagnostics
% - 2. Zewotir's paper (including Haslett)
% - 3. Augmented GLMS
% - 4. Applying Diagnostics to MCS
% - 5. Extra Material
%---------------------------------------------------------------------------%
