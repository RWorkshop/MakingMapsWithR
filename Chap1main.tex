\documentclass{report}

\usepackage{standalone}
\usepackage{amssymb}
\usepackage{amsbsy}
\usepackage{amsthm}
%\usepackage[dvips]{graphicx}
\usepackage{framed}
\usepackage{amsmath}

\usepackage{graphicx}

\usepackage{natbib}
\bibliographystyle{chicago}

\usepackage{subfiles}

\usepackage[final]{pdfpages}

\begin{document}
\tableofcontents
\newpage
%-------------------------------------------------------------------------------------------%
% Chapter 1
\chapter{Introduction to Method Comparison Studies}
\subfile{introMCS-intro}
\subfile{introMCS-grubbsdata}

\newpage
\section{Agreement}
\begin{itemize}
\item The FDA define precision as the closeness of agreement (degree of
 scatter) between a series of measurements obtained from multiple
 sampling of the same homogeneous sample under prescribed
 conditions. 
\item \textbf{Barnhart} describes precision as being further
 subdivided as either within-run, intra-batch precision or
 repeatability (which assesses precision during a single analytical
 run), or between-run, inter-batch precision or repeatability
(which measures precision over time).
\end{itemize}

\section{Method Comparison Studies}

Agreement between two methods of clinical measurement can be quantified using the differences between observations made using the two methods on the same subjects. The 95% limits of agreement, estimated by mean difference +/- 1.96 standard deviation of the differences, provide an interval within which 95% of differences between measurements by the two methods are expected to lie.


\end{document}
