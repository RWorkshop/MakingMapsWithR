\documentclass{report}

\usepackage{standalone}
\usepackage{amssymb}
\usepackage{amsbsy}
\usepackage{amsthm}
%\usepackage[dvips]{graphicx}
\usepackage{framed}
\usepackage{amsmath}

\usepackage{graphicx}

\usepackage{natbib}
\bibliographystyle{chicago}

\usepackage{subfiles}

\usepackage[final]{pdfpages}

\begin{document}
\tableofcontents
\newpage
\begin{itemize}		
	%-------------------------------------------------%
	% Chapter 1
	
	\item[1]	Introduction to Method Comparison Studies	
	\begin{itemize}	
		\item	Accuracy and Precision
		\item	Repeatability (Bland Altman 1999)
		\item	Remarks in Barnhart's Paper
		\item	Regression Techniques (i.e. Orthonormal Regression and Deming Regression)
		\item   Bradley and Blackwood's, and Bartko's Techniques
	    \item   Coefficient of Repeatability
	\end{itemize}	
	
	%-------------------------------------------------%
	% Chapter 2
	
	\item[2]	Bland and Altman Plot	
	\begin{itemize}	
		\item	Bland and Altman 1983 and 1986
		\item	Limits of Agreement
		\item	Variants of the Bland-Altman Technique
		\item   Prevalence and Usage of BA's approach
		\item   Discussion of ShinyMCS appendix
	\end{itemize}
	%-------------------------------------------------%		
	% Chapter 3
	
	\item[3]	Introduction to LME Models	
	\begin{itemize}	
		\item	Model Specification of LME Models
		\item	Carstensen et al's Techniques
		\item	Using the nlme R package
		\item   Using the lme4 R package
	\end{itemize}
	
	%-------------------------------------------------%				
	% Chapter 4		
	
	\item[4]	Roy's Hypothesis Tests	
	\begin{itemize}	
		\item	Roy's Hypothesis Tests
	    \item   Likelihood Ratio Tests
		\item	Differences with Bendix Carstensens's Approach
		\item	Other Research Questions prompted by Roy's Methods
	\end{itemize}		
	
	%-------------------------------------------------%				
	% Chapter 5		
	
	\item[5]	Model Diagnostics for LMEs 
	\begin{itemize}	
		\item	Review of Model Diagnostics for Linear Models
		\item	Model Diagnostics for LME Models
		\item	Applications to MCS problems
		\item   The influence.MCS R package
	\end{itemize}		
	
	
	%-------------------------------------------------%				
	% Chapter 6		
	
	\item[6]	Other Matters: Profile Likelihood, Augmented GLMs	
	\begin{itemize}	
		\item   \textit{Spare Chapter for some as-yet unspecified matters}	
		\item	Douglas Bates Comments on Interval Estimation
		\item	Augmented GLMS
			
	\end{itemize}			
	
	\item[A]   ShinyMCS web application
	\begin{itemize}
		\item What is Shiny
		\item Why use Shiny for MCS?
		\item Technology Acceptance Model
		\item Design Considerations and Deployment
		\item Citation of a Shiny Web Application
	\end{itemize}
\end{itemize}		
\newpage
%-------------------------------------------------------------------------------------------%
% Chapter 1
\chapter{Introduction to Method Comparison Studies}
\begin{abstract}
The first chapter will consider the topic of Method Comparison Studies, and discuss the impact of the Bland-Altman Methodology. A detailed discussion of the Bland-Altman Methodology will be covered in chapter two.
\end{abstract}


\section{Introductory Definitions}



\section{Introduction}
The problem of assessing the agreement between two or more methods of measurement is ubiquitous in scientific research, and is commonly referred to as a `method comparison study'. Published examples of method comparison studies can be found in disciplines
as diverse as Pharmacology \citep{ludbrook97}, Anaesthesia \citep{Myles}, and cardiac imaging methods \citep{Krumm}.
\smallskip
Method Comparison Studies is a branch of statistics used to compare the results of two different method of measurement, measuring the same subject samples. Consider a set of n samples. Measurements are taken on each of the n samples using both methods. This will enable comparison of the method used.
\smallskip
In many cases the purpose of the study is to calibrate a new method of measurement against a ‘Gold Standard’ method. A ‘Gold Standard’ method is the known method that is considered most precise in its measurement. It should not be assumed that there is no error present in its measurements.
\smallskip
The Gold Standard may not be financially feasible for general use, and therefore more economical methods, of suitable levels of precisions, must be devised. Method Comparison studies is used to ascertain the levels of precision of such methods.
\smallskip

To illustrate the characteristics of a typical method comparison study consider the data in Table I, taken from \citet{Grubbs73}.
\smallskip
In each of twelve experimental trials a single round of ammunition was fired from a 155mm gun, and its velocity was measured
simultaneously (and independently) by three chronographs devices, referred to here as `Fotobalk', `Counter' and `Terma'.
\smallskip


\newpage

\begin{table}[ht]
	\begin{center}
		\begin{tabular}{rrrr}
			\hline
			Round& Fotobalk [F] & Counter [C]& Terma [T]\\
			\hline
			1 & 793.8 & 794.6 & 793.2 \\
			2 & 793.1 & 793.9 & 793.3 \\
			3 & 792.4 & 793.2 & 792.6 \\
			4 & 794.0 & 794.0 & 793.8 \\
			5 & 791.4 & 792.2 & 791.6 \\
			6 & 792.4 & 793.1 & 791.6 \\
			7 & 791.7 & 792.4 & 791.6 \\
			8 & 792.3 & 792.8 & 792.4 \\
			9 & 789.6 & 790.2 & 788.5 \\
			10 & 794.4 & 795.0 & 794.7 \\
			11 & 790.9 & 791.6 & 791.3 \\
			12 & 793.5 & 793.8 & 793.5 \\
			\hline
		\end{tabular}
		\caption{Measurement of the three chronographs (Grubbs 1973)}
	\end{center}
\end{table}

An important aspect of the these data is that all three methods of
measurement are assumed to have an attended measurement error, and
the velocities reported in Table I can not be assumed to be `true
values' in any absolute sense. For expository purposes only the
first two methods `Fotobalk' and `Counter' will enter in the
immediate discussion.

%While lack of
%agreement between two methods is inevitable, the question , as
%posed by \citet{BA83}, is 'do the two methods of measurement agree
%sufficiently closely?'

A method of measurement should ideally be both accurate and
precise.An accurate measurement methods shall give a result close
to the `true value'. Precision of a method is indicated by how
tightly clustered its measurements are around their mean
measurement value.

\newpage

A precise and accurate method should yield results consistently
close to the true value. However a method may be accurate, but not
precise. The average of its measurements is close to the true
value, but those measurements are highly dispersed. Conversely an
inaccurate method may be quite precise , as it consistently
indicates the same level of inaccuracy.

The tendency of a method of measurement to consistently give
results above or below the true value is a source of systematic
bias. The lesser the systematic bias, the greater the accuracy of
the method.

In the context of the agreement of two methods, there is also a
tendency of one measurement method to consistently give results
above or below the other method. Lack of agreement is a
consequence of the existence of `inter-method bias'. For two
methods to be considered in good agreement, the inter-method bias
should be in the region of zero.

A simple estimation of the inter-method bias can be calculated
using the differences of the paired measurements. The data in
Table 1.2 are a good example of possible inter-method bias; the
`Fotobalk' consistently recording smaller velocities than the
`Counter' method. Consequently there is lack of agreement between
the two methods.
\newpage
% latex table generated in R 2.6.0 by xtable 1.5-5 package
% Wed Aug 26 15:22:41 2009
\begin{table}[h!]
	\begin{center}
		
		\begin{tabular}{rrrr}
			\hline
			Round& Fotobalk (F) & Counter (C) & F-C \\
			\hline
			1 & 793.80 & 794.60 & -0.80 \\
			2 & 793.10 & 793.90 & -0.80 \\
			3 & 792.40 & 793.20 & -0.80 \\
			4 & 794.00 & 794.00 & 0.00 \\
			5 & 791.40 & 792.20 & -0.80 \\
			6 & 792.40 & 793.10 & -0.70 \\
			7 & 791.70 & 792.40 & -0.70 \\
			8 & 792.30 & 792.80 & -0.50 \\
			9 & 789.60 & 790.20 & -0.60 \\
			10 & 794.40 & 795.00 & -0.60 \\
			11 & 790.90 & 791.60 & -0.70 \\
			12 & 793.50 & 793.80 & -0.30 \\
			\hline
		\end{tabular}
		\caption{Difference between Fotobalk and Counter measurements}
	\end{center}
\end{table}

\bigskip

\noindent The absence of inter-method bias by itself is not
sufficient to establish whether two measurement methods agree or
not. These methods must also have equivalent levels of precision.
Should one method yield results considerably more variable than
that of the other, they can not be considered to be in agreement.

Therefore a methodology must be introduced that allows an analyst
to estimate the inter-method bias, and to compare the precision of
both methods of measurement.
%%%%%%%%%%%%%%%%%%%%%%%%%%%%%%%%%%%%%%%%%%%%%%%%%%%%%%%%%%%%%%%%%%%%%%%%%%%%%%%%%%%%%%
\newpage
\section{Bland Altman Plots}
The issue of whether two measurement methods comparable to the
extent that they can be used interchangeably with sufficient
accuracy is encountered frequently in scientific research.
Historically comparison of two methods of measurement was carried
out by use of correlation coefficients or simple linear
regression. Bland and Altman recognized the inadequacies of these
analyses and articulated quite thoroughly the basis on which of
which they are unsuitable for comparing two methods of measurement
\citep*{BA83}.


Furthermore they proposed their simple methodology specifically
constructed for method comparison studies. They acknowledge that
there are other valid, but complex, methodologies, and argue that
a simple approach is preferable to this complex approaches,
\emph{especially when the results must be explained to
	non-statisticians} \citep*{BA83}.

\smallskip

Notwithstanding previous remarks about regression, the first step
recommended ,which the authors argue should be mandatory,is
construction of a simple scatter plot of the data. The line of
equality ($X=Y$) should also be shown, as it is necessary to give
the correct interpretation of how both methods compare. A scatter
plot of the Grubbs data is shown in figure 2.1. A visual
inspection thereof confirms the previous conclusion that there is
an inter method bias present, i.e. Fotobalk device has a tendency
to record a lower velocity.

%\begin{figure}[h!]
%	\begin{center}
%		\includegraphics[width=130mm]{GrubbsScatter.jpeg}
%		\caption{Scatter plot For Fotobalk and Counter Methods}\label{GrubbsScatter}
%	\end{center}
%\end{figure}

In light of shortcomings associated with scatterplots,
\citet*{BA83} recommend a further analysis of the data. Firstly
differences of measurements of two methods on the same subject
should  be calculated, and then the average of those measurements
(Table 2.1). These differences and averages are then plotted
(Figure 2.2).




The dashed line in Figure 2.2 alludes to the inter method bias
between the two methods, as mentioned previously. Bland and Altman
recommend the estimation of inter method bias by calculating the
average of the differences. In the case of Grubbs data the inter
method bias is $-0.6083$ metres per second.
\newpage

% latex table generated in R 2.6.0 by xtable 1.5-5 package
% Thu Aug 27 16:31:52 2009
\begin{table}[tbh]
	\begin{center}
		
		\begin{tabular}{rrrrr}
			\hline
			Round & Fotobalk [F] & Counter [C] & Differences [F-C] & Averages [(F+C)/2] \\
			\hline
			1 & 793.80 & 794.60 & -0.80 & 794.20 \\
			2 & 793.10 & 793.90 & -0.80 & 793.50 \\
			3 & 792.40 & 793.20 & -0.80 & 792.80 \\
			4 & 794.00 & 794.00 & 0.00 & 794.00 \\
			5 & 791.40 & 792.20 & -0.80 & 791.80 \\
			6 & 792.40 & 793.10 & -0.70 & 792.80 \\
			7 & 791.70 & 792.40 & -0.70 & 792.00 \\
			8 & 792.30 & 792.80 & -0.50 & 792.50 \\
			9 & 789.60 & 790.20 & -0.60 & 789.90 \\
			10 & 794.40 & 795.00 & -0.60 & 794.70 \\
			11 & 790.90 & 791.60 & -0.70 & 791.20 \\
			12 & 793.50 & 793.80 & -0.30 & 793.60 \\
			\hline
		\end{tabular}
		\caption{Fotobalk and Counter Methods: Differences and Averages}
	\end{center}
\end{table}


%\begin{figure}[h!]
%	\begin{center}
%		\includegraphics[width=120mm]{GrubbsBAplot.jpeg}
%		\caption{Bland Altman Plot For Fotobalk and Counter Methods}\label{GrubbsBA}
%	\end{center}
%\end{figure}

\newpage
By inspection of the plot, it is also possible to compare the
precision of each method. Noticeably the differences tend to
increase as the averages increase.

\subsection{Inspecting the Data}
Bland-Altman plots are a powerful graphical methodology for making
a visual assessment of the data. \citet*{BA83} express the
motivation for this plot thusly:
\begin{quote}
	"From this type of plot it is much easier to assess the magnitude
	of disagreement (both error and bias), spot outliers, and see
	whether there is any trend, for example an increase in
	(difference) for high values. This way of plotting the data is a
	very powerful way of displaying the results of a method comparison
	study."
\end{quote}


Figures 1.3 1.4 and 1.5 are three Bland-Altman plots derived from
simulated data, each for the purpose of demonstrating how the plot
would inform an analyst of trends that would adversely affect use
of the recommended methodology. Figure 1.3 demonstrates how the
Bland Altman plot would indicate increasing variance of
differences over the measurement range. Figure 1.4 is an example
of cases where the inter-method bias changes over the measurement
range. This is known as proportional bias \citep{ludbrook97}.


%\begin{figure}[h!]
%	\begin{center}
%		\includegraphics[width=125mm]{BAFanEffect.jpeg}
%		\caption{Bland-Altman Plot demonstrating the increase of variance over the range}\label{BAFanEffect}
%	\end{center}
%\end{figure}
%
%\begin{figure}[h!]
%	\begin{center}
%		\includegraphics[width=125mm]{PropBias.jpeg}
%		\caption{Bland-Altman Plot indicating the presence of proportional bias}\label{PropBias}
%	\end{center}
%\end{figure}

\newpage
Figure 1.4 is an example of cases where the inter-method bias
changes over the measurement range. This is known as\textit{ proportional
bias} (Ludbrook, 1997). Both of these cases violate the assumptions
necessary for further analysis using limits of agreement ,which
shall be discussed later. The plot also can be used to identify
outliers. An outlier is an observation that is numerically distant
from the rest of the data. Classification thereof is a subjective
decision in any analysis, but must be informed by the logic of the
formulation. Figure 1.5 is a Bland Altman plot with two
conspicuous observations, at the extreme left and right of the
plot respectively.


%\begin{figure}[h!]
%	\begin{center}
%		\includegraphics[width=125mm]{BAOutliers.jpeg}
%		\caption{Bland-Altman Plot indicating the presence of Outliers}\label{PropBias}
%	\end{center}
%\end{figure}

In the Bland-Altman plot, the horizontal displacement of any
observation is supported by two independent measurements. Hence
any observation , such as the one on the extreme right of figure
1.5, should not be considered an outlier on the basis of a
noticeable horizontal displacement from the main cluster. The one
on the extreme left should be considered an outlier, as it has a
noticeable vertical displacement from the rest of the
observations.

\citet*{BA99} do not recommend excluding outliers from analyses.
However recalculation of the inter-method bias estimate , and
further calculations based upon that estimate, are useful for
assessing the influence of outliers.\citep{BA99} states that
\emph{"We usually find that this method of analysis is not too
	sensitive to one or two large outlying differences."}

\subsection{Limits of Agreement}
\citet{BA86} introduces an elaboration of the plot, adding to the
plot `limits of agreement' to the plot. These limits are based
upon the standard deviation of the differences. The discussion
shall be reverted to these limits of agreement in due course.

\subsection{Variations of the Bland Altman Plot}
\citet{BA99} remarks that it is possible to ignore the issue
altogether, but the limits of agreement would wider apart than
necessary when just lower magnitude measurements are considered.
Conversely the limits would be too narrow should only higher
magnitude measurements be used. To address the issue, they propose
the logarithmic transformation of the data. The plot is then
formulated as the difference of paired log values against their
mean. \citet{BA99} acknowledge that this is not easy to interpret,
and that it is not suitable in all cases.

\citet{BA99} offers two variations of the Bland -Altman plot that
are intended to overcome potential problems that the conventional
plot would inappropriate for.

The first variation is a plot of casewise differences as
percentage of averages, and is appropriate when there is an
increase in variability of the differences as the magnitude
increases. The second variation is a plot of casewise ratios as
percentage of averages.


% When selecting this option the differences will be expressed as
% percentage of the averages. This option is useful when there is an
% increase in variability of the differences as the magnitude of the
% measurement increases.




% Plot ratios When this option is selected then the ratios of the
% measurements will be plotted instead of the differences (avoiding
% the need for log transformation). This option as well is useful
% when there is an increase in variability of the differences as the
% magnitude of the measurement increases.

%----------------------------------------------------------------------------%
\subsection{Agreement} Bland and Altman (1986) defined perfect
agreement as the case where all of the pairs of rater data lie
along the line of equality, where the line of equality is defined
as the $45$ degree line passing through the origin(i.e. the $X=Y$
line).

Bland and Altman (1986)expressed this in the terms \emph{we want
	to know by how much the new method is likely to differ from the
	old; if this is not enough to cause problems in clinical
	interpretation we can replace the old method by the new or use the
	two interchangeably. How far apart measurements can be without
	causing difficulties will be a question of judgment. Ideally, it
	should be defined in advance to help in the interpretation of the
	method comparisonand to choose the sample size” .}
%----------------------------------------------------------------------------%
\subsection{Bias}
Bland and Altman define bias a \emph{a consistent tendency for one
	method to exceed the other} and propose estimating its value
by determining the mean of the differences. The variation about
this mean shall be estimated by the  standard deviation of the
differences. Bland and Altman remark that these estimates are based on the
assumption that bias and variability are constant throughout the
range of measures.
%----------------------------------------------------------------------------%
\subsection{Inappropriate assessment of Agreement}
\subsubsection{Paired T tests} This method can be applied to test for
statistically significant deviations in bias. This method can be
potentially misused for method comparison studies.
\\It is a poor measure of agreement when the rater's measurements
are perpendicular to the line of equality[Hutson et al]. In this
context, an average difference of zero between the two raters, yet
the scatter plot displays strong negative correlation.
\subsubsection{Inappropriate Methodologies} Use of the Pearson
Correlation Coefficient , although seemingly intuitive, is not
appropriate approach to assessing agreement of two methods.
Arguments against its usage have been made repeatedly in the
relevant literature. It is possible for two analytical methods to
be highly correlated, yet have a poor level of agreement.
\subsubsection{Pearson's Correlation Coefficient} It is well known that
Pearson's correlation coefficient is a measure of the linear
association between two variables, not the agreement between two
variables (e.g., see Bland and Altman 1986)..This is a well known
as a measure of linear association between two
variables.Nonetheless this is not necessarily the same as
Agreement. This method is considered wholly inadequate to assess
agreement because it only evaluates only the association of two
sets of observations.

%----------------------------------------------------------------------------%
\subsection{Inappropriate use of the Correlation Coefficient}
It is intuitive when dealing with two sets of related data, i.e
the results of the two raters,  to calculate the correlation
coefficient (r). Bland and Altman attend to this in their $1999$
paper.

They present a data set from two sets of meters, and an
accompanying scatterplot. An hypothesis test on the data set leads
us to conclude that there is a relationship between both sets of
meter measurements. The correlation coeffiecient is determined to
be r =0.94.However, this high correlation does not mean that the
two methods agree. It is possible to determine from the
scatterplot that the intercept is not zero, a requirement for
stating both methods have high agreement. Essentially, should two
methods have highly correlated results, it does not follow that
they have high agreement.

%----------------------------------------------------------------------------%
\subsection{Bland Altman Plot}
Bland Altman have recommended the use of graphical techniques to
assess agreement. Principally their method is calculating , for
each pair of corresponding two methods of measurement of some
underlying quantity, with no replicate measurements, the
difference and mean. Differences are then plotted against the
mean.

\textbf{\textit{Hopkins}} argued that the bias in a subsequent Bland-Altman plot was
due, in part, to using least-squares regression at the calibration
phase.


%This page also shows the standard deviation (SD) of the
%differences between the two assay methods. The SD value is used to
%calculate the limits of agreement, computed as the mean bias plus
%or minus 1.96 times its SD.
%----------------------------------------------------------------------------%
\subsection{The Bland Altman Plot}
In 1986 Bland and Altman published a paper in the Lancet proposing
the difference plot for use for method comparison purposes. It has
proved highly popular ever since. This is a simple, and widely
used , plot of the differences of each data pair, and the
corresponding average value. An important requirement is that the
two measurement methods use the same scale of measurement.

\subsubsection{scatter plots} The authors advise the
use of scatter plots to identify outliers, and to determine if
there is curvilinearity present. In the region of linearity
,simple linear regression may yield results of interest.

\subsection{Effect of Outliers} Another argument against
the use of model I regression is based on outliers. Outliers can
adversely influence the fitting of a regression model. Cornbleet
and Cochrane compare a regression model influenced by an outlier
with a model for the same data set, with the outlier excluded from
the data set. A demonstration of the effect of outliers was made
in Bland Altman's 1986 paper. However they discourage the
exclusion of outliers.

%----------------------------------------------------------------------------%
\subsection{Limits Of Agreement}
Bland and Altman proposed a pair of Limits of agreement. These
limits are intended to demonstrate the range in which 95\% of the
sample data should lie. The Limits of agreement centre on the
average difference line and are 1.96 times the standard deviation
above and below the average difference line.

How this relates the overall population is unclear. It seems that
it depends on an expert to decide whether or not the range of
differences is acceptable. In a study A Bland-Altman plots compare
two assay methods. It plots the difference between the two
measurements on the Y axis, and the average of the two
measurements on the X axis.

The bias is computed as the average of the difference of paired
assays.

If one method is sometimes higher, and sometimes the other method
is higher, the average of the differences will be close to zero.
If it is not close to zero, this indicates that the two assay
methods are producing different results systematically.

\subsubsection{Precision of Limits of Agreement}
The limits of agreement are estimates derived from the sample
studied, and will differ from values relevant to the whole
population. A different sample would give different limits of
agreement. \citet*{BA86} advance a formulation for confidence
intervals of the inter-method bias and the limits of agreement.
These calculations employ quantiles of the `t' distribution with
$n -1$ degrees of freedom.

%This page also shows the standard deviation (SD) of the
%differences between the two assay methods. The SD value is used to
%calculate the limits of agreement, computed as the mean bias plus
%or minus 1.96 times its SD.
%----------------------------------------------------------------------------%
\subsection{Appropriate Use of Limits of Agreement}
Importantly \citet{BA99} makes the following point:
\begin{quote}These estimates are meaningful only if we can assume
	bias and variability are uniform throughout the range of
	measurement, assumptions which can be checked graphically.
\end{quote}

The import of this statement is that , should the Bland Altman
plot indicate that these assumptions are not met, then their
entire methodology, as posited thus far, is inappropriate for use
in a method comparison study. Again, in the context of potential
outlier in the Grubbs data (figure 1.2), this raises the question
on how to correctly continue.

Carstensen attends to the issue of repeated data, using the
expression replicate to express a repeated measurement on a
subject by the same methods. Carstensen formulates the data as
follows Repeated measurement - Arrangement of data into groups,
based on the series of results of each subject.

%----------------------------------------------------------------------------%
\subsection{The Bland Altman Plot - Variations}
Variations of the Bland Altman plot is the use of ratios, in the
place of differences.
\begin{equation}
D_{i} = X_{i} - Y_{i}   \label{BA01}
\end{equation}
Altman and Bland suggest plotting the within subject differences $
D = X_{1} - X_{2} $ on the ordinate versus the average of $x_{1}$
and  $x_{2}$ on the abscissa.
%----------------------------------------------------------------------------%
\subsection{Pitman \& Morgan Test} This test assess tthe equaltiy
of population vairances. Pitman's test tests for zero corrleation
between the sums and products.

Correlation between differences and means is a test statistics for
the null hypothesis of equal variances given bivariate normality.
%----------------------------------------------------------------------------%
\subsection{Lin's Reproducibility Index} Lin proposes the use of a
reproducibility index, called the Concordance Correlation
Coefficent (CCC).While it is not strictly a measure of agreement
as such, it can form part of an overall method comparision
methodology.

\subfile{introMCS-intro}
\subfile{introMCS-grubbsdata}

\newpage
\section{Agreement}
\begin{itemize}
\item The FDA define precision as the \textit{closeness of agreement} (degree of
 scatter) between a series of measurements obtained from multiple
 sampling of the same homogeneous sample under prescribed
 conditions. 
\item \textbf{Barnhart} describes precision as being further
 subdivided as either within-run, intra-batch precision or
 repeatability (which assesses precision during a single analytical
 run), or between-run, inter-batch precision or repeatability
(which measures precision over time).
\end{itemize}

\section{Method Comparison Studies}

Agreement between two methods of clinical measurement can be quantified using the differences between observations made using the two methods on the same subjects. The 95\% limits of agreement, estimated by mean difference +/- 1.96 standard deviation of the differences, provide an interval within which 95\% of differences between measurements by the two methods are expected to lie.

%----------------------------------------------------------------------------%
\addcontentsline{toc}{section}{Bibliography}

\bibliography{transferbib}
\end{document}

