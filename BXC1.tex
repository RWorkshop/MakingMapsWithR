\documentclass{report}
\voffset=-1.5cm
\oddsidemargin=0.0cm
\textwidth = 470pt

\usepackage{amssymb}
\usepackage{amsbsy}
\usepackage{amsthm}
%\usepackage[dvips]{graphicx}
\usepackage{framed}
\usepackage{amsmath}

\usepackage{graphicx}

\usepackage{natbib}
\bibliographystyle{chicago}

\begin{document}

\subsection{Carstensen's Model}

\citet{BXC2004} presents a model to describe the relationship between a value of measurement and its
real value. The non-replicate case is considered first, as it is the context of the Bland Altman plots. This model assumes that inter-method bias is the only difference between the two methods.

A measurement $y_{mi}$ by method $m$ on individual $i$ is formulated as follows;
\begin{equation}
y_{mi}  = \alpha_{m} + \mu_{i} + e_{mi} \qquad  e_{mi} \sim
\mathcal{N}(0,\sigma^{2}_{m})
\end{equation}
The differences are expressed as $d_{i} = y_{1i} - y_{2i}$. For the replicate case, an interaction term $c$ is added to the model, with an associated variance component. All the random effects are assumed independent, and that all replicate measurements are assumed to be exchangeable within each method.

\begin{equation}
y_{mir}  = \alpha_{m} + \mu_{i} + c_{mi} + e_{mir}, \qquad  e_{mi}
\sim \mathcal{N}(0,\sigma^{2}_{m}), \quad c_{mi} \sim \mathcal{N}(0,\tau^{2}_{m}).
\end{equation}
%----

Of particular importance is terms of the model, a true value for item $i$ ($\mu_{i}$).  The fixed effect of Roy's model comprise of an intercept term and fixed effect terms for both methods, with no reference to the true value of any individual item. A distinction can be made between the two models: Roy's model is a standard LME model, whereas Carstensen's model is a more complex additive model.


\newpage
Let $y_{mir} $ denote the $r$th replicate measurement on the $i$th item by the $m$th method, where $m=1,2$ ; $i=1,\ldots,N;$ and $r = 1,\ldots,n_i.$ When the design is balanced and there is no ambiguity we can set $n_i=n.$ The LME model underpinning Roy's approach can be written
\begin{equation}\label{Roy-model}
y_{mir} = \beta_{0} + \beta_{m} + b_{mi} + \epsilon_{mir}.
\end{equation}
Here $\beta_0$ and $\beta_m$ are fixed-effect terms representing, respectively, a model intercept and an overall effect for method $m.$ The model can be reparameterized by gathering the $\beta$ terms together into (fixed effect) intercept terms $\alpha_m=\beta_0+\beta_m.$ The $b_{1i}$ and $b_{2i}$ terms are correlated random effect parameters having $\mathrm{E}(b_{mi})=0$ with $\mathrm{Var}(b_{mi})=g^2_m$ and $\mathrm{Cov}(b_{1i}, b_{2 i})=g_{12}.$ The random error term for each response is denoted $\epsilon_{mir}$ having $\mathrm{E}(\epsilon_{mir})=0$, $\mathrm{Var}(\epsilon_{mir})=\sigma^2_m$, $\mathrm{Cov}(\epsilon_{1ir}, \epsilon_{2 ir})=\sigma_{12}$, $\mathrm{Cov}(\epsilon_{mir}, \epsilon_{mir^\prime})= 0$ and $\mathrm{Cov}(\epsilon_{1ir}, \epsilon_{2 ir^\prime})= 0.$ Additionally these parameter are assumed to have Gaussian distribution. Two methods of measurement are in complete agreement if the null hypotheses $\mathrm{H}_1\colon \alpha_1 = \alpha_2$ and $\mathrm{H}_2\colon \sigma^2_1 = \sigma^2_2 $ and $\mathrm{H}_3\colon g^2_1= g^2_2$ hold simultaneously. \citet{roy} uses a Bonferroni correction to control the familywise error rate for tests of $\{\mathrm{H}_1, \mathrm{H}_2, \mathrm{H}_3\}$ and account for difficulties arising due to multiple testing. Additionally, Roy combines $\mathrm{H}_2$ and $\mathrm{H}_3$ into a single testable hypothesis $\mathrm{H}_4\colon \omega^2_1=\omega^2_2,$ where $\omega^2_m = \sigma^2_m + g^2_m$ represent the overall variability of method $m.$
%Disagreement in overall variability may be caused by different between-item variabilities, by different within-item variabilities, or by both.

%If the exact cause of disagreement between the two methods is not of interest, then the overall variability test $H_4$ %is an alternative to testing $H_2$ and $H_3$ separately.

\bigskip

\cite{BXC2008} also use a LME model for the purpose of comparing two methods of measurement where replicate measurements are available on each item. Their interest lies in generalizing the popular limits-of-agreement (LOA) methodology advocated by \citet{BA86} to take proper cognizance of the replicate measurements. \citet{BXC2008} demonstrate statistical flaws with two approaches proposed by \citet{BA99} for the purpose of calculating the variance of the inter-method bias when replicate measurements are available. Instead, they recommend a fitted mixed effects model to obtain appropriate estimates for the variance of the inter-method bias. As their interest mainly lies in extending the Bland-Altman methodology, other formal tests are not considered.

\bigskip

\citet{BXC2008} develop their model from a standard two-way analysis of variance model, reformulated for the case of replicate measurements, with random effects terms specified as appropriate.
Their model can be written as
%describing $y_{mir} $, again the $r$th replicate measurement on the $i$th item by the $m$th method ($m=1,2,$ %$i=1,\ldots,N,$ and $r = 1,\ldots,n$),

\begin{equation}\label{BXC-model}
y_{mir}  = \alpha_{m} + \mu_{i} + a_{ir} + c_{mi} + \varepsilon_{mir}.
\end{equation}
The fixed effects $\alpha_{m}$ and $\mu_{i}$ represent the intercept for method $m$ and the `true value' for item $i$ respectively. The random-effect terms comprise an item-by-replicate interaction term $a_{ir} \sim \mathcal{N}(0,\varsigma^{2})$, a method-by-item interaction term $c_{mi} \sim \mathcal{N}(0,\tau^{2}_{m}),$ and model error terms $\varepsilon_{mir} \sim \mathcal{N}(0,\varphi^{2}_{m}).$ All random-effect terms are assumed to be independent. For the case when replicate measurements are assumed to be exchangeable for item $i$, $a_{ir}$ can be removed. The model expressed in (2) describes measurements by $m$ methods, where $m = \{1,2,3\ldots\}$. Based on the model expressed in (2), \citet{BXC2008} compute the limits of agreement as
\[
\alpha_1 - \alpha_2 \pm 2 \sqrt{ \tau^2_1 +  \tau^2_2 +  \varphi^2_1 +  \varphi^2_2 }
\]
\citet{BXC2008} notes that, for $m=2$,  separate estimates of $\tau^2_m$ can not be obtained. To overcome this, the assumption of equality, i.e. $\tau^2_1 = \tau^2_2$ is required.

%%---Comparative Complexity
There is a substantial difference in the number of fixed parameters used by the respective models; the model in (\ref{Roy-model}) requires two fixed effect parameters, i.e. the means of the two methods, for any number of items $N$, whereas the model in (\ref{BXC-model}) requires $N+2$ fixed effects.

Allocating fixed effects to each item $i$ by (\ref{BXC-model}) accords with earlier work on comparing methods of measurement, such as \citet{Grubbs48}. However allocation of fixed effects in ANOVA models suggests that the group of items is itself of particular interest, rather than as a representative sample used of the overall population. However this approach seems contrary to the purpose of LOAs as a prediction interval for a population of items. Conversely, \citet{roy}
uses a more intuitive approach, treating the observations as a random sample population, and allocating random effects accordingly.

\newpage
\section{The Research of Carstensen et al}
\subsection{Bendix Carstensen's data sets}
\citet{bxc2008}describes the sampling method when discussing of a motivating example.Diabetes patients attending an outpatient clinic in Denmark have their $HbA_{1c}$ levels routinely measured at every visit.Venous and Capillary blood samples were obtained from all patients appearing at the clinic over two days.

Samples were measured on four consecutive days on each machines, hence there are five analysis days.Carstensen notes that every machine was calibrated every day to  the manufacturers guidelines.


\subsection{Limits of agreement for Carstensen's data}


\citet{bxc2008} describes the calculation of the limits of agreement (with the inter-method bias implicit) for both data sets, based on his formulation;

\[\hat{\alpha}_1 - \hat{\alpha}_2 \pm 2\sqrt{2\hat{\tau}^2 +\hat{\sigma}_1^2 +\hat{\sigma}_2^2 }.\]

For the `Fat' data set, the inter-method bias is shown to be $0.045$. The limits of agreement are $(-0.23 , 0.32)$

Carstensen demonstrates the use of the interaction term when computing the limits of agreement for the `Oximetry' data set. When the interaction term is omitted, the limits of agreement are $(-9.97, 14.81)$. Carstensen advises the inclusion of the interaction term for linked replicates, and hence the limits of agreement are recomputed as $(-12.18,17.12)$.

\subsection{Using LME Models for Method Comparison}
\citet{bxc2008} formulates an LME model, both in the absence and the presence of an interaction term.\citet{bxc} uses both to demonstrate the importance of using an interaction term. Failure to take the replication structure into
account results in over-estimation of the limits of agreement. For the Carstensen estimates below, an interaction term was included when computed.

\subsection{Computing LoAs with LMEs}
%\subsection{Carstensen's LOAs}


Carstensen presents a model where the variation between items for
method $m$ is captured by $\sigma_m$ and the within item variation
by $\tau_m$.

Further to his model, Carstensen computes the limits of agreement
as

\[
\hat{\alpha}_1 - \hat{\alpha}_2 \pm \sqrt{2 \hat{\tau}^2 +
	\hat{\sigma}^2_1 + \hat{\sigma}^2_2}
\]

%---------------------------------------------------------------------------------%



\section{Carstensen's Model}
\citet{BXC2008} proposes an approach for comparing two or more methods of measurement based on linear mixed effects models. This approach extends the well established Bland-Altman methodology for the case of replicate measurements on each item. Carstensen considers the matter of computing an appropriate estimate for the standard deviation of case-wise differences, so as to determine the limits of agreement. As the interest lies in extending the Bland-Altman methodology, other formal tests are not described.

Using Carstensen's notation, a measurement $y_{mi}$ by method $m$ on individual $i$ the measurement $y_{mir} $ is the $r$th replicate measurement on the $i$th item by the $m$th method, where $m=1,2,\ldots,M$ $i=1,\ldots,N,$ and $r = 1,\ldots,n_i$ is formulated as follows;
\begin{equation}
	y_{mir}  = \alpha_{m} + \mu_{i} + c_{mi} + a_{ir} + \epsilon_{mir}, \qquad \quad c_{mi} \sim \mathcal{N}(0,\tau^{2}_{m}) , a_{ir} \sim \mathcal{N}(0,\varsigma^{2}),  \varepsilon_{mi} \sim \mathcal{N}(0,\varphi^{2}_{m}) .
\end{equation}

Here the terms $\alpha_{m}$ and $\mu_{i}$ represent the fixed effect for method $m$ and a true value for item $i$ respectively. The random effect terms comprise an interaction term $c_{mi}$ and the residuals $\varepsilon_{mir}$.
The $c_{mi}$ term represent random effect parameters corresponding to the two methods, having $\mathrm{E}(c_{mi})= 0$ with $\mathrm{Var}(c_{mi})=\tau^2_m$.  

%%%%Stuff about extra interaction term

The random error term for each response is denoted $\varepsilon_{mir}$ having $\mathrm{E}(\varepsilon_{mir})=0$, $\mathrm{Var}(\varepsilon_{mir})=\varphi^2_m$. All the random effects are assumed independent, and that all replicate measurements are assumed to be exchangeable within each method.

%Carstensen specifies the variance of the interaction terms as being univariate normally distributed. As such, $\mathrm{Cov}(c_{mi}, c_{m^\prime i})= 0.$

When only two methods are to be compared, separate estimates of $\tau^2_m$ can not be obtained. Instead the average value $\tau^2$ is obtained and used.


Carstensen's approach is that of a standard two-way mixed effects ANOVA with replicate measurements. With regards to the specification of the variance terms, Carstensen remarks that using his approach is common, remarking that \emph{
	The only slightly non-standard (meaning "not often used") feature is the differing residual variances between methods }\citep{bxc2010}.

In contrast to Roy's model, Carstensen's model requires that commonly used assumptions be applied, specifically that the off-diagonal elements of the between-item and within-item variability matrices are zero. By
extension the overall variability off-diagonal elements are also zero. Also, implementation requires that the between-item variances are estimated as the same value: $\tau^2_1 = \tau^2_2 = \tau^2$.


\[\left(\begin{array}{cc}
\omega^1_2  & 0 \\
0 & \omega^2_2 \\
\end{array}  \right)
=  \left(
\begin{array}{cc}
\tau^2  & 0 \\
0 & \tau^2 \\
\end{array} \right)+
\left(
\begin{array}{cc}
\sigma^2_1  & 0 \\
0 & \sigma^2_2 \\
\end{array}\right)
\]


%---Key difference 1---The True Value
%---Colollary -- Difference in model types
The presence of the true value term $\mu_i$ gives rise to an important difference between Carstensen's and Roy's models. The fixed effect of Roy's model comprise of an intercept term and fixed effect terms for both methods, with no reference to the true value of any individual item. In other words, Roy considers the group of items being measured as a sample taken from a population. Therefore a distinction can be made between the two models: Roy's model is a standard LME model, whereas Carstensen's model is a more complex additive model.

%---Carstensen's limits of agreement
%---The between item variances are not individually computed. An estimate for their sum is used.
%---The within item variances are indivdually specified.
%---Carstensen remarks upon this in his book (page 61), saying that it is "not often used".
%---The Carstensen model does not include covariance terms for either VC matrices.
%---Some of Carstensens estimates are presented, but not extractable, from R code, so calculations have to be done by %---hand.
%---All of Roys stimates are  extractable from R code, so automatic compuation can be implemented
%---When there is negligible covariance between the two methods, Roys LoA and Carstensen's LoA are roughly the same.
%---When there is covariance between the two methods, Roy's LoA and Carstensen's LoA differ, Roys usually narrower.


\section{Carstensen's Mixed Models}

\citet{BXC2004} proposes linear mixed effects models for deriving
conversion calculations similar to Deming's regression, and for
estimating variance components for measurements by different
methods. The following model ( in the authors own notation) is
formulated as follows, where $y_{mir}$ is the $r$th replicate
measurement on subject $i$ with method $m$.

\begin{equation}
	y_{mir}  = \alpha_{m} + \beta_{m}\mu_{i} + c_{mi} + e_{mir} \qquad
	( e_{mi} \sim N(0,\sigma^{2}_{m}), c_{mi} \sim N(0,\tau^{2}_{m}))
\end{equation}
The intercept term $\alpha$ and the $\beta_{m}\mu_{i}$ term follow
from \citet{DunnSEME}, expressing constant and proportional bias
respectively , in the presence of a real value $\mu_{i}.$
$c_{mi}$ is a interaction term to account for replicate, and
$e_{mir}$ is the residual associated with each observation.
Since variances are specific to each method, this model can be
fitted separately for each method.

The above formulation doesn't require the data set to be balanced.
However, it does require a sufficient large number of replicates
and measurements to overcome the problem of identifiability. The
import of which is that more than two methods of measurement may
be required to carry out the analysis. There is also the
assumptions that observations of measurements by particular
methods are exchangeable within subjects. (Exchangeability means
that future samples from a population behaves like earlier
samples).

%\citet{BXC2004} describes the above model as a `functional model',
%similar to models described by \citet{Kimura}, but without any
%assumptions on variance ratios. A functional model is . An
%alternative to functional models is structural modelling

\citet{BXC2004} uses the above formula to predict observations for
a specific individual $i$ by method $m$;

\begin{equation}BLUP_{mir} = \hat{\alpha_{m}} + \hat{\beta_{m}}\mu_{i} +
	c_{mi} \end{equation}. Under the assumption that the $\mu$s are
the true item values, this would be sufficient to estimate
parameters. When that assumption doesn't hold, regression
techniques (known as updating techniques) can be used additionally
to determine the estimates. The assumption of exchangeability can
be unrealistic in certain situations. \citet{BXC2004} provides an
amended formulation which includes an extra interaction term ($
d_{mr} \sim N(0,\omega^{2}_{m}$)to account for this.


\newpage
\citet{BXC2008} sets out a methodology of computing the limits of
agreement based upon variance component estimates derived using
linear mixed effects models. Measures of repeatability, a
characteristic of individual methods of measurements, are also
derived using this method.




\citet{BXC2004} uses the above formula to predict observations for
a specific individual $i$ by method $m$;

\begin{equation}BLUP_{mir} = \hat{\alpha_{m}} + \hat{\beta_{m}}\mu_{i} +
	c_{mi} \end{equation}. Under the assumption that the $\mu$s are
the true item values, this would be sufficient to estimate
parameters. When that assumption doesn't hold, regression
techniques (known as updating techniques) can be used additionally
to determine the estimates. The assumption of exchangeability can
be unrealistic in certain situations. \citet{BXC2004} provides an
amended formulation which includes an extra interaction term ($
d_{mr} \sim N(0,\omega^{2}_{m}$)to account for this.


\newpage
\citet{BXC2008} sets out a methodology of computing the limits of
agreement based upon variance component estimates derived using
linear mixed effects models. Measures of repeatability, a
characteristic of individual methods of measurements, are also
derived using this method.

\citet{BXC2004} also advocates the use of linear mixed models in
the study of method comparisons. The model is constructed to
describe the relationship between a value of measurement and its
real value. The non-replicate case is considered first, as it is
the context of the Bland-Altman plots. This model assumes that
inter-method bias is the only difference between the two methods.
A measurement $y_{mi}$ by method $m$ on individual $i$ is
formulated as follows;
\begin{equation}
	y_{mi}  = \alpha_{m} + \mu_{i} + e_{mi} \qquad ( e_{mi} \sim
	N(0,\sigma^{2}_{m}))
\end{equation}
The differences are expressed as $d_{i} = y_{1i} - y_{2i}$ For the
replicate case, an interaction term $c$ is added to the model,
with an associated variance component. All the random effects are
assumed independent, and that all replicate measurements are
assumed to be exchangeable within each method.

\begin{equation}
	y_{mir}  = \alpha_{m} + \mu_{i} + c_{mi} + e_{mir} \qquad ( e_{mi}
	\sim N(0,\sigma^{2}_{m}), c_{mi} \sim N(0,\tau^{2}_{m}))
\end{equation}

\citet{BXC2008} proposes a methodology to calculate prediction
intervals in the presence of replicate measurements, overcoming
problems associated with Bland-Altman methodology in this regard.
It is not possible to estimate the interaction variance components
$\tau^{2}_{1}$ and $\tau^{2}_{2}$ separately. Therefore it must be
assumed that they are equal. The variance of the difference can be
estimated as follows:
\begin{equation}
	var(y_{1j}-y_{2j})
\end{equation}


%-----------------------------------------------------------------------------------------------------%
\newpage

\subsection{Bendix Carstensen's data sets}
\citet{bxc2008}describes the sampling method when discussing of a motivating example.Diabetes patients attending an outpatient clinic in Denmark have their $HbA_{1c}$ levels routinely measured at every visit.Venous and Capillary blood samples were obtained from all patients appearing at the clinic over two days.

Samples were measured on four consecutive days on each machines, hence there are five analysis days.Carstensen notes that every machine was calibrated every day to  the manufacturers guidelines.

\subsection{Carstensen Methods}
Bendix Carstensen et al. proposed the use of LME models to allow for a more statistically rigourous approach to computing Limits of Agreement.  The respective papers also discuss several shortcoming for techniques for dealing with replicate measurements, as proposed by Bland-Altman 1999.

%---------------------------------------------------------------%
Components

\begin{verbatim}



Section 5.3 Models for replicate measurements
Section 5 Replicate measurements.

Carstensen page 56
%----------------------------------------------------------------%
air extra random effect that does not depend on method.
It is treated as an extension of i.
The variance of air represents the variation between replication condition (common for all methods), within items, .
\end{verbatim}
\[ymir=m+i+cmi+emir\]

\[cmi=N(0,m2)\]

\[emir=N(0,m2)\]

\begin{verbatim}
Carstensen page 58

var(y10-y20) =12+22+12+22

1-2222+12+22

Roy further to Carstensen

ymir=m+i+cmi+emir

\end{verbatim}
%-----------------------------------------------------------------%


Section 7 A general model for method comparisons.

Carstensen discusses the model and its use as if all parameter estimates are available.

In this model, intermethod bias is assumed to be constant at all measurement levels.

i : True value for item i

The parameter i can be thought of as the underlying, but unobtainable, true measurement for item i.

m: Fixed effect for method m

%%-----------------------------------------------------------------%
%
%\subsection{7.2 Interpretation of Random effects}
%
%\begin{itemize}
%	\item method by item
%	\item item by replicate
%	\item method by item by replicate
%\end{itemize}

%Carstensen’s LME model
%LoA as computed by Carstensen’s LME model Papers
%Carstensen et Al 2006
%Carstensen et al 2008
%Bendix Carstensen 2010
% Section 5.3 Models for replicate measurements
% Section 7 A general model for method comparisons.
% Section 7.2 Interpretation of Random effects
%
\textbf{Carstensen et al - Mixed Models}
\begin{itemize}
	\item Carstensen et al [4] also advocates the use of linear mixed models in
	the study of method comparisons. The model is constructed to
	describe the relationship between a value of measurement and its
	real value. 
	
	\item The non-replicate case is considered first, as it is
	the context of the Bland-Altman plots. 
	\item This model assumes that
	\textit{inter-method bias} is the only difference between the two methods.
	A measurement $y_{mi}$ by method $m$ on individual $i$ is
	formulated as follows;
\end{itemize}

\begin{equation}
	y_{mi}  = \alpha_{m} + \mu_{i} + e_{mi} \qquad ( e_{mi} \sim
	N(0,\sigma^{2}_{m}))
\end{equation}

%%%%%%%%%%%%%%%%%%%%%%%%%%%%%%%%%%%%%%%%%%%%%%%%%%%%%%%%%%%%%%%%%%%%%%%%%%%%%%%%%%%%%%

%%%%%%%%%%%%%%%%%%%%%%%%%%%%%%%%%%%%%%%%%%%%%%%%%%%%%%%%%%%%%%%%%%%%%%%%%%%%%%%%%%%%%%

%
% \frametitle{Carstensen's Mixed Models}
\begin{itemize}
	\item Carstensen et al [5] sets out a methodology of computing the limits of
	agreement based upon variance component estimates derived using
	linear mixed effects models. 
	\item Measures of repeatability, a
	characteristic of individual methods of measurements, are also
	derived using this method.
\end{itemize}


%%%%%%%%%%%%%%%%%%%%%%%%%%%%%%%%%%%%%%%%%%%%%%%%%%%%%%%%%%%%%%%%%%%%%%%%%%%%%%%%%%%%%%
%
% \frametitle{Carstensen's Mixed Models}

\begin{itemize}
	\item The differences are expressed as $d_{i} = y_{1i} - y_{2i}$.
	\item For the
	replicate case, an interaction term $c$ is added to the model,
	with an associated variance component. 
	\item All the random effects are
	assumed independent, and that all replicate measurements are
	assumed to be exchangeable within each method.
\end{itemize}


\begin{equation}
	y_{mir}  = \alpha_{m} + \mu_{i} + c_{mi} + e_{mir} \qquad ( e_{mi}
	\sim N(0,\sigma^{2}_{m}), c_{mi} \sim N(0,\tau^{2}_{m}))
\end{equation}
%%%%%%%%%%%%%%%%%%%%%%%%%%%%%%%%%%%%%%%%%%%%%%%%%%%%%%%%%%%%%%%%%%%%%%%%%%%%%%%%%%%%%%


\begin{itemize}
	\item Carstensen \textit{et al} \cite{BXC2004} also advocates the use of linear mixed models in
	the study of method comparisons. 
	\item The model is constructed to
	describe the relationship between a value of measurement and its
	real value.
	\item  The non-replicate case is considered first, as it is
	the context of the Bland Altman plots. This model assumes that
	inter-method bias is the only difference between the two methods.
	A measurement $y_{mi}$ by method $m$ on individual $i$ is
	formulated as follows;
\end{itemize}
\begin{equation}
	y_{mi}  = \alpha_{m} + \mu_{i} + e_{mi} \qquad ( e_{mi} \sim
	N(0,\sigma^{2}_{m}))
\end{equation}


%
\Large
\begin{itemize}
	\item The differences are expressed as $d_{i} = y_{1i} - y_{2i}$ For the
	replicate case, an interaction term $c$ is added to the model,
	with an associated variance component. 
	\item All the random effects are
	assumed independent, and that all replicate measurements are
	assumed to be exchangeable within each method.
\end{itemize}
\begin{eqnarray}
	y_{mir}  = \alpha_{m} + \mu_{i} + c_{mi} + e_{mir} 
\end{eqnarray}

%------------------------------------------------------------------------------ %

\begin{itemize}
	\item The following model (in the authors own notation) is
	formulated as follows, where $y_{mir}$ is the $r$th replicate
	measurement on subject $i$ with method $m$.
\end{itemize}
{

	\begin{equation}
		y_{mir}  = \alpha_{m} + \mu_{i} + c_{mi} + e_{mir} \qquad ( e_{mi}
		\sim N(0,\sigma^{2}_{m}), c_{mi} \sim N(0,\tau^{2}_{m}))
	\end{equation}
	
	
	\begin{equation}
		y_{mir}  = \alpha_{m} + \beta_{m}\mu_{i} + c_{mi} + e_{mir} 
	\end{equation}
}

{

	\[ e_{mi} \sim N(0,\sigma^{2}_{m}), c_{mi} \sim N(0,\tau^{2}_{m})\]
}

%------------------------------------------------------- %
%SLIDE 3
%[fragile]
% \frametitle{Carstensen's Mixed Models}
\begin{itemize}
	\item The intercept term $\alpha$ and the $\beta_{m}\mu_{i}$ term follow
	from \textit{Dunn} \cite{DunnSEME}, expressing constant and proportional bias
	respectively , in the presence of a real value $\mu_{i}.$
	\item $c_{mi}$ is a interaction term to account for replicate, and
	$e_{mir}$ is the residual associated with each observation.
	\item Since variances are specific to each method, this model can be
	fitted separately for each method.
\end{itemize}


%---------------------------------------------------------------- %
%
% \frametitle{Carstensen's Mixed Models}
\begin{itemize}
	\item The above formulation doesn't require the data set to be balanced.
	However, it does require a sufficient large number of replicates
	and measurements to overcome the problem of identifiability. 
	\item The
	import of which is that more than two methods of measurement may
	be required to carry out the analysis. 

	\item There is also the
	assumptions that observations of measurements by particular
	methods are exchangeable within subjects. \item \textbf{\textit{Exchangeability}} means
	that future samples from a population behaves like earlier
	samples).
\end{itemize}

%---------------------------------------------------------------- %

%-----------------------%
%
% \frametitle{Computing LoAs from LME models}
\emph{
	One important feature of replicate observations is that they should be independent
	of each other. In essence, this is achieved by ensuring that the observer makes each
	measurement independent of knowledge of the previous value(s). This may be difficult
	to achieve in practice.}

\subsection*{Using LME models to create Prediction Intervals}


%\[  e_{mi} \sim N(0,\sigma^{2}_{m}) \c_{mi} \sim N(0,\tau^{2}_{m}) \]
%\end{eqnarray}
%


\begin{itemize}
	\item Carstensen \textit{et al} \cite{BXC2008} proposes a methodology to calculate prediction
	intervals in the presence of replicate measurements, overcoming
	problems associated with Bland-Altman methodology in this regard.
	\item It is not possible to estimate the interaction variance components
	$\tau^{2}_{1}$ and $\tau^{2}_{2}$ separately. Therefore it must be
	assumed that they are equal. The variance of the difference can be
	estimated as follows:
	\begin{equation}
		var(y_{1j}-y_{2j})
	\end{equation}
\end{itemize}
%==================================================================== %
\subsection{Using LME models to create Prediction Intervals}
\citet{BXC2004} also advocates the use of linear mixed models in
the study of method comparisons. The model is constructed to
describe the relationship between a value of measurement and its
real value. The non-replicate case is considered first, as it is
the context of the Bland-Altman plots. This model assumes that
inter-method bias is the only difference between the two methods.
A measurement $y_{mi}$ by method $m$ on individual $i$ is
formulated as follows;
\begin{equation}
	y_{mi}  = \alpha_{m} + \mu_{i} + e_{mi} \qquad ( e_{mi} \sim
	N(0,\sigma^{2}_{m}))
\end{equation}
The differences are expressed as $d_{i} = y_{1i} - y_{2i}$ For the
replicate case, an interaction term $c$ is added to the model,
with an associated variance component. All the random effects are
assumed independent, and that all replicate measurements are
assumed to be exchangeable within each method.

\begin{equation}
	y_{mir}  = \alpha_{m} + \mu_{i} + c_{mi} + e_{mir} \qquad ( e_{mi}
	\sim N(0,\sigma^{2}_{m}), c_{mi} \sim N(0,\tau^{2}_{m}))
\end{equation}




%==================================================================== %
\citet{BXC2008} proposes a methodology to calculate prediction
intervals in the presence of replicate measurements, overcoming
problems associated with Bland-Altman methodology in this regard.
It is not possible to estimate the interaction variance components
$\tau^{2}_{1}$ and $\tau^{2}_{2}$ separately. Therefore it must be
assumed that they are equal. The variance of the difference can be
estimated as follows:
\begin{equation}
	var(y_{1j}-y_{2j})
\end{equation}

\subsection{Computation} Modern software
packages can be used to fit models accordingly. The best linear
unbiased predictor (BLUP) for a specific subject $i$ measured with
method $m$ has the form $BLUP_{mir} = \hat{\alpha_{m}} +
\hat{\beta_{m}}\mu_{i} + c_{mi}$, under the assumption that the
$\mu$s are the true item values.




%%%%%%%%%%%%%%%%%%%%%%%%%%%%%%%%%%%%%%%%%%%%%%%%%%%%%%%%%%%%%%%%%%%%%%%%%%%%%%%%%%%%%%%%%%%%%%%%%%%%%%%%% Other Approaches


\newpage



\citet{pkcng} generalize this approach to account for situations
where the distributions are not identical, which is commonly the
case. The TDI is not consistent and may not preserve its
asymptotic nominal level, and that the coverage probability
approach of \citet{lin2002} is overly conservative for moderate
sample sizes. This methodology proposed by \citet{pkcng} is a
regression based approach that models the mean and the variance of
differences as functions of observed values of the average of the
paired measurements.
%%%%%%%%%%%%%%%%%%%%%%%%%%%%%%%%%%%%%%%%%%%%%%%%%%%%%%%%%%%%%%%%%%%%%%%%%%%%%%%%%%%%%%%%%%%%%%%%%%%%%%%%%5

Maximum likelihood estimation is used to estimate the parameters.
The REML estimation is not considered since it does not lead to a
joint distribution of the estimates of fixed effects and random
effects parameters, upon which the assessment of agreement is
based.



\subsection{Carstensen's Mixed Models}

%-----------------------------------------------------------------------------------%
%
% \frametitle{Carstensen model in the single measurement case}
\begin{itemize}
	\item Carstensen \textit{et al}[4] presents a model to describe the relationship between a value of measurement and its real value.
	\item The non-replicate case is considered first, as it is the context of the Bland-Altman plots.
	\item This model assumes that inter-method bias is the only difference between the two methods.
\end{itemize}% Cut This Slide?

Carstensen \textit{et al}[4] proposes linear mixed effects models for deriving
conversion calculations similar to Deming's regression, and for
estimating variance components for measurements by different
methods. The following model ( in the authors own notation) is
formulated as follows, where $y_{mir}$ is the $r$th replicate
measurement on subject $i$ with method $m$.

\begin{equation}
	y_{mir}  = \alpha_{m} + \beta_{m}\mu_{i} + c_{mi} + e_{mir} \qquad
	( e_{mi} \sim N(0,\sigma^{2}_{m}), c_{mi} \sim N(0,\tau^{2}_{m}))
\end{equation}

%%%%%%%%%%%%%%%%%%%%%%%%%%%%%%%%%%%%%%%%%%%%%%%%%%%%%%%%%%%%%%%%%%%%%%%%%%%%%%%%%%%%%%
%
The intercept term $\alpha$ and the $\beta_{m}\mu_{i}$ term follow
from Dunn[7], expressing constant and proportional bias
respectively , in the presence of a real value $\mu_{i}.$
$c_{mi}$ is a interaction term to account for replicate, and
$e_{mir}$ is the residual associated with each observation.
Since variances are specific to each method, this model can be
fitted separately for each method.

%-----------------------------------------------------------------------%
%
% \frametitle{Carstensen's Mixed Models}
\begin{itemize}
	\item This model includes a method by item interaction term.\\
	
	\item Carstensen presents two models. One for the case where the replicates, and a second for when they are linked.\\
	\item Carstensen's model does not take into account either between-item or within-item covariance between methods.\\
	\item In the presented example, it is shown that Roy's LoAs are lower than those of Carstensen.
\end{itemize}



\[\left(\begin{array}{cc}
\omega^1_2  & 0 \\
0 & \omega^2_2 \\
\end{array}  \right)
=  \left(
\begin{array}{cc}
\tau^2  & 0 \\
0 & \tau^2 \\
\end{array} \right)+
\left(
\begin{array}{cc}
\sigma^2_1  & 0 \\
0 & \sigma^2_2 \\
\end{array}\right)
\]









%-----------------------------------------------------------------------------------%
%
% \frametitle{Carstensen model in the single measurement case}

\begin{equation}
	y_{mi}  = \alpha_{m} + \mu_{i} + e_{mi} \qquad  e_{mi} \sim \mathcal{N}(0,\sigma^{2}_{m})
\end{equation}

The differences are expressed as $d_{i} = y_{1i} - y_{2i}$.

For the replicate case, an interaction term $c$ is added to the model, with an associated variance component.
%-------------------------------------------------------------------------------------%
\subsection{Computing LoAs from LME models}


	One important feature of replicate observations is that they should be independent
	of each other. In essence, this is achieved by ensuring that the observer makes each
	measurement independent of knowledge of the previous value(s). This may be difficult
	to achieve in practice.


Carstensen presents a model where the variation between items for
method $m$ is captured by $\sigma_m$ and the within item variation
by $\tau_m$.

Further to his model, Carstensen computes the limits of agreement
as

\[
\hat{\alpha}_1 - \hat{\alpha}_2 \pm \sqrt{2 \hat{\tau}^2 +
	\hat{\sigma}^2_1 + \hat{\sigma}^2_2}
\]

%-------------------------------------------------------------------------------------%


\begin{itemize}
	\item The respective estimates computed by both methods are tabulated as follows. Evidently there is close correspondence between both sets of estimates.
	
	\item \citet{BXC2008} formulates an LME model, both in the absence and the presence of an interaction term.\citet{BXC2008} uses both to demonstrate the importance of using an interaction term. Failure to take the replication structure into
	account results in over-estimation of the limits of agreement. 
	\item For the Carstensen estimates below, an interaction term was included when computed.
\end{itemize}


\section{Carstensen's Limits of agreement}
\citet{BXC2008} presents a methodology to compute the limits of
agreement based on LME models. Importantly, Carstensen's underlying model differs from Roy's model in some key respects, and therefore a prior discussion of Carstensen's model is required.

\subsection{Carstensen 2004's Model}

\citet{BXC2004} presents a model to describe the relationship between a value of measurement and its
real value. The non-replicate case is considered first, as it is the context of the Bland Altman plots. This model assumes that inter-method bias is the only difference between the two methods.

A measurement $y_{mi}$ by method $m$ on individual $i$ is formulated as follows;
\begin{equation}
	y_{mi}  = \alpha_{m} + \mu_{i} + e_{mi} \qquad  e_{mi} \sim
	\mathcal{N}(0,\sigma^{2}_{m})
\end{equation}
The differences are expressed as $d_{i} = y_{1i} - y_{2i}$. For the replicate case, an interaction term $c$ is added to the model, with an associated variance component. All the random effects are assumed independent, and that all replicate measurements are assumed to be exchangeable within each method.

\begin{equation}
	y_{mir}  = \alpha_{m} + \mu_{i} + c_{mi} + e_{mir}, \qquad  e_{mi}
	\sim \mathcal{N}(0,\sigma^{2}_{m}), \quad c_{mi} \sim \mathcal{N}(0,\tau^{2}_{m}).
\end{equation}
%----
\citet{BXC2008} uses an approach based on linear mixed effects (LME) models for the purpose of computing the limits of agreement for two methods of measurement, where replicate measurements are taken on items. As the emphasis of this methodology lies on the inter-method bias and the limits of agreement, the two key elements of the Bland-Altman methodology, other formal tests are not described.

Using Carstensen's notation, a measurement $y_{mi}$ by method $m$ on individual $i$ the measurement $y_{mir} $ is the $r$th replicate measurement on the $i$th item by the $m$th method, where $m=1,2,$ $i=1,\ldots,N,$ and $r = 1,\ldots,n_i$ is formulated as follows;

\begin{equation}
	y_{mir}  = \alpha_{m} + \mu_{i} + c_{mi} + \epsilon_{mir}, \qquad  e_{mi}
	\sim \mathcal{N}(0,\sigma^{2}_{m}), \quad c_{mi} \sim \mathcal{N}(0,\tau^{2}_{m}).
\end{equation}

Here the terms $\alpha_{m}$ and $\mu_{i}$ represent the fixed effect for method $m$ and a true value for item $i$ respectively. The random effect terms comprise an interaction term $c_{mi}$ and the residuals $\epsilon_{mir}$.
The $c_{mi}$ term represent random effect parameters corresponding to the two methods, having $\mathrm{E}(c_{mi})=0$ with $\mathrm{Var}(c_{mi})=\tau^2_m$. Carstensen specifies the variance of the interaction terms as being univariate normally distributed. As such, $\mathrm{Cov}(c_{mi}, c_{m^\prime i})= 0.$ All the random effects are assumed independent, and that all replicate measurements are assumed to be exchangeable within each method.

With regards to specifying the variance terms, Carstensen remarks that using his approach is common, remarking that \emph{
	The only slightly non-standard (meaning "not often used") feature is the differing residual variances between methods }\citep{bxc2010}.

In contrast to Roy's model, Carstensen's model requires that commonly used assumptions be applied, specifically that the off-diagonal elements of the between-item and within-item variability matrices are zero. By
extension the overall variability off-diagonal elements are also zero. Also, implementation requires that the between-item variances are estimated as the same value: $g^2_1 = g^2_2 = g^2$.
As a consequence, Carstensen's method does not allow for a formal test of the between-item variability.

\[\left(\begin{array}{cc}
\omega^1_2  & 0 \\
0 & \omega^2_2 \\
\end{array}  \right)
=  \left(
\begin{array}{cc}
\tau^2  & 0 \\
0 & \tau^2 \\
\end{array} \right)+
\left(
\begin{array}{cc}
\sigma^2_1  & 0 \\
0 & \sigma^2_2 \\
\end{array}\right)
\]


%---Key difference 1---The True Value
%---Colollary -- Difference in model types
The presence of the true value term $\mu_i$ gives rise to an important difference between Carstensen's and Roy's models. The fixed effect of Roy's model comprise of an intercept term and fixed effect terms for both methods, with no reference to the true value of any individual item. In other words, Roy considers the group of items being measured as a sample taken from a population. Therefore a distinction can be made between the two models: Roy's model is a standard LME model, whereas Carstensen's model is a more complex additive model.

%---Carstensen's limits of agreement
%---The between item variances are not individually computed. An estimate for their sum is used.
%---The within item variances are indivdually specified.
%---Carstensen remarks upon this in his book (page 61), saying that it is "not often used".
%---The Carstensen model does not include covariance terms for either VC matrices.
%---Some of Carstensens estimates are presented, but not extractable, from R code, so calculations have to be done by %---hand.
%---All of Roys stimates are  extractable from R code, so automatic compuation can be implemented
%---When there is negligible covariance between the two methods, Roys LoA and Carstensen's LoA are roughly the same.
%---When there is covariance between the two methods, Roy's LoA and Carstensen's LoA differ, Roys usually narrower.


%-----------------------------------------------------------------------------------%

\section{Carstensen's Limits of agreement}
\citet{BXC2008} presents a methodology to compute the limits of agreement based on LME models. The method of computation is the
same as Roy's model, but with the covariance estimates set to zero.

In cases where there is negligible covariance between methods, the limits of agreement computed using Roy's model accord with those computed using Carstensen's model. In cases where some degree of
covariance is present between the two methods, the limits of agreement computed using models will differ. In the presented
example, it is shown that Roy's LoAs are lower than those of Carstensen, when covariance is present.

Importantly, estimates required to calculate the limits of agreement are not extractable, and therefore the calculation must
be done by hand.
%-----------------------------------------------------------------------------------------------------%
\section{Repeated Measurements}

In cases where there are repeated measurements by each of the two
methods on the same subjects , Bland Altman suggest calculating
the mean for each method on each subject and use these pairs of
means to compare the two methods.
The estimate of bias will be unaffected using this approach, but
the estimate of the standard deviation of the differences will be
too small, because of the reduction of the effect of repeated
measurement error. Bland Altman propose a correction for this.
Carstensen attends to this issue also, adding that another
approach would be to treat each repeated measurement separately.
\section{Carstensen 2004 's Mixed Models}

\citet{BXC2004} proposes linear mixed effects models for deriving
conversion calculations similar to Deming's regression, and for
estimating variance components for measurements by different
methods. The following model ( in the authors own notation) is
formulated as follows, where $y_{mir}$ is the $r$th replicate
measurement on subject $i$ with method $m$.

\begin{equation}
	y_{mir}  = \alpha_{m} + \beta_{m}\mu_{i} + c_{mi} + e_{mir} \qquad
	( e_{mi} \sim N(0,\sigma^{2}_{m}), c_{mi} \sim N(0,\tau^{2}_{m}))
\end{equation}
The intercept term $\alpha$ and the $\beta_{m}\mu_{i}$ term follow
from \citet{DunnSEME}, expressing constant and proportional bias
respectively , in the presence of a real value $\mu_{i}.$
$c_{mi}$ is a interaction term to account for replicate, and
$e_{mir}$ is the residual associated with each observation.
Since variances are specific to each method, this model can be
fitted separately for each method.

The above formulation doesn't require the data set to be balanced.
However, it does require a sufficient large number of replicates
and measurements to overcome the problem of identifiability. The
import of which is that more than two methods of measurement may
be required to carry out the analysis. There is also the
assumptions that mobservations of measurements by particular
methods are exchangeable within subjects. (Exchangeability means
that future samples from a population behaves like earlier
samples).

%\citet{BXC2004} describes the above model as a `functional model',
%similar to models described by \citet{Kimura}, but without any
%assumptions on variance ratios. A functional model is . An
%alternative to functional models is structural modelling

\citet{BXC2004} uses the above formula to predict observations for
a specific individual $i$ by method $m$;

\begin{equation}BLUP_{mir} = \hat{\alpha_{m}} + \hat{\beta_{m}}\mu_{i} +
	c_{mi} \end{equation}. Under the assumption that the $\mu$s are
the true item values, this would be sufficient to estimate
parameters. When that assumption doesn't hold, regression techniques (known as updating techniques)
can be used additionally to determine the estimates.
The assumption of exchangeability can be unrealistic in certain situations.
\citet{BXC2004} provides an amended formulation which includes an extra interaction
term ($d_{mr} d_{mr} \sim N(0,\omega^{2}_{m}$)to account for this.

\citet{BXC2008} sets out a methodology of computing the limits of
agreement based upon variance component estimates derived using
linear mixed effects models. Measures of repeatability, a
characteristic of individual methods of measurements, are also
derived using this method.


\section{LME}
Consistent with the conventions of mixed models, \citet{pkc}
formulates the measurement $y_{ij} $from method $i$ on individual
$j$ as follows;
\begin{equation}
	y_{ij} =P_{ij}\theta + W_{ij}v_{i} + X_{ij}b_{j} + Z_{ij}u_{j} +
	\epsilon_{ij},     (j=1,2, i=1,2....n)
\end{equation}
The design matrix $P_{ij}$ , with its associated column vector
$\theta$, specifies the fixed effects common to both methods. The
fixed effect specific to the $j$th method is articulated by the
design matrix $W_{ij}$ and its column vector $v_{i}$. The random
effects common to both methods is specified in the design matrix
$X_{ij}$, with vector $b_{j}$ whereas the random effects specific
to the $i$th subject by the $j$th method is expressed by $Z_{ij}$,
and vector $u_{j}$. Noticeably this notation is not consistent
with that described previously.  The design matrices are specified
so as to includes a fixed intercept for each method, and a random
intercept for each individual. Additional assumptions must also be
specified;
\begin{equation}
	v_{ij} \sim N(0,\Sigma),
\end{equation}
These vectors are assumed to be independent for different $i$s,
and are also mutually independent. All Covariance matrices are
positive definite.  In the above model effects can be classed as
those common to both methods, and those that vary with method.
When considering differences, the effects common to both
effectively cancel each other out. The differences of each pair of
measurements can be specified as following;
\begin{equation}
	d_{ij} = X_{ij}b_{j} + Z_{ij}u_{j} + \epsilon_{ij},     (j=1,2,
	i=1,2....n)
\end{equation}
This formulation has seperate distributional assumption from the
model stated previously.

This agreement covariate $x$ is the key step in how this
methodology assesses agreement.
%%%%%%%%%%%%%%%%%%%%%%%%%%%%%%%%%%%%%%%%%%%%%%%%%%%%%%%%%%%%%%%%%%%%%%%%%%%%%%%%%%%%%%%%%%%%%%%%%%%%%%%5

\bibliography{DB-txfrbib}
\end{document}
