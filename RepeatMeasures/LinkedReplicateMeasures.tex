\subsection{Repeated Measurements }
In cases where there are repeated measurements by each of the two
methods on the same subjects , Bland Altman suggest calculating
the mean for each method on each subject and use these pairs of
means to compare the two methods.
\\
The estimate of bias will be unaffected using this approach, but
the estimate of the standard deviation of the differences will be
too small, because of the reduction of the effect of repeated
measurement error. Bland Altman propose a correction for this.
\\
Carstensen attends to this issue also, adding that another
approach would be to treat each repeated measurement separately.

%%%%%%%%%%%%%%%%%%%%%%%%%%%%%%%%%%%%%%%%%%%%%%%%%%%%%%%%%%%%%%%%%%%%%%%%%%%%%%%%%%%%%%%%%%%%%%%%%%%%%%%%%%%%%%%

In this model , the variances of the random effects must depend on
$m$, since the different methods do not necessarily measure on the
same scale, and different methods naturally must be assumed to
have different variances. \citet{BXC2004} attends to the issue of
comparative variances.
%----------------------------------------------------------------------------%
\section{Definition of Replicate measurements}
Further to \citet{BA99}, a formal definition is required of what exactly replicate measurements are

\emph{By replicates we mean two or more measurements on the same
individual taken in identical conditions. In general this requirement means that the
measurements are taken in quick succession.}

\citet{BA99} also remark that an important feature of replicate observations is that they should be independent
of each other. This issue is addressed by \citet{bxc2010}, in terms of exchangeability and linkage. Carstenen advises that repeated measurements come in two \emph{substantially different} forms, depending on the circumstances of their measurement: exchangable and linked.
%----------------------------------------------------------------------------%
\subsection{Exchangeable measurements}
Repeated measurements are said to be exchangeable if no relationship exists between successive measurements across measurements. If the condition of exchangeability exists, a group of measurement of the same item determined by the same method can be re-arranged in any permutation without prejudice to proper analysis. There is no reason to believe that the true value of the underlying variable has changed over the course of the measurements.

For the purposes of method comparison studies the following remarks can be made. The $r-$th measurement made by method $1$ has no special correspondence to the $r-$th measurement made by method $2$, and consequently any pairing of repeated measurements are as good as each other.

Exchangeable repeated measurements can be treated as true replicates.
%----------------------------------------------------------------------------%
\subsection{Linked measurements}
Repeated measurements are said to be linked if a direct correspondence exists between successive measurements across measurements, i.e. pairing. Such measurements are commonly made with a time interval between them, but simultaneously for both methods. Paired measurements are exchangeable, but individual measurements are not.

If the paired measurements are taken
in a short period of time so that no real systemic changes can take place on each item, they can be considered true replicates.
Should enough time elapse for systemic changes, linked repeated measurements can not be treated as true replicates.

\subsection{Replicate measurements in Roy's paper}
\citet{Roy} takes its definition of replicate measurement: two or more measurements on the same item taken
under identical conditions. Roy also assumes linked measurements, but it is can be used for the non-linked case.

%----------------------------------------------------------------------------------------------------%
\newpage
\subsection{Random effects}

Further to \citet{barnhart}, if the measurements by a method on an item are not necessarily true replications, e.g., repeated measures over time, then additional terms may be needed for $e_{mir}$. \citet{bxc2008} also addresses this issue by the addition of an interaction term (i.e. a random effect) $u_mi$, yielding

\[ y_{mir} =  \alpha_{mi} + u_{mi} + e_{mi}.  \]

The additional interaction term is characterized as $u_{mi}  \sim \mathcal{N}(0, \tau^2_m)$ \citep{bxc2008}.

This extra interaction term provides a source of extra variability, but this variance is not relevant to computing the case-wise differences.

\citet{bxc2008} advises that the formulation of the model should take the exchangeability (in other words, whether or not the measurements are `true replicates') into account. If there is a linkage between measurements (therefore not `true' replicates) , the `item by replicate' should be included in the model. If there is no linkage, and the replicates are indeed true replicates, the interaction term should be omitted.

\citet{bxc2008} demonstrates how to compute the limits of agreement for two methods in the case of linked measurements. As a surplus source of variability is excluded from the computation, the limits of agreement are not unduly wide, which would have been the case if the measurements were treated as true replicates.

\citet{Roy} also assigns a random effect $u_{mi}$ for each response $y_{mir}$. Importantly Roy's model assumes linkage.

%----------------------------------------------------------------------------%
\section{Model for replicate measurements}

We generalize the single measurement model for the replicate measurement case, by additionally specifying replicate values. Let $y_{mir}$ be the $r-$th replicate measurement for subject ``i" made by method ``m". Further to \citet{barnhart} fixed effect can be expressed with a single term $\alpha_{mi}$, which incorporate the true value $\mu_i$.

\[ y_{mir} = \mu_{i} + \alpha_{m} + e_{mir}  \]

Combining fixed effects \citep{barnhart}, we write,

\[ y_{mir} = \alpha_{mi} + e_{mir}.\]

The following assumptions are required

\begin{itemize}
  \item $e_{mir}$ is independent of the fixed effects with mean $\mbox{E}(e_{mir}) = 0$.
  \item Further to \citet{barnhart} between-item and within-item variances $\mbox{Var}(\alpha_{mi}) = \sigma^2_{Bm}$ and $\mbox{Var}(e_{mir}) = \sigma^2_{Wm}$
  \item In keeping with \citet{Roy}, these variance shall be considered as part of the between-item variance covariance matrix $\boldsymbol{D}$ and the within-item variance covariance matrix  $\boldsymbol{\Sigma}$
        respectively, and will be denoted accordingly ( i.e. $d^2_{m}$ and $\sigma^2_{m}$).
 \item Additionally, the total variability of method "m", denoted $\omega^2_m$ is the sum of the within-item and between-item variabilities.

    \[ \omega^2_m = d^2_{m}+ \sigma^2_{m} \]

\end{itemize}
%----------------------------------------------------------------------------%
\newpage
