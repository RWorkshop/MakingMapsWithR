
\subsection{Using LME models to create Prediction Intervals}
\citet{BXC2004} also advocates the use of linear mixed models in
the study of method comparisons. The model is constructed to
describe the relationship between a value of measurement and its
real value. The non-replicate case is considered first, as it is
the context of the Bland Altman plots. This model assumes that
inter-method bias is the only difference between the two methods.
A measurement $y_{mi}$ by method $m$ on individual $i$ is
formulated as follows;
\begin{equation}
y_{mi}  = \alpha_{m} + \mu_{i} + e_{mi} \qquad ( e_{mi} \sim
N(0,\sigma^{2}_{m}))
\end{equation}
The differences are expressed as $d_{i} = y_{1i} - y_{2i}$ For the
replicate case, an interaction term $c$ is added to the model,
with an associated variance component. All the random effects are
assumed independent, and that all replicate measurements are
assumed to be exchangeable within each method.

\begin{equation}
y_{mir}  = \alpha_{m} + \mu_{i} + c_{mi} + e_{mir} \qquad ( e_{mi}
\sim N(0,\sigma^{2}_{m}), c_{mi} \sim N(0,\tau^{2}_{m}))
\end{equation}

\citet{BXC2008} proposes a methodology to calculate prediction
intervals in the presence of replicate measurements, overcoming
problems associated with Bland-Altman methodology in this regard.
It is not possible to estimate the interaction variance components
$\tau^{2}_{1}$ and $\tau^{2}_{2}$ separately. Therefore it must be
assumed that they are equal. The variance of the difference can be
estimated as follows:
\begin{equation}
var(y_{1j}-y_{2j})
\end{equation}

\subsection{Computation} Modern software
packages can be used to fit models accordingly. The best linear
unbiased predictor (BLUP) for a specific subject $i$ measured with
method $m$ has the form $BLUP_{mir} = \hat{\alpha_{m}} +
\hat{\beta_{m}}\mu_{i} + c_{mi}$, under the assumption that the
$\mu$s are the true item values.


\subsection{Using LME models to create Prediction Intervals}
\citet{BXC2004} also advocates the use of linear mixed models in
the study of method comparisons. The model is constructed to
describe the relationship between a value of measurement and its
real value. The non-replicate case is considered first, as it is
the context of the Bland Altman plots. This model assumes that
inter-method bias is the only difference between the two methods.
A measurement $y_{mi}$ by method $m$ on individual $i$ is
formulated as follows;
\begin{equation}
y_{mi}  = \alpha_{m} + \mu_{i} + e_{mi} \qquad ( e_{mi} \sim
N(0,\sigma^{2}_{m}))
\end{equation}
The differences are expressed as $d_{i} = y_{1i} - y_{2i}$ For the
replicate case, an interaction term $c$ is added to the model,
with an associated variance component. All the random effects are
assumed independent, and that all replicate measurements are
assumed to be exchangeable within each method.

\begin{equation}
y_{mir}  = \alpha_{m} + \mu_{i} + c_{mi} + e_{mir} \qquad ( e_{mi}
\sim N(0,\sigma^{2}_{m}), c_{mi} \sim N(0,\tau^{2}_{m}))
\end{equation}

\citet{BXC2008} proposes a methodology to calculate prediction
intervals in the presence of replicate measurements, overcoming
problems associated with Bland-Altman methodology in this regard.
It is not possible to estimate the interaction variance components
$\tau^{2}_{1}$ and $\tau^{2}_{2}$ separately. Therefore it must be
assumed that they are equal. The variance of the difference can be
estimated as follows:
\begin{equation}
var(y_{1j}-y_{2j})
\end{equation}

\subsection{Computation} Modern software
packages can be used to fit models accordingly. The best linear
unbiased predictor (BLUP) for a specific subject $i$ measured with
method $m$ has the form $BLUP_{mir} = \hat{\alpha_{m}} +
\hat{\beta_{m}}\mu_{i} + c_{mi}$, under the assumption that the
$\mu$s are the true item values.
