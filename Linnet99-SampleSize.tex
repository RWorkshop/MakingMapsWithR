\subsection{Background} 
In method comparison studies, it is of importance to assure that the presence of a difference of medical importance is detected. 
For a given difference, the necessary number of samples depends on the range of values and the analytical standard deviations of the methods involved. For typical examples, the present study evaluates the statistical power of least-squares and Deming regression analyses applied to the method comparison data.

\subsection{Methods} 
Theoretical calculations and simulations were used to consider the statistical power for detection of slope deviations from 
unity and intercept deviations from zero. For situations with proportional analytical standard deviations, weighted forms of regression analysis were evaluated.

\subsection{Results} In general, sample sizes of 40–100 samples conventionally used in method comparison studies often must 
be reconsidered. A main factor is the range of values, which should be as wide as possible for the given analyte. 
For a range ratio (maximum value divided by minimum value) of 2, 544 samples are required to detect one standardized slope 
deviation; the number of required samples decreases to 64 at a range ratio of 10 (proportional analytical error). For electrolytes having very narrow ranges of values, very large sample sizes usually are necessary. In case of proportional analytical error, application of a weighted approach is important to assure an efficient analysis; e.g., for a range ratio of 10, the weighted approach reduces the requirement of samples by >50%.

\subsection{Conclusions} Estimation of the necessary sample size for a method comparison study assures a valid result; either no difference is found or the existence of a relevant difference is confirmed.


\end{document}

%------------------------------------------------------------------%
