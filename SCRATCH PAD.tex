\documentclass[12pt, a4paper]{report}
\usepackage{natbib}
\usepackage{vmargin}
\usepackage{graphicx}
\usepackage{epsfig}
\usepackage{subfigure}
%\usepackage{amscd}
\usepackage{amssymb}
\usepackage{amsbsy}
\usepackage{amsthm, amsmath}
%\usepackage[dvips]{graphicx}
\bibliographystyle{chicago}
\renewcommand{\baselinestretch}{1.8}

% left top textwidth textheight headheight % headsep footheight footskip
\setmargins{3.0cm}{2.5cm}{15.5 cm}{23.5cm}{0.5cm}{0cm}{1cm}{1cm}

\pagenumbering{arabic}


\begin{document}
\author{Kevin O'Brien}
\title{SCRATCH}
\date{\today}
\maketitle

\tableofcontents \setcounter{tocdepth}{2}

%%%%%%%%%%%%%
%1 Method Comparison Studies            %%%%%%%%%%%%%%%%%%%%%%%%%%%%%%%%%
%%%%%%%%%%%%%%%%%%%%%%%%%%%%%%%%%%%%%%%%%%%%%%%%%%%%%%%%%%%%%%%%%%%%%%%%%

\section{Bland Altman Methodology}

\subsection{Bias}
Bland and Altman define bias a \emph{a consistent tendency for one
method to exceed the other} [$3$] and propose estimating its value
by determining the mean of the differences. The variation about
this mean shall be estimated by the  standard deviation of the
differences. Bland and Altman remark that these estimates are based on the
assumption that bias and variability are constant throughout the
range of measures.

\section{The Bland Altman Plot}
In 1986 Bland and Altman published a paper in the Lancet proposing
the difference plot for use for method comparison purposes. It has
proved highly popular ever since. This is a simple, and widely
used , plot of the differences of each data pair, and the
corresponding average value. An important requirement is that the
two measurement methods use the same scale of measurement.
\\
Variations of the Bland Altman plot is the use of ratios, in the
place of differences.
\begin{equation}
D_{i} = X_{i} - Y_{i}   \label{BA01}
\end{equation}
Altman and Bland suggest plotting the within subject differences $
D = X_{1} - X_{2} $ on the ordinate versus the average of $x_{1}$
and  $x_{2}$ on the abscissa.



\subsection{Criticism of Bland Altman Plot}
Unfortunately the Bland-Altman plot has a fatal flaw: it indicates
incorrectly that there are systematic differences or bias in the
relationship between two measures, when one has been calibrated
against the other. (Hopkins)
\subsection{Treatment of Outliers}
Bland and Altman attend to the issue of outliers in their 1986
paper, wherein they present a data set with an extreme outlier



\section{Paired T tests}
This method can be applied to test for statisitcally significant
deviations in bias. This method can be potentially misused for
method comparison studies.
\\It is a poor measure of agreement when the rater's measurements
are perpendicular to the line of equality[Hutson et al]. In this
context, an average difference of zero between the two raters, yet
the scatter plot displays strong negative correlation.

\subsection*{Components in assessing agreement}

\begin{enumerate}
\item The degree of linear relationship between the two sets \item
The amount of bias as represented by the difference in the
means\item The Differences in the two variances.
\end{enumerate}

\section{Methods of assessing agreement}

\begin{enumerate}
\item Pearson's Correlation Coefficient\item Intraclass
correlation coefficient \item Bland Altman Plot \item Bartko's
Ellipse (1994) \item Blackwood Bradley Test \item Lin's
Reproducibility Index \item Luiz Step function
\end{enumerate}

Bland and Altman attend to the issue of repeated measures in
$1996$.
\\
Repeated measurements on several subjects can be used to quantify
measurement error, the variation between measurements of the same
quantity on the same individual.
\\
Bland and Altman discuss two metrics for measurement error; the
within-subject standard deviation ,and the correlation
coefficient.

The above plot incorporates both the conventional limits of
agreement ( the inner pair of dashed lines), the `t' limits of
agreement ( the outer pair of dashed lines) centred around the
inter-method bias (indicated by the full line). This plot is
intended for expository purposes only, as the sample size is
small.





\subsection{Equivalence and Interchangeability}
Limits of agreement are intended to analyse equivalence. How this
is assessed is the considered judgement of the practitioner. In
\citet{BA86} an example of good agreement is cited. For two
methods of measuring `oxygen saturation', the limits of agreement
are calculated as (-2.0,2.8).A practitioner would ostensibly find
this to be sufficiently narrow.

If the limits of agreement are not clinically important, which is
to say that the differences tend not to be substantial, the two
methods may be used interchangeably. \citet{DunnSEME} takes issue
with the notion of `equivalence', remarking that while agreement
indicated equivalence, equivalence does not reflect agreement.




\section{Bland Altman Plots In Literature}
\citet{mantha} contains a study the use of Bland Altman plots of
44 articles in several named journals over a two year period. 42
articles used Bland Altman's limits of agreement, wit the other
two used correlation and regression analyses. \citet{mantha}
remarks that 3 papers, from 42 mention predefined maximum width
for limits of agreement which would not impair medical care.

The conclusion of \citet{mantha} is that there are several
inadequacies and inconsistencies in the reporting of results ,and
that more standardization in the use of Bland Altman plots is
required. The authors recommend the prior determination of limits
of agreement before the study is carried out. This contention is
endorsed by \citet{lin}, which makes a similar recommendation for
the sample size, noting that\emph{sample sizes required either was
not mentioned or no rationale for its choice was given}.

\begin{quote}
In order to avoid the appearance of "data dredging", both the
sample size and the (limits of agreement) should be specified and
justified before the actual conduct of the trial. \citep{lin}
\end{quote}

\citet{Dewitte} remarks that the limits of agreement should be
compared to a clinically acceptable difference in measurements.
%%%%%%%%%%%%%%%%%%%%%%%%%%%%%%%%%%%%%%%%%%%%%%%%%%%%%%%%%%%%%%%%%%%%%%%%%
%4 Inappropriate assessment of Agreement       %%%%%%%%%%%%%%%%%%%%%%%%%%
%%%%%%%%%%%%%%%%%%%%%%%%%%%%%%%%%%%%%%%%%%%%%%%%%%%%%%%%%%%%%%%%%%%%%%%%%


\subsection{Gold Standard} This is considered to be the most
accurate measurement of a particular parameter.
\section{Discussion on Method Comparison Studies}

The need to compare the results of two different measurement
techniques is common in medical statistics.
\\
\\
In particular, in medicine, new methods or devices that are
cheaper, easier to use, or less invasive, are routinely developed.
Agreement between a new method and a traditional reference or gold
standard must be evaluated before the new one is put into
practice. Various methodologies have been proposed for this
purpose in recent years.

Indications on how to deal with outliers in Bland Altman plots
\\
We wish to determine how outliers should be treated in a Bland
Altman Plot
\\
In their 1983 paper they merely state that the plot can be used to
'spot outliers'.
\\
In  their 1986 paper, Bland and Altman give an example of an
outlier. They state that it could be omitted in practice, but make
no further comments on the matter.
\\
In Bland and Altmans 1999 paper, we get the clearest indication of
what Bland and Altman suggest on how to react to the presence of
outliers. Their recommendation is to recalculate the limits
without them, in order to test the difference with the calculation
where outliers are retained.\\

The span has reduced from 77 to 59 mmHg, a noticeable but not
particularly large reduction.
\\
However, they do not recommend removing outliers. Furthermore,
they say:
\\
We usually find that this method of analysis is not too sensitive
to one or two large outlying differences.
\\
We ask if this would be so in all cases. Given that the limits of
agreement may or may not be disregarded, depending on their
perceived suitability, we examine whether it would possible that
the deletion of an outlier may lead to a calculation of limits of
agreement that are usable in all cases?
\\
Should an Outlying Observation be omitted from a data set? In
general, this is not considered prudent.
\\
Also, it may be required that the outliers are worthy of
particular attention themselves.
\\
Classifying outliers and recalculating We opted to examine this
matter in more detail. The following points have to be considered
\\how to suitably identify an outlier (in a generalized sense)
\\Would a recalculation of the limits of agreement generally
results in  a compacted range between the upper and lower limits
of agreement?
\subsection{Agreement} Bland and Altman (1986) define Perfect
agreement as 'The case where all of the pairs of rater data lie
along the line of equality'. The Line of Equality is defined as
the 45 degree line passing through the origin, or X=Y on a XY
plane.

\subsection{Lack Of Agreement}
\begin{enumerate}
\item Constant Bias\item Proportional Bias
\end{enumerate}

\subsubsection*{Constant Bias} This is a form of systematic
deviations estimated as the average difference between the test
and the reference method


\subsubsection*{Proportional Bias} Two methods may agree on
average, but they may exhibit differences over a range of
measurements\section{Bland Altman Plot} Bland Altman have
recommended the use of graphical techniques to assess agreement.
Principally their method is calculating , for each pair of
corresponding two methods of measurement of some underlying
quantity, with no replicate measurements, the difference and mean.
Differences are then plotted against the mean.
\\
Hopkins argued that the bias in a subsequent Bland-Altman plot was
due, in part, to using least-squares regression at the calibration
phase.

\subsection{Bland Altman plots using 'Gold Standard' raters}
According to Bland and Altman, one should use the methodology
previous outlined, even when one of the raters is a Gold Standard.


\subsection{Bias Detection}
further to this method, the presence of constant bias may be
indicated if the average value differences is not equal to zero.
Bland and Altman does, however, indicate the indication of absence
of bias does not provide sufficient information to allow a
judgement as to whether or not one method can be substituted for
another.


\chapter{The Bland Altman Plot}
\section{Bland Altman Plots}
The issue of whether two measurement methods are comparable to the
extent that they can be used interchangeably with sufficient
accuracy is encountered frequently in scientific research.
Historically comparison of two methods of measurement was carried
out by use of matched pairs correlation coefficients or simple
linear regression. Bland and Altman recognized the inadequacies of
these analyses and articulated quite thoroughly the basis on which
of which they are unsuitable for comparing two methods of
measurement \citep*{BA83}.

As an alternative they proposed a simple statistical methodology
specifically appropriate for method comparison studies. They
acknowledge that there are other valid methodologies, but argue
that a simple approach is preferable to complex approaches,
\emph{"especially when the results must be explained to
non-statisticians"} \citep*{BA83}.

The first step recommended which the authors argue should be
mandatory is construction of a simple scatter plot of the data.
The line of equality ($X=Y$) should also be shown, as it is
necessary to give the correct interpretation of how both methods
compare. A scatter plot of the Grubbs data is shown in figure 2.1.
A visual inspection thereof confirms the previous conclusion that
there is an inter method bias present, i.e. Fotobalk device has a
tendency to record a lower velocity.



In light of shortcomings associated with scatterplots,
\citet*{BA83} recommend a further analysis of the data. Firstly
differences of measurements of two methods on the same subject
should  be calculated, and then the average of those measurements
(Table 1.1). The averages of the two measurements is considered by
Bland and Altman to the best estimate for the unknown true value.
Importantly both methods must measure with the same units. These
results are then plotted, with differences on the ordinate and
averages on the abscissa (figure 1.2). \citet*{BA83}express the
motivation for this plot thusly:
\begin{quote}
"From this type of plot it is much easier to assess the magnitude
of disagreement (both error and bias), spot outliers, and see
whether there is any trend, for example an increase in
(difference) for high values. This way of plotting the data is a
very powerful way of displaying the results of a method comparison
study."
\end{quote}
\newpage
% latex table generated in R 2.6.0 by xtable 1.5-5 package
% Thu Aug 27 16:31:52 2009
\begin{table}[tbh]
\begin{center}

\begin{tabular}{|c|c|c|c|c|}
  \hline
 Round & Fotobalk [F] & Counter [C] & Differences [F-C] & Averages [(F+C)/2] \\
  \hline
1 & 793.80 & 794.60 & -0.80 & 794.20 \\
  2 & 793.10 & 793.90 & -0.80 & 793.50 \\
  3 & 792.40 & 793.20 & -0.80 & 792.80 \\
  4 & 794.00 & 794.00 & 0.00 & 794.00 \\
  5 & 791.40 & 792.20 & -0.80 & 791.80 \\
  6 & 792.40 & 793.10 & -0.70 & 792.80 \\
  7 & 791.70 & 792.40 & -0.70 & 792.00 \\
  8 & 792.30 & 792.80 & -0.50 & 792.50 \\
  9 & 789.60 & 790.20 & -0.60 & 789.90 \\
  10 & 794.40 & 795.00 & -0.60 & 794.70 \\
  11 & 790.90 & 791.60 & -0.70 & 791.20 \\
  12 & 793.50 & 793.80 & -0.30 & 793.60 \\
   \hline
\end{tabular}
\caption{Fotobalk and Counter Methods: Differences and Averages}
\end{center}
\end{table}




\subsection{Repeated Measurements }
In cases where there are repeated measurements by each of the two
methods on the same subjects , Bland Altman suggest calculating
the mean for each method on each subject and use these pairs of
means to compare the two methods.
\\
The estimate of bias will be unaffected using this approach, but
the estimate of the standard deviation of the differences will be
too small, because of the reduction of the effect of repeated
measurement error. Bland Altman propose a correction for this.
\\
Carstensen attends to this issue also, adding that another
approach would be to treat each repeated measurement separately.

\subsection{Criticism of Bland Altman Plot}
Hopkins[$8$] argues that the plot indicates incorrectly that there
are systematic differences or bias in the relationship between two
measures, when one has been calibrated against the other.
\\
An Evaluation of the correlation between the difference and means
complement the analysis.
\\
Bland and Altman caution that the calculations are based on the
assumption that the data is normally distributed. This can be
verified by using a histogram. If Data is not normally
distributed, it can be transformed.









\chapter{REGRESSION}%%%%%%%%%%%%%%%%%%%%%%%%%%%%%%%%%%%%%%%%%%%%%%%%%%%%%%%%%%%%%%%%%%%%%%%%%%%%%%%%%%%%%%%%%%%%%%%%% Regression
\section{Model II Regression}
\subsection{Simple Linear Regression} Simple Linear Regression is  well
known statistical technique , wherein estimates for slope and
intercept of the line of best fit are derived according to the
Ordinary Least Square (OLS) principle.This method is known to
Cornbleet and Cochrane as Model I regression.
\\
\\
In Model I regression, the independent variable is assumed to be
measured without error. For method comparison studies, both sets
of measurement must be assumed to be measured with imprecision and
neither case can be taken to be a reference method. Arbitrarily
selecting either method as the reference will yield two
conflicting outcomes. A fitting based on '$X$ on $Y$' will give
inconsistent results with a fitting based on '$Y$ on $X$'.
Consequently model I regression is inappropriate for such cases.
\\
\\
Conversely, Cornbleet Cochrane state that when the independent
variable $X$ is a precisely measured reference method, Model I
regression may be considered suitable. They qualify this statement
by referring the $X$ as \emph{the 'correct' value}, tacitly
implying that there must still be some measurement error present.
The validity of this approach has been disputed elsewhere.




\subsection{Model II regression}
Cochrane and Cornbleet argue for the use of methods that based on
the assumption that both methods are imprecisely measured ,and
that yield a fitting that is consistent with both '$X$ on $Y$' and
'$Y$ on $X$' formulations. These methods uses alternatives to the
OLS approach to determine the slope and intercept.
\\
They describe three such alternative methods of regression; Deming
, Mandel, and Bartlett regression. Collectively the authors refer
to these approaches as Model II regression techniques.

%%%%%%%%%%%%%%%%%%%%%%%%%%%%%%%%%%%%%%%%%%%%%%%%%%%%%%%%%%%%%%%%%%%%%%%%%

\subsection{Distribution of Maxima} It is possible to use Order
Statistics theory to assess conditional probabilities. With two
random variables $T_{0}$ and $T_{1}$, we define two variables $Z$
and $W$ such that they take the maximum and minimum values of the
pair of $T$ values.\subsection{Plot of the Maxima against the
Minima}


In Figure 1,  The Maximas are plotted against their corresponding
minima. The Critical values of the Maxima and Minima are displayed
in the dotted lines.The Line of Equality depicts the obvious
logical constraint of the each Maximum value being greater than
its corresponding minimum value.



The scientific question at hand is the correct approach to
assessing whether two methods can be used interchangeably.
\citet{BA99} expresses this as follows:
\begin{quote}We want to
know by how much (one) method is likely to differ from the
(other), so that if it not enough to cause problems in the
mathematical interpretation we can ... use the two
interchangeably.
\end{quote}



Consequently, of the categories of method comparison study,
comparison studies, the second category, is of particular
importance, and the following discussion shall concentrate upon
it. Less emphasis shall be place on the other three categories.

 \bigskip Further to \citet{BA86}, 'equivalence' of two methods expresses
 that both can be used interchangeably.
\citet[p.49]{DunnSEME} remarks that this is a very restrictive
interpretation of equivalence, and that while agreement indicated
equivalence, equivalence does not necessarily reflect agreement.

The main difference between Myers proposed method and the Bland
Altman is that the random effects model is used to estimate the
within-subject variance after adjusting for known and unknown
variables. The Bland Altman approach uses one way analysis of
variance to estimate the within subject variance. In general, the
random effects model is an extension of the analysis of the ANOVA
method and it can adjust for many more covariates than the ANOVA
method



\subsection{Criticism of Bland Altman Plots}

An Evaluation of the correlation between the difference and means
complement the analysis.
\\
Bland and Altman caution that the calculations are based on the
assumption that the data is normally distributed. This can be
verified by using a histogram. If Data is not normally
distributed, it can be transformed.
\\
Luiz \emph{et al} remarks that that Bland Altman Plot should be
used only for small data sets, as the use of an index will be of
little value to the analysis.


\newpage
\section{Conclusions about Existing Methodologies}

Scatterplots are recommended by \citet{BA83} for an initial
examination of the data, facilitating an initial judgement and
helping to identify potential outliers. They are not useful for a
thorough examination of the data. \citet{BritHypSoc} notes that
data points will tend to cluster around the line of equality,
obscuring interpretation.


The Bland Altman methodology is well noted for its ease of use,
and can be easily implemented with most software packages. Also it
doesn't require the practitioner to have more than basic
statistical training. The plot is quite informative about the
variability of the differences over the range of measurements. For
example, an inspection of the plot will indicate the 'fan effect'.
They also can be used to detect the presence of an outlier.

 \citet{ludbrook97,ludbrook02}criticizes these plots on the
basis that they presents no information on effect of constant bias
or proportional bias. These plots are only practicable when both
methods measure in the same units. Hence they are totally
unsuitable for conversion problems. The limits of agreement are
somewhat arbitrarily constructed. They may or may not be suitable
for the data in question. It has been found that the limits given
are too wide to be acceptable. There is no guidance on how to deal
with outliers. Bland and Altman recognize effect they would have
on the limits of agreeement, but offer no guidance on how to
correct for those effects.

There is no formal testing procedure provided. Rather, it is upon
the practitioner opinion to judge the outcome of the methodology.






%%%%%%%%%%%%%%%%%%%%%%%%%%%%%%%%%%%%%%%%%%%%%%%%%%%%%%%%%%%%%%%%%%%%%%%%%
%9 Appendix                  %%%%%%%%%%%%%%%%%%%%%%%%%%%%%%%%%%%%%%%%%%%%%
%%%%%%%%%%%%%%%%%%%%%%%%%%%%%%%%%%%%%%%%%%%%%%%%%%%%%%%%%%%%%%%%%%%%%%%%%

\chapter{Appendix}


\subsection{Contention }
Several papers have commented that this approach is undermined
when the basic assumptions underlying linear regression are not
met, the regression equation, and consequently the estimations of
bias are undermined. Outliers are a source of error in regression
estimates.In method comparison studies, the X variable is a
precisely measured reference method. Cornbleet Gochman (1979)
argued that criterion may be regarded as the correct value. Other
papers dispute this.
\subsection{Least Product Regression}
Least Product Regression , also known as 'Model II regression'
caters for cases in which random error is attached to both
dependent and independent variables. Ludbrook cites this
methodology as being pertinent to Method comparison studies.
\\
\\
The sum of the products of the vertical and horizontal deviations
of the x,y values from the line is minimized.
\\
\\
Least products regression analysis is considered suitable for
calibrating one method against another.Ludbrook comments that it
is also a sensitive technique for detecting and distinguishing
fixed and proportional bias between methods.
\\
\\
Proposed as an alternative to Bland \& Altman methodology ,this
method is also known as 'Geometric Mean Regression' and 'Reduced
Major Axis Regression'.
%%%%%%%%%%%%%%%%%%%%%%%%%%%%%%%%%%%%%%%%%%%%%%%%%%%%%%
%%%%%%%%%%%%%%%%%%%%%%%%%%%%%%%%%%%%%%%%%%%%%%%%%%%%%%%%%%%%%%%%%%%
%%%%%%%%%%%%%%%%%%%%%%%%%%%%%%%%%%%%%%%%%%%%%%%%%%%%%%%%%%%%%%%%%%%%%%%%%%%%%%%

\subsubsection{Difference with Least Squares Regression}
Least-products regression can lead to inflated SEEs and estimates
that do not tend to their true values an N approaches infinity
(Draper and Smith, 1998).



\subsection{Ordinary Least Product Regression}
\citet{ludbrook97} states that the grouping structure can be
straightforward, but there are more complex data sets that have a
hierarchical(nested) model.
\\
\\
Observations between groups are independent, but observations
within each groups are dependent because they belong to the same
subpopulation. Therefore there are two sources of variation:
between-group and within-group variance.
 \vspace{5 mm} \noindent Mean correction is a method of reducing
bias.







\subsection{A regression based approach based on Bland Altman Analysis}
Lu et al used such a technique in their comparison of DXA
scanners. They also used the Blackwood Bradley test. However it
was shown that, for particular comparisons,  agreement between
methods was indicated according to one test, but lack of agreement
was indicated by the other.


\section{Measurement Error Models}
\citet{DunnSEME} proposes a measurement error model for use in
method comparison studies. Consider n pairs of measurements
$X_{i}$ and $Y_{i}$ for $i=1,2,...n$.
\begin{equation}
X_{i} = \tau_{i}+\delta_{i}\\
\end{equation}
\begin{equation}
 Y_{i} = \alpha +\beta\tau_{i}+\epsilon_{i} \nonumber
\end{equation}

In the above formulation is in the form of a linear structural
relationship, with $\tau_{i}$ and $\beta\tau_{i}$ as the true
values , and $\delta_{i}$ and $\epsilon_{i}$ as the corresponding
measurement errors. In the case where the units of measurement are
the same, then $\beta =1$.

\begin{equation}
E(X_{i}) = \tau_{i}\\
\end{equation}
\begin{equation}
E(Y_{i}) = \alpha +\beta\tau_{i} \nonumber
\end{equation}
\begin{equation}
E(\delta_{i}) = E(\epsilon_{i}) = 0 \nonumber
\end{equation}

The value $\alpha$ is the inter-method bias between the two
methods.

\begin{eqnarray}
  z_0 &=& d = 0 \\
  z_{n+1} &=& z_n^2+c
\end{eqnarray}




\addcontentsline{toc}{section}{Bibliography}

\bibliography{transferbib}
\end{document}
