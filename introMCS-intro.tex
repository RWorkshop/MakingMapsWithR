
% introMCS - very first section

\documentclass[Chap1main.tex]{subfiles}

% Load any packages needed for this document
\begin{document}
\newpage
\section{Introduction}
The problem of assessing the agreement between two or more methods of measurement is ubiquitous in scientific research, and is
commonly referred to as a `method comparison study'. Published examples of method comparison studies can be found in disciplines
as diverse as pharmacology \citep{ludbrook97}, anaesthesia
\citep{Myles}, and cardiac imaging methods \citep{Krumm}.
\smallskip

To illustrate the characteristics of a typical method comparison study consider the data in Table I \citep{Grubbs73}. In each of twelve experimental trials, a single round of ammunition was fired from a 155mm gun and its velocity was measured simultaneously (and independently) by three chronographs devices, identified here by the labels `Fotobalk', `Counter' and `Terma'.
\smallskip


\newpage

\begin{table}[ht]
\begin{center}
\begin{tabular}{rrrr}
  \hline
  Round& Fotobalk [F] & Counter [C]& Terma [T]\\
  \hline
  1 & 793.8 & 794.6 & 793.2 \\
  2 & 793.1 & 793.9 & 793.3 \\
  3 & 792.4 & 793.2 & 792.6 \\
  4 & 794.0 & 794.0 & 793.8 \\
  5 & 791.4 & 792.2 & 791.6 \\
  6 & 792.4 & 793.1 & 791.6 \\
  7 & 791.7 & 792.4 & 791.6 \\
  8 & 792.3 & 792.8 & 792.4 \\
  9 & 789.6 & 790.2 & 788.5 \\
  10 & 794.4 & 795.0 & 794.7 \\
  11 & 790.9 & 791.6 & 791.3 \\
  12 & 793.5 & 793.8 & 793.5 \\
   \hline
\end{tabular}
\caption{Velocity measurement from the three chronographs (Grubbs
1973).}
\end{center}
\end{table}

An important aspect of the these data is that all three methods of
measurement are assumed to have an attended measurement error, and
the velocities reported in Table 1.1 can not be assumed to be
`true values' in any absolute sense.

%While lack of
%agreement between two methods is inevitable, the question , as
%posed by \citet{BA83}, is 'do the two methods of measurement agree
%sufficiently closely?'

A method of measurement should ideally be both accurate and
precise. \citet{Barnhart} describes agreement as being a broader
term that contains both of those qualities. An accurate
measurement method will give results close to the unknown `true
value'. The precision of a method is indicated by how tightly
measurements obtained under identical conditions are distributed
around their mean measurement value. A precise and accurate method
will yield results consistently close to the true value. Of course
a method may be accurate, but not precise, if the average of its
measurements is close to the true value, but those measurements
are highly dispersed. Conversely a method that is not accurate may
be quite precise, as it consistently indicates the same level of
inaccuracy. The tendency of a method of measurement to
consistently give results above or below the true value is a
source of systematic bias. The smaller the systematic bias, the
greater the accuracy of the method.

% The FDA define precision as the closeness of agreement (degree of
% scatter) between a series of measurements obtained from multiple
% sampling of the same homogeneous sample under prescribed
% conditions. \citet{Barnhart} describes precision as being further
% subdivided as either within-run, intra-batch precision or
% repeatability (which assesses precision during a single analytical
% run), or between-run, inter-batch precision or repeatability
%(which measures precision over time).

In the context of the agreement of two methods, there is also a
tendency of one measurement method to consistently give results
above or below the other method. Lack of agreement is a
consequence of the existence of `inter-method bias'. For two
methods to be considered in good agreement, the inter-method bias
should be in the region of zero. A simple estimation of the
inter-method bias can be calculated using the differences of the
paired measurements. The data in Table 1.2 are a good example of
possible inter-method bias; the `Fotobalk' consistently recording
smaller velocities than the `Counter' method. Consequently one
would conclude that there is lack of agreement between the two
methods.

The absence of inter-method bias by itself is not sufficient to
establish whether two measurement methods agree. The two methods
must also have equivalent levels of precision. Should one method
yield results considerably more variable than those of the other,
they can not be considered to be in agreement. With this in mind a
methodology is required that allows an analyst to estimate the
inter-method bias, and to compare the precision of both methods of
measurement.

\end{document}