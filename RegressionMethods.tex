

\documentclass[MASTER.tex]{subfiles}
\begin{document}
\newpage




%\section{Bartko's Bradley-Blackwood Test}
%
%\begin{itemize}
%	\item The Bradley Blackwood test is a simultaneous test for bias and
%	precision. They propose a regression approach which fits D on M,
%	where D is the difference and average of a pair of results.
%	\item Both beta values, the intercept and slope, are derived from the respective means and
%	standard deviations of their respective data sets.
%	\item We determine if the respective means and variances are equal if
%	both beta values are simultaneously equal to zero. The Test is
%	conducted using an F test, calculated from the results of a
%	regression of D on M.
%	\item We have identified this approach  to be examined to see if it can
%	be used as a foundation for a test perform a test on means and
%	variances individually.
%	\item Russell et al have suggested this method be used in conjunction
%	with a paired t-test , with estimates of slope and intercept.
%\end{itemize}
%subsection{t-test}




\newpage
\section{Bradley-Blackwood Test (Kevin Hayes Talk)}
%--------------------------------------------------------------------------------%
% KH - UW

This work considers the problem of testing $\mu_1$ = $\mu_2$ and $\sigma^2_1 = \sigma^2_2$ using a random sample from a 
bivariate normal distribution with parameters $(\mu_1, \mu_2, \sigma^2_1, \sigma^2_2, \rho)$. 

The new contribution is a decomposition of the Bradley-Blackwood test statistic (\textit{Bradley and Blackwood, 1989})for 
the simultaneous test of {$\mu_1$ = $\mu_2$; $\sigma^2_1 = \sigma^2_2$}  as a sum of two statistics. 

One is equivalent to the Pitman-Morgan (\textit{Pitman, 1939; Morgan, 1939}) test statistic 
for $\sigma^2_1 = \sigma^2_2$ and the other one is a new alternative to the standard paired-t test of $\mu_D = \mu_1 = \mu_2 = 0$. 

Surprisingly, the classic Student paired-t test makes no assumptions about the equality (or otherwise) of the 
variance parameters. 

The power functions for these tests are quite easy to derive, and show that when $\sigma^2_1 = \sigma^2_2$, 
the paired t-test has a slight advantage over the new alternative in terms of power, but when $\sigma^2_1 \neq \sigma^2_2$, the 
new test has substantially higher power than the paired-t test.

While Bradley and Blackwood provide a test on the joint hypothesis of equal means and equal variances their regression 
based approach does not separate these two issues.

The rejection of the joint hypothesis may be 
due to two groups with unequal means and unequal variances; unequal means and equal variances, or equal means and unequal variances. 

We propose an approach for resolving this (model selection) problem in a manner controlling the magnitudes of the 
relevant type I error probabilities.


%This application of the
%Grubbs method presumes the existence of this condition, and necessitates
%replication of observations by means external to and independent of the first
%means. The Grubbs estimators method is based on the laws of propagation of
%error. By making three independent simultaneous measurements on the same
%physical material, it is possible by appropriate mathematical manipulation of
%the sums and differences of the associated variances to obtain a valid
%estimate of the precision of the primary means. Application of the Grubbs
%estimators procedure to estimation of the precision of an apparatus uses
%the results of a physical test conducted in such a way as to obtain a series
%of sets of three independent observations.



\section{Conclusions about Existing Methodologies}

The Bland Altman methodology is well noted for its ease of use,
and can be easily implemented with most software packages. Also it
doesn't require the practitioner to have more than basic
statistical training. The plot is quite informative about the
variability of the differences over the range of measurements. For
example, an inspection of the plot will indicate the 'fan effect'.
They also can be used to detect the presence of an outlier.

 \citet{ludbrook97,ludbrook02} criticizes these plots on the
basis that they presents no information on effect of constant bias
or proportional bias. These plots are only practicable when both
methods measure in the same units. Hence they are totally
unsuitable for conversion problems. The limits of agreement are
somewhat arbitrarily constructed. They may or may not be suitable
for the data in question. It has been found that the limits given
are too wide to be acceptable. There is no guidance on how to deal
with outliers. Bland and Altman recognize effect they would have
on the limits of agreeement, but offer no guidance on how to
correct for those effects.

There is no formal testing procedure provided. Rather, it is upon
the practitioner opinion to judge the outcome of the methodology.





%%%%%%%%%%%%%%%%%%%%%%%%%%%%%%%%%%%%%%%%%%%%%%%%%%%%%%%%%%%%%%%%%%%%%%%%%
%9 Appendix                  %%%%%%%%%%%%%%%%%%%%%%%%%%%%%%%%%%%%%%%%%%%%%
%%%%%%%%%%%%%%%%%%%%%%%%%%%%%%%%%%%%%%%%%%%%%%%%%%%%%%%%%%%%%%%%%%%%%%%%%


%
%\subsection{Contention }
%Several papers have commented that this approach is undermined
%when the basic assumptions underlying linear regression are not
%met, the regression equation, and consequently the estimations of
%bias are undermined. Outliers are a source of error in regression
%estimates.In method comparison studies, the X variable is a
%precisely measured reference method. Cornbleet Gochman (1979)
%argued that criterion may be regarded as the correct value. Other
%papers dispute this.
%



%
%
%\subsection{A regression based approach based on Bland Altman Analysis}
%Lu et al used such a technique in their comparison of DXA
%scanners. They also used the Blackwood Bradley test. However it
%was shown that, for particular comparisons,  agreement between
%methods was indicated according to one test, but lack of agreement
%was indicated by the other.

%============================================================================================ %
\section{Regression Methods}
	Conventional regression models are estimated using the ordinary
	least squares (OLS) technique, and are referred to as `Model I
	regression' \citep{CornCoch,ludbrook97}. A key feature of Model I
	models is that the independent variable is assumed to be measured
	without error. As often pointed out in several papers
	\citep{BA83,ludbrook97}, this assumption invalidates simple linear
	regression for use in method comparison studies, as both methods
	must be assumed to be measured with error.
	
	The use of regression models that assumes the presence of error in
	both variables $X$ and $Y$ have been proposed for use instead
	\citep{CornCoch,ludbrook97}. These methodologies are collectively
	known as `Model II regression'. They differ in the method used to
	estimate the parameters of the regression.
	
	Regression estimates depend on formulation of the model. A
	formulation with one method considered as the $X$ variable will
	yield different estimates for a formulation where it is the $Y$
	variable. With Model I regression, the models fitted in both cases
	will entirely different and inconsistent. However with Model II
	regression, they will be consistent and complementary.
	
	Regression approaches are useful for a making a detailed examination of the biases across the range of measurements, allowing bias to be decomposed into fixed bias and proportional bias.
	Fixed bias describes the case where one method gives values that are consistently different
	to the other across the whole range. Proportional
	bias describes the difference in measurements getting progressively greater, or smaller, across the range of measurements. A measurement method may have either an attendant fixed bias or proportional bias, or both. \citep{ludbrook}. Determination of these biases shall be discussed in due course.
	
	
	
	\bibliography{DB-txfrbib}
\end{document}