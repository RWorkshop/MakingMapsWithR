\documentclass[12pt, a4paper]{article}
\usepackage{epsfig}
\usepackage{subfigure}
%\usepackage{amscd}
\usepackage{amssymb}
\usepackage{amsbsy}
\usepackage{amsthm}
%\usepackage[dvips]{graphicx}
\usepackage{natbib}
\bibliographystyle{chicago}
\usepackage{vmargin}
% left top textwidth textheight headheight
% headsep footheight footskip
\setmargins{3.0cm}{2.5cm}{15.5 cm}{22cm}{0.5cm}{0cm}{1cm}{1cm}
\renewcommand{\baselinestretch}{1.5}
\pagenumbering{arabic}
\theoremstyle{plain}
\newtheorem{theorem}{Theorem}[section]
\newtheorem{corollary}[theorem]{Corollary}
\newtheorem{ill}[theorem]{Example}
\newtheorem{lemma}[theorem]{Lemma}
\newtheorem{proposition}[theorem]{Proposition}
\newtheorem{conjecture}[theorem]{Conjecture}
\newtheorem{axiom}{Axiom}
\theoremstyle{definition}
\newtheorem{definition}{Definition}[section]
\newtheorem{notation}{Notation}
\theoremstyle{remark}
\newtheorem{remark}{Remark}[section]
\newtheorem{example}{Example}[section]
\renewcommand{\thenotation}{}
\renewcommand{\thetable}{\thesection.\arabic{table}}
\renewcommand{\thefigure}{\thesection.\arabic{figure}}
\title{Roy's Methodology}
\author{ } \date{ }

\begin{document}
%----------------------------------------------------------------------------------------%
\tableofcontents
\newpage

%\section{Modelling Agreement with LME Models}


\section{Introduction}

\citet{roy} uses an approach based on linear mixed effects (LME) models for the purpose of comparing the agreement between two methods of measurement, where replicate measurements on items, typically individuals, by both methods are available. She provides three tests of hypothesis appropriate for evaluating the agreement between the two methods of measurement under this sampling scheme. These tests consider null hypotheses that assume: absence of inter-method bias; equality of between-subject variabilities of the two methods; equality of within-subject variabilities of the two methods. By inter-method bias we mean that a systematic difference exists between observations recorded by the two methods. Differences in between-subject variabilities of the two methods arise when one method is yielding average response levels for individuals than are more variable than the average response levels for the same sample of individuals taken by the other method.  Differences in within-subject variabilities of the two methods arise when one method is yielding responses for an individual than are more variable than the responses for this same individual taken by the other method. The two methods of measurement can be considered to agree, and subsequently can be used interchangeably, if all three null hypotheses are true.

\subsection{Model Specification}
Let $y_{mir} $ be the $r$th replicate measurement on the $i$th item by the $m$th method, where $m=1,2,$ $i=1,\ldots,N,$ and $r = 1,\ldots,n_i.$ When the design is balanced and there is no ambiguity we can set $n_i=n.$ The LME model can be written
\begin{equation}
y_{mir} = \beta_{0} + \beta_{m} + b_{mi} + \epsilon_{mir}.
\end{equation}
Here $\beta_0$ and $\beta_m$ are fixed-effect terms representing, respectively, a model intercept and an overall effect for method $m.$ The $b_{1i}$ and $b_{2i}$ terms represent random effect parameters corresponding to the two methods, having $\mathrm{E}(b_{mi})=0$ with $\mathrm{Var}(b_{mi})=g^2_m$ and $\mathrm{Cov}(b_{mi}, b_{m^\prime i})=g_{12}.$ The random error term for each response is denoted $\epsilon_{mir}$ having $\mathrm{E}(\epsilon_{mir})=0$, $\mathrm{Var}(\epsilon_{mir})=\sigma^2_m$, $\mathrm{Cov}(b_{mir}, b_{m^\prime ir})=\sigma_{12}$, $\mathrm{Cov}(\epsilon_{mir}, \epsilon_{mir^\prime})= 0$ and $\mathrm{Cov}(\epsilon_{mir}, \epsilon_{m^\prime ir^\prime})= 0.$
When two methods of measurement are in agreement, there is no significant differences between $\beta_1$ and $\beta_2,$ $g^2_1 $ and$ g^2_2$, and $\sigma^2_1 $ and$ \sigma^2_2$.
\bigskip

% Complete paragraph by specifying variances and covariances for epsilons.
% I thing that these are your sigmas?
% Also, state equality of the parameters in this model when each of the three hypotheses above are true.


\newpage

\section{Roy's Hypotheses Tests}

In order to express Roy's LME model in matrix notation we gather all $2n_i$ observations specific to item $i$ into a single vector  $\boldsymbol{y}_{i} = (y_{1i1},y_{2i1},y_{1i2},\ldots,y_{mir},\ldots,y_{1in_{i}},y_{2in_{i}})^\prime.$ The LME model can be written
\[
\boldsymbol{y_{i}} = \boldsymbol{X_{i}\beta} + \boldsymbol{Z_{i}b_{i}} + \boldsymbol{\epsilon_{i}},
\]
where $\boldsymbol{\beta}=(\beta_0,\beta_1,\beta_2)^\prime$ is a vector of fixed effects, and $\boldsymbol{X}_i$ is a corresponding $2n_i\times 3$ design matrix for the fixed effects. The random effects are expressed in the vector $\boldsymbol{b}=(b_1,b_2)^\prime$, with $\boldsymbol{Z}_i$ the corresponding $2n_i\times 2$ design matrix. The vector $\boldsymbol{\epsilon}_i$ is a $2n_i\times 1$ vector of residual terms.

It is assumed that $\boldsymbol{b}_i \sim N(0,\boldsymbol{G})$, $\boldsymbol{\epsilon}_i$ is a matrix of random errors distributed as $N(0,\boldsymbol{R}_i)$ and that the random effects and residuals are independent of each other.
% Assumptions made on the structures of $\boldsymbol{G}$ and $\boldsymbol{R}_i$ will be discussed in due course.

% \texttt{finish}

$\boldsymbol{G}$ is the variance covariance matrix for the random effects $\boldsymbol{b}$.
i.e. between-item sources of variation. The between-item variance covariance matrix $\boldsymbol{G}$ is constructed as follows:

\[ \mbox{Var}  \left[
\begin{array}{c}
b_1   \\
b_2  \\
\end{array}
\right] =  \boldsymbol{G} =\left(
\begin{array}{cc}
g^2_1  & g_{12} \\
g_{12} & g^2_2 \\
\end{array}
\right) \]
It is important to note that no special assumptions about the structure of $\boldsymbol{G}$ are made. An example of such an assumption would be that $\boldsymbol{G}$ is the product of a scalar value and the identity matrix.

$\boldsymbol{R}_{i}$ is the variance covariance matrix for the residuals, i.e. the within-item sources of variation between both methods. Computational analysis of linear mixed effects models allow for the explicit analysis of both $\boldsymbol{G}$ and $\boldsymbol{R_i}$.

\citet{hamlett} shows that $\boldsymbol{R}_{i}$  can be expressed as $\boldsymbol{R}_{i} = \boldsymbol{I}_{n_{i}} \otimes \boldsymbol{\Sigma}$. The partial within-item variance?covariance matrix of two methods at any replicate is denoted $\boldsymbol{\Sigma}$, where $\sigma^2_{1}$ and $\sigma^2_{2}$ are the within-subject variances of the respective methods, and $\sigma_{12}$ is the within-item covariance between the two methods. It is assumed that the within-item variance?covariance matrix $\boldsymbol{\Sigma}$ is the same for all replications. Again it is important to note that no special assumptions are made about the structure of the matrix.

\begin{equation}
\boldsymbol{\Sigma} = \left( \begin{array}{cc}
\sigma^2_{1} & \sigma_{12} \\
\sigma_{12} & \sigma^2_{2} \\
\end{array}\right)
\end{equation}

\vspace{1in}

For expository purposes consider the case where each item provides three replicates by each method. Then in matrix notation the model has the structure
\begin{equation}
\boldsymbol{y}_{i} =
\left(
\begin{array}{ccc}
1 & 1 & 0 \\
1 & 0 & 1 \\
1 & 1 & 0 \\
1 & 0 & 1 \\
1 & 1 & 0 \\
1 & 0 & 1 \\
\end{array}
\right)
\left(
\begin{array}{c}         \beta_0 \\ \beta_1 \\ \beta_2 \\
\end{array}
\right)
+  \left(
\begin{array}{cc}
1 & 0 \\
0 & 1 \\
1 & 0 \\
0 & 1 \\
1 & 0 \\
0 & 1 \\
\end{array}
\right)\left(
\begin{array}{c}
b_{1i} \\   b_{2i} \\
\end{array}
\right)
+
\left(
\begin{array}{c}
\epsilon_{1i1} \\
\epsilon_{2i1} \\
\epsilon_{1i2} \\
\epsilon_{2i2} \\
\epsilon_{1i3} \\
\epsilon_{2i3} \\
\end{array}
\right) ,
\end{equation}
where
\[
\boldsymbol{G} =
\]
and
\[
\boldsymbol{R}_i =
\]

It is assumed that $\boldsymbol{b}_i \sim N(0,\boldsymbol{G})$,
$\boldsymbol{\epsilon}_i$ is a matrix of random errors distributed as $N(0,\boldsymbol{R}_i)$ and
that the random effects and residuals are independent of each other. Assumptions made on the structures of $\boldsymbol{G}$ and $\boldsymbol{R}_i$ will be discussed in due course.

\newpage





The overall variability between
the two methods is the sum of between-item variability
$\boldsymbol{G}$ and within-item variability
$\boldsymbol{\Sigma}$. \citet{roy} denotes the overall variability
as ${\mbox{Block - }\boldsymbol \Omega_{i}}$. The overall
variation for methods $1$ and $2$ are given by

\begin{center}
	\[\left(\begin{array}{cc}
	\omega^2_1  & \omega_{12} \\
	\omega_{12} & \omega^2_2 \\
	\end{array}  \right)
	=  \left(
	\begin{array}{cc}
	g^2_1  & g_{12} \\
	g_{12} & g^2_2 \\
	\end{array} \right)+
	\left(
	\begin{array}{cc}
	\sigma^2_1  & \sigma_{12} \\
	\sigma_{12} & \sigma^2_2 \\
	\end{array}\right)
	\]
\end{center}
The computation of the limits of agreement require that the variance of the difference of measurements. This variance is easily computable from the estimate of the ${\mbox{Block - }\boldsymbol \Omega_{i}}$ matrix. Lack of agreement can arise if there is a disagreement in overall variabilities. This may be due to due to the disagreement in either between-item
variabilities or within-item variabilities, or both. \citet{roy} allows for a formal test of each.

\subsection{Hypothesis Testing}
The formulation presented above usefully facilitates a series of
significance tests that advise as to how well the two methods
agree. These tests are as follows:
\begin{itemize}
	\item A formal test for the equality of between-item variances,
	\item A formal test for the equality of within-item variances,
	\item A formal test for the equality of overall variances.
\end{itemize}
These tests are complemented by the ability to consider the inter-method bias and the overall correlation coefficient. Two methods can be considered to be in agreement if criteria based upon these methodologies are met. Additionally Roy makes reference to the overall correlation coefficient of the two methods, which is determinable from variance estimates.

\newpage
\section{Carstensen's Limits of agreement}
\citet{bxc2008} presents a methodology to compute the limits of
agreement based on LME models. Importantly, Carstensen's underlying model differs from Roy's model in some key respects, and therefore a prior discussion of Carstensen's model is required.



\subsection{Assumptions on Variability}

Aside from the fixed effects, another important difference is that Carstensen's model requires that particular assumptions be applied, specifically that the off-diagonal elements of the between-item
and within-item variability matrices are zero. By extension the
overall variability off diagonal elements are also zero.

Also, implementation requires that the between-item variances are
estimated as the same value: $g^2_1 = g^2_2 = g^2$. Necessarily
Carstensen's method does not allow for a formal test of the
between-item variability.

\[\left(\begin{array}{cc}
\omega^1_2  & 0 \\
0 & \omega^2_2 \\
\end{array}  \right)
=  \left(
\begin{array}{cc}
g^2  & 0 \\
0 & g^2 \\
\end{array} \right)+
\left(
\begin{array}{cc}
\sigma^2_1  & 0 \\
0 & \sigma^2_2 \\
\end{array}\right)
\]

In cases where the off-diagonal terms in the overall variability
matrix are close to zero, the limits of agreement due to
\citet{bxc2008} are very similar to the limits of agreement that
follow from the general model.

\newpage


\section{Note 1: Coefficient of Repeatability}
The coefficient of repeatability is a measure of how well a
measurement method agrees with itself over replicate measurements
\citep{BA99}. Once the within-item variability is known, the
computation of the coefficients of repeatability for both methods
is straightforward.


\section{Note 2: Model terms}
It is important to note the following characteristics of this model.
\begin{itemize}
	\item Let the number of replicate measurements on each item $i$ for both methods be $n_i$, hence $2 \times n_i$ responses. However, it is assumed that there may be a different number of replicates made for different items. Let the maximum number of replicates be $p$. An item will have up to $2p$ measurements, i.e. $\max(n_{i}) = 2p$.
	
	% \item $\boldsymbol{y}_i$ is the $2n_i \times 1$ response vector for measurements on the $i-$th item.
	% \item $\boldsymbol{X}_i$ is the $2n_i \times  3$ model matrix for the fixed effects for observations on item $i$.
	% \item $\boldsymbol{\beta}$ is the $3 \times  1$ vector of fixed-effect coefficients, one for the true value for item $i$, and one effect each for both methods.
	
	\item Later on $\boldsymbol{X}_i$ will be reduced to a $2 \times 1$ matrix, to allow estimation of terms. This is due to a shortage of rank. The fixed effects vector can be modified accordingly.
	\item $\boldsymbol{Z}_i$ is the $2n_i \times  2$ model matrix for the random effects for measurement methods on item $i$.
	\item $\boldsymbol{b}_i$ is the $2 \times  1$ vector of random-effect coefficients on item $i$, one for each method.
	\item $\boldsymbol{\epsilon}$  is the $2n_i \times  1$ vector of residuals for measurements on item $i$.
	\item $\boldsymbol{G}$ is the $2 \times  2$ covariance matrix for the random effects.
	\item $\boldsymbol{R}_i$ is the $2n_i \times  2n_i$ covariance matrix for the residuals on item $i$.
	\item The expected value is given as $\mbox{E}(\boldsymbol{y}_i) = \boldsymbol{X}_i\boldsymbol{\beta}.$ \citep{hamlett}
	\item The variance of the response vector is given by $\mbox{Var}(\boldsymbol{y}_i)  = \boldsymbol{Z}_i \boldsymbol{G} \boldsymbol{Z}_i^{\prime} + \boldsymbol{R}_i$ \citep{hamlett}.
\end{itemize}

\subsection{Roy's methodology}

For the purposes of comparing two methods of measurement, \citet{roy} presents a methodology utilizing linear mixed effects model. This methodology provides for the formal testing of inter-method bias, between-subject variability and within-subject variability of two methods. The formulation contains a Kronecker product covariance structure in a doubly multivariate setup. By doubly multivariate set up, Roy means that the information on each patient or item is multivariate at two levels, the number of methods and number of replicated measurements. Further to \citet{lam}, it is assumed that the replicates are linked over time. However it is easy to modify to the unlinked case.

\citet{roy} sets out three criteria for two methods to be considered in agreement. Firstly that there be no significant bias. Second that there is no difference in the between-subject variabilities, and lastly that there is no significant difference in the within-subject variabilities. Roy further proposes examination of the the overall variability by considering the second and third criteria be examined jointly. Should both the second and third criteria be fulfilled, then the overall variabilities of both methods would be equal.

A formal test for inter-method bias can be implemented by examining the fixed effects of the model. This is common to well known classical linear model methodologies. The null hypotheses, that both methods have the same mean, which is tested against the alternative hypothesis, that both methods have different means.
The inter-method bias and necessary $t-$value and $p-$value are presented in computer output. A decision on whether the first of Roy's criteria is fulfilled can be based on these values.

Importantly \citet{roy} further proposes a series of three tests on the variance components of an LME model, which allow decisions on the second and third of Roy's criteria. For these tests, four candidate LME models are constructed. The differences in the models are specifically in how the the $D$ and $\Lambda$ matrices are constructed, using either an unstructured form or a compound symmetry form. To illustrate these differences, consider a generic matrix $A$,

\[
\boldsymbol{A} = \left( \begin{array}{cc}
    a_{11} & a_{12}  \\
    a_{21} & a_{22}  \\
    \end{array}\right).
\]

A symmetric matrix allows the diagonal terms $a_{11}$ and $a_{22}$ to differ. The compound symmetry structure requires that both of these terms be equal, i.e $a_{11} = a_{22}$.

The first model acts as an alternative hypothesis to be compared against each of three other models, acting as null hypothesis models, successively. The models are compared using the likelihood ratio test. Likelihood ratio tests are a class of tests based on the comparison of the values of the likelihood functions of two candidate models. LRTs can be used to test hypotheses about covariance parameters or fixed effects parameters in the context of LMEs. The test statistic for the likelihood ratio test is the difference of the log-likelihood functions, multiplied by $-2$.
The probability distribution of the test statistic is approximated by the $\chi^2$ distribution with ($\nu_{1} - \nu_{2}$) degrees of freedom, where $\nu_{1}$ and $\nu_{2}$ are the degrees of freedom of models 1 and 2 respectively. Each of these three test shall be examined in more detail shortly.
%---------------------------------------------------------------------------------------------------%
\section{Roy's LME approach}

\newpage



The methodology uses a linear mixed effects regression fit using
compound symmetry (CS) correlation structure on \textbf{V}.


$\Lambda = \frac{\mbox{max}_{H_{0}}L}{\mbox{max}_{H_{1}}L}$

\newpage

\citet{ARoy2009} considers the problem of assessing the agreement
between two methods with replicate observations in a doubly
multivariate set-up using linear mixed effects models.


\citet{ARoy2009} uses examples from \citet{BA86} to be able to
compare both types of analysis.



For the the RV-IC comparison, $\hat{D}$ is given by


\begin{equation}
\hat{D}= \left[ \begin{array}{cc}
  1.6323 & 1.1427  \\
  1.1427 & 1.4498 \\
\end{array} \right]
\end{equation}

The estimate for the within-subject variance covariance matrix is
given by
\begin{equation}
\hat{\Sigma}= \left[ \begin{array}{cc}
  0.1072 & 0.0372  \\
  0.0372 & 0.1379  \\
\end{array}\right]
\end{equation}
The estimated overall variance covariance matrix for the the 'RV
vs IC' comparison is given by
\begin{equation}
Block \Omega_{i}= \left[ \begin{array}{cc}
  1.7396 & 1.1799  \\
  1.1799 & 1.5877  \\
\end{array} \right].
\end{equation}

 The power of the
likelihood ratio test may depends on specific sample size and the
specific number of  replications, and the author proposes
simulation studies to examine this further.

\section{Roy's LME methodology for assessing agreement}

\citet{Barnhart}  describes the sources of disagreement as
differing population means, different between-subject variances,
different within-subject variances between two methods and poor
correlation between measurements of two methods.


\citet{ARoy2009}proposes the use of LME models to perform a test
on two methods of agreement to determine whether they can be used
interchangeably.

Bivariate correlation coefficients have been shown to be of
limited use in method comparison studies \citep{BA86}. However,
recently correlation analysis has been developed to cope with
repeated measurements, enhancing their potential usefulness. Roy
incorporates the use of correlation into his methodology.


\citet{ARoy2009} considers the problem of assessing the agreement
between two methods with replicate observations in a doubly
multivariate set-up using linear mixed effects models.


\citet{ARoy2009} uses examples from \citet{BA86} to be able to
compare both types of analysis.

\citet{ARoy2009} proposes a LME based approach with Kronecker
product covariance structure with doubly multivariate setup to
assess the agreement between two methods. This method is designed
such that the data may be unbalanced and with unequal numbers of
replications for each subject.

\citet{ARoy2009} considers four independent hypothesis tests.
\begin{itemize}
\item Testing of hypotheses of differences between the means of
two methods\item Testing of hypotheses in between subject
variabilities in two methods, \item Testing of hypotheses of
differences in within-subject variability of the two methods,
\item Testing of hypotheses in differences in overall variability
of the two methods.
\end{itemize}


\section{Replicates}
Measurements taken in quick succession by the same observer using the same instrument on the same subject can be considered true replicates. \citet{Roy2009} notes that some measurements may not be `true' replicates.

Roy's methodology assumes the use of `true replicates'. However data may not be collected in this way. In such cases, the correlation matrix on the replicates may require a different structure, such as the autoregressive order one $AR(1)$ structure. However determining MLEs with such a structure would be computational intense, if possible at all.

\section{Roy's methodology with single measurements}


\section{Roy's examples}
Roy provides three case studies, using data sets well known in method comparison studies, to demonstrate how the methodology should be used.


%--------------------------------------------------------------------------Example 1a  ----  JSR Data %

The first case study is the Systolic blood pressure data, taken from \citet{BA99}.


%--------------------------------------------------------------------------Example 1b  ----  JSR Data %

To complete the study, the relevant values are provided for the $R \mbox{vs} S$ comparison also.

%--------------------------------------------------------------------------Example 2  ----  PEFR Data %

The second data set, a comparison of two peak expiratory flow rate measurements, is referenced by \citet{BA86}.


%--------------------------------------------------------------------------Example 3 Cardiac Ejection Fraction Data %
The last case study is also based on a data set from  \citet{BA99}. It contains the measurements of left ventricular cardiac eject fraction, measured by impedance cartography and radionuclide ventriculography, on twelve patients.
The number of replicated differs for each patient.

The bias is shown to be $0.7040$, with a p-value of $0.0204$. The MLEa of the between-method and within-method variance-covariance matrices of methods $RV$ and $IC$ are given by

\begin{equation}\hat{D}=\left(
                                      \begin{array}{cc}
                                        1.6323 & 1.1427 \\
                                        1.1427 & 1.4498 \\
                                      \end{array}
                                    \right),
\end{equation}



\begin{equation}\hat{\Sigma}=\left(
                                      \begin{array}{cc}
                                        1.6323 & 1.1427 \\
                                        1.1427 & 1.4498 \\
                                      \end{array}
                                    \right).
\end{equation}

\citet{roy} notes that these are the same estimate for variance as given by \citet{BA99}.


The repeatability coefficients are determined to be $0.9080$ for the RV method and $1.0293$ for the IC method.

From the estimated $\boldsymbol{\Omega_{i}}$ correlation matrix, the overall correlation coefficient is $0.7100$.
The overall correllation coefficients between two methods RV and IC are $0.9384$ and $0.9131$ respectively.

\citet{roy} concludes that is appropriate to switch between the two methods if needed.

%--------------------------------------------------------------------------Example 4 Coronary Artery Calcium data%

\citet{haber}

\citet{roy} recommends to not switch between the two method.

\section{LME}

Fitting model according to Roy

\newpage
\begin{verbatim}
Linear mixed-effects model fit by REML
 Data: BA99
       AIC      BIC    logLik
  4319.707 4336.629 -2155.853

Random effects:
 Formula: ~1 | subj
        (Intercept) Residual
StdDev:    29.39085 12.44454

Fixed effects: ob.js ~ method
                Value Std.Error  DF  t-value p-value
(Intercept) 127.40784  3.281757 424 38.82306       0

methodS     15.61961   1.102107 424 14.17250       0

 Correlation:
        (Intr)
methodS -0.168

Standardized Within-Group Residuals:
        Min          Q1         Med          Q3         Max
-3.61292639 -0.42538402 -0.02467651  0.40166235  4.84280044

Number of Observations: 510 Number of Groups: 85

\end{verbatim}

%---------------------------------------------------------------------------------------------------%
\newpage


%----------------------------------------------------------------------------%



\newpage
\subsection{Difference Variance further to Carstensen}

\citet{bxc2008} states a model where the variation between items
for method $m$ is captured by $\tau_m$ (our notation $d^2_m$) and the within-item
variation by $\sigma_m$.

\emph{The formulation of this model is general and refers to comparison
	of any number of methods — however, if only two methods are
	compared, separate values of $\tau^2_1$ and $\tau^2_2$ cannot be
	estimated, only their average value $\tau$, so in the case of only
	two methods we are forced to assume that $\tau_1 = \tau_2 = \tau$}\citep{bxc2008}.

Another important point is that there is no covariance terms, so
further to  \citet{bxc2008} the variance covariance matrices for
between-item and within-item variability are respectively.

\[\boldsymbol{D} = \left(
\begin{array}{cc}
d^1_2  & 0 \\
0 & d^2_2 \\
\end{array}
\right) \]
and  $\boldsymbol{\Sigma}$ is constructed as follows:
\[\boldsymbol{\Sigma} = \left(
\begin{array}{cc}
\sigma^1_2  & 0 \\
0 & \sigma^2_2 \\
\end{array}
\right) \]


Under this model the limits of agreement should be computed based
on the standard deviation of the difference between a pair of
measurements by the two methods on a new individual, j, say:

\[ \mbox{var}(y_{1j} - y_{2j}) = 2d^2 + \sigma^2_1 + \sigma^2_2  \]

Further to his model, Carstensen computes the limits of agreement
as

\[
\hat{\alpha}_1 - \hat{\alpha}_2 \pm \sqrt{2 \hat{d}^2 + 	\hat{\sigma}^2_1 + \hat{\sigma}^2_2}
\]


\subsection{Relevance of Roy's Methodology}

The relevance of Roy's methodology is that estimates for the between-item variances for both methods $\hat{d}^2_m$ are computed. Also the VC matrices are constructed with covariance
terms and, so the difference variance must be formulated accordingly.


\[
\hat{\alpha}_1 - \hat{\alpha}_2 \pm \sqrt{ \hat{d}^2_1  +
	\hat{d}^2_1 + \hat{\sigma}^2_1 + \hat{\sigma}^2_2 - 2 \hat{d}_{12}
	- 2 \hat{\sigma}_12}
\]

\newpage



\section{LME models in method comparison studies}
%With the greater computing power available for scientific
%analysis, it is inevitable that complex models such as linear
%mixed effects models should be applied to method comparison
%studies.

Linear mixed effects (LME) models can facilitate greater understanding of the potential causes of bias and differences in
precision between two sets of measurement. \citet{LaiShiao} views the uses of linear mixed effects models as an expansion on the
Bland-Altman methodology, rather than as a replacement.
\citet{BXC2008} remarks that modern statistical computation, such as that used for LME models, greatly improve the efficiency of
calculation compared to previous `by-hand' methods. In this chapter various LME approaches to method comparison studies shall
be examined.

\newpage

\section{Roy's LME methodology for assessing agreement}

\citet{ARoy2009} proposes the use of LME models to perform a test on two methods of agreement to determine whether they can be used
interchangeably.

Bivariate correlation coefficients have been shown to be of limited use in method comparison studies \citep{BA86}. However,
recently correlation analysis has been developed to cope with
repeated measurements, enhancing their potential usefulness. Roy
incorporates the use of correlation into his methodology.

Roy's method considers two methods to be in agreement if three
conditions are met.

\begin{itemize}
	\item no significant bias, i.e. the difference between the two
	mean readings is not "statistically significant",
	
	\item high overall correlation coefficient,
	
	\item the agreement between the two methods by testing their
	repeatability coefficients.
	
\end{itemize}

The methodology uses a linear mixed effects regression fit using
compound symmetry (CS) correlation structure on \textbf{V}.


$\Lambda = \frac{\mbox{max}_{H_{0}}L}{\mbox{max}_{H_{1}}L}$

\newpage

\citet{ARoy2009} considers the problem of assessing the agreement
between two methods with replicate observations in a doubly
multivariate set-up using linear mixed effects models.

\citet{ARoy2009} uses examples from \citet{BA86} to be able to
compare both types of analysis.

\citet{ARoy2009} proposes a LME based approach with Kronecker
product covariance structure with doubly multivariate setup to
assess the agreement between two methods. This method is designed
such that the data may be unbalanced and with unequal numbers of
replications for each subject.

The maximum likelihood estimate of the between-subject variance
covariance matrix of two methods is given as $D$. The estimate for
the within-subject variance covariance matrix is $\hat{\Sigma}$.
The estimated overall variance covariance matrix `Block
$\Omega_{i}$' is the addition of $\hat{D}$ and $\hat{\Sigma}$.


\begin{equation}
	\mbox{Block  }\Omega_{i} = \hat{D} + \hat{\Sigma}
\end{equation}

For the the RV-IC comparison, $\hat{D}$ is given by


\begin{equation}
	\hat{D}= \left[ \begin{array}{cc}
		1.6323 & 1.1427  \\
		1.1427 & 1.4498 \\
	\end{array} \right]
\end{equation}

The estimate for the within-subject variance covariance matrix is
given by
\begin{equation}
	\hat{\Sigma}= \left[ \begin{array}{cc}
		0.1072 & 0.0372  \\
		0.0372 & 0.1379  \\
	\end{array}\right]
\end{equation}
The estimated overall variance covariance matrix for the the 'RV
vs IC' comparison is given by
\begin{equation}
	Block \Omega_{i}= \left[ \begin{array}{cc}
		1.7396 & 1.1799  \\
		1.1799 & 1.5877  \\
	\end{array} \right].
\end{equation}

The power of the
likelihood ratio test may depends on specific sample size and the
specific number of  replications, and the author proposes
simulation studies to examine this further.


% \begin{equation}
% data here
% \end{equation}

%-----------------------------------------------------------------------------------------------------%
\newpage





%-----------------------------------------------------------------------------------------------------%
\newpage
\section{Hamlett and Lam}
The methodology proposed by \citet{Roy2009} is largely based on \citet{hamlett}, which in turn follows on from \citet{lam}.

%Lam 99
%In many cases, repeated observation are collected from each subject in sequence  and/or longitudinally.

%Hamlett
%Hamlett re-analyses the data of lam et al to generalize their model to cover other settings not covered by the Lam %method.


%-----------------------------------------------------------------------------------------------------%
\newpage
\subsection{Roy's variability tests}
Variability tests proposed by \citet{Roy2009} affords the opportunity to expand upon Carstensen's approach.

The first test allows of the comparison the begin-subject variability of two methods. Similarly, the second test
assesses the within-subject variability of two methods. A third test is a test that compares the overall variability of the two methods.

The tests are implemented by fitting a specific LME model, and three variations thereof, to the data. These three variant models introduce equality constraints that act null hypothesis cases.

Other important aspects of the method comparison study are consequent. The limits of agreement are computed using the results of the first model.



\bibliography{DB-txfrbib}
\end{document}

