\documentclass[Chap4cmain.tex]{subfiles}
\begin{document}

\begin{verbatim}
Linear mixed-effects model fit by REML
  Data: dat
  Log-restricted-likelihood: -2047.714
  Fixed: BP ~ method
(Intercept)     methodS
  127.40784    15.61961

Random effects:
 Formula: ~1 | subject
        (Intercept)
StdDev:    28.28452

 Formula: ~1 | method %in% subject
        (Intercept) Residual
StdDev:    12.61562 7.763666

Number of Observations: 510
Number of Groups:
            subject method %in% subject
                 85                 170
\end{verbatim}

subsection{Model Fit 2}


This is a simple model, this time with an interaction effect.
There is a fixed effect for each method. This model has random effects at two levels $b_{i}$ for the subject, and
another, $b_{ij}$, for the respective method within each subject.
\begin{equation*}
y_{ijk} = \beta_{j}  + b_{i} + b_{ij} + \epsilon_{ijk}, \qquad i=1,\dots,2, j=1,\dots,85, k=1,\dots,3
\end{equation*}
\begin{eqnarray*}
b_{i} \sim \mathcal{N}(0,\sigma^2_{1}), \qquad b_{ij} \sim \mathcal{N}(0,\sigma^2_{2}), \qquad \epsilon_{i} \sim \mathcal{N}(0,\sigma^2)
\end{eqnarray*}

In this model, the random interaction terms all have the same variance $\sigma^2_{2}$. These terms are assumed to be independent of each other, even
within the same subject.


\newpage
\subsection{Model Fit 3}

This model is a more general model, compared to 'model fit 2'. This model treats the random interactions for each subject as a vector and
allows the variance-covariance matrix for that vector to be estimated from the set of all positive-definite matrices.
$\boldsymbol{y_{i}}$ is the entire response vector for the $i$th subject.
$\boldsymbol{X_{i}}$ and $\boldsymbol{Z_{i}}$  are the fixed- and random-effects design matrices respectively.
\begin{equation*}
\boldsymbol{y_{i}} = \boldsymbol{X_{i}\beta}  + \boldsymbol{Z_{i}b_{i}} + \boldsymbol{\epsilon_{i}}, \qquad i=1,\dots,85
\end{equation*}
\begin{eqnarray*}
\boldsymbol{Z_{i}} \sim \mathcal{N}(\boldsymbol{0,\Psi}),\qquad
\boldsymbol{\epsilon_{i}} \sim \mathcal{N}(\boldsymbol{0,\sigma^2\Lambda})
\end{eqnarray*}

For the first subject the response vector, $\boldsymbol{y_{1}}$, is:
\begin{table}[ht]
\begin{center}
\begin{tabular}{rrllr}
  \hline
observation & BP & subject & method & replicate \\
  \hline
1 & 100.00 & 1 & J &   1 \\
  86 & 106.00 & 1 & J &   2 \\
  171 & 107.00 & 1 & J &   3 \\
  511 & 122.00 & 1 & S &   1 \\
  596 & 128.00 & 1 & S &   2 \\
  681 & 124.00 & 1 & S &   3 \\
   \hline
\end{tabular}
\end{center}
\end{table}
\newpage
The fixed effects design matrix $\boldsymbol{X_{i}}$ is given by:
\begin{table}[ht]
\begin{center}
\begin{tabular}{r|r}
  \hline
  (Intercept) & method S \\
  \hline
 1 & 0 \\
 1 & 0 \\
 1 & 0 \\
 1 & 1 \\
 1 & 1 \\
 1 & 1 \\
   \hline
\end{tabular}
\end{center}
\end{table}

\newpage




The random effects design matrix $\boldsymbol{Z_{i}}$ is given by:
\begin{table}[ht]
\begin{center}
\begin{tabular}{r|r}
  \hline
 method J & method S \\
  \hline
 1 & 0 \\
 1 & 0 \\
 1 & 0 \\
 0 & 1 \\
 0 & 1 \\
 0 & 1 \\
   \hline
\end{tabular}
\end{center}
\end{table}
\newpage



\end{document}
