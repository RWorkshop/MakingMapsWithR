% Demidenko
% Influence Analysis
%--------------------------------------%

%Standardized Residuals
Standardised residuals are typicallu used to detect outliers. However, the presence of outliers does not necessarily affect the model fit or any related statistical inference.

%Leverage
Leverage, defined as the identification of data points that influence the fitted values, are detected by exploring large values of the diagonal elements of the projection matrix (also known as the hat matrix.

%Cook's Distance
Cook and Weisber suggested analysin the standardised squared distance between the OLS estimate and the estimate after case deletion. This has become known as Cook's Distance.

The goal of Demidenko's paper is generalize several common measures of influence for the fixed effects parameters of an LME model.

% Generalization of leverage measure to a leverage matrix for the LME model.
% Proposal of a generalzation of Cook's Distance to the LME model
%-----------------------------------%


%------------------------------------%
The LME model is typciall estimated using restricted maximum likelihood (REML) which simultaneously produces an estimate of \boldsymbol{D} and \boldsymbol{\beta}.
 
%------------------------------------% 
%Hat Matrix (OLS)
The hat matrix is 
\[\boldsymbol{H}  =  \boldsymbol{X} (\boldsymbol{X}^{\prime}\boldsymbol{X})^{-1}\boldsymbol{X} ^{\prime} \]

%Leverage - page 895
Leverage is the partial derivate of  the predicted value with respect to the corresponding dependent variable.

Hence the $i-$th leverage indicates how the predicted value of the $i$th case is influenced by the $i$th observation.


Leverage Matrix  for the LME Model $n_i \times n_i$
\[ \boldsymbol{H}_i = \frac{ \partial \hat{\boldsymbol{y}}_i } {\partial \boldsymbol{y}_i } \]
