\documentclass[12pt, a4paper]{article}
\usepackage{natbib}
\usepackage{vmargin}
\usepackage{graphicx}
\usepackage{epsfig}
\usepackage{subfigure}
%\usepackage{amscd}
\usepackage{amssymb}
\usepackage{subfigure}
\usepackage{amsbsy}
\usepackage{amsthm, amsmath}
%\usepackage[dvips]{graphicx}
\bibliographystyle{chicago}
\renewcommand{\baselinestretch}{1.4}

% left top textwidth textheight headheight % headsep footheight footskip
\setmargins{3.0cm}{2.5cm}{15.5 cm}{23.5cm}{0.5cm}{0cm}{1cm}{1cm}

\pagenumbering{arabic}


\begin{document}
\author{Kevin O'Brien}
\title{Roy's Candidate models}
The original Bland Altman Method was developed for two sets of
measurements done on one occasion (i.e. independent data), and so
this approach is not suitable for repeated measures data. However,
as a na�ve analysis, it may be used to explore the data because of
the simplicity of the method. Myles states that such misuse of the
standards Bland Altman method is widespread in Anaesthetic and
critical care literature.
\\
\\
Bland and Altman have provided a modification for analysing
repeated measures under stable or chaninging conditions, where
repeated data is collected over a period of time. Myers proposes
an alternative Random effects model for this purpose.
\\
\\
 with repeated measures data, we can
calculate the mean of the repeated measurements by each method on
each individuals. \emph{ The pairs of means can then be used to
compare the two methods based on the 95\% limits of agreement for
the difference of means. The bias between the two methods will not
be affected by averaging the repeated measurements.}.However the
variation of the differences will be underestimated by this
practice because the measurement error is, to some extent,
removed. Some advanced statistical calculations are needed to take
into account these measurement errors. \emph{Random effects models
can be used to estimate the within-subject variation after
accounting for other observed and unobserved variations, in which
each subject has a different intercept and slope over the
observation period .On the basis of the within-subject variance
estimated by the random effects model, we can then create an
appropriate Bland Altman Plot.}The sequence or the time of the
measurement over the observation period can be taken as a random
effect.
\section{Random effects Model} \citet{Myles} proposes the use of
Random effects models to address the issue of repeated
measurement. Myles proposes a formulation of the Bland�Altman
plot, using the within-subject variance estimated by the random
effects model, with the time of the measurement taken as a random
effect. He states that \emph{random effects models account for the
dependent nature of the data, and additional explanatory
variables, to provide reliable estimates of agreement in this
setting.}
\\
Agreement between methods is reflected by the between-subject
variation.The Random Effects Model takes this into account before
calculating the within-subject standard deviation.
\subsection{Myers Random Effects Model} The presentation of the
95\% limits of agreement is for visual judgement of how well two
methods of measurement agree. The smaller the range between the
two, the better the agreement is The question of small is small is
a question of clinical judgement
\\
\\
Repeated measurements for each subjects are often used in clinical
research.

\subsection{Random Effects Modelling}
Random effects models are used to examine the within-subject
variation after adjusting for known and unknown variables, in
which each subject has a different intercept and slope over a time
period period.
\\
\\
\citet{Myles} remarks that the random effects model is an
extension of the analysis of variance method, accounting for more
covariates.
\\
\\
A random effect (in Myles's case, time of measurement) is chosen
to reflect the different intercept and slope for each subject with
respect to their change of measurements over the time period.
\\
\\
In Myles's methodology, the standard deviation of difference
between the means of the repeated measurements can be calculated
based on the within-subject standard deviation estimates.

A random effects model (also variance components model)is a type
of hierarchical linear model. Hierarchical linear modelling (HLM)
is a more advanced form of simple linear regression and multiple
linear regression. HLM is appropriate for use with nested
data.\\Faraway comments that the random effects approach is
\emph{more ambitious than the LME model in that it attempts to say
something about the wider population beyond the particular
sample}.


\end{document} 