\documentclass[]{article}

\usepackage{amsmath}
\usepackage{amssymb}
\usepackage{graphicx}

\begin{document}
\section{Bartko's Bradley-Blackwood Test}
This is a regression based
approach that performs a simultaneous test for the equivalence of
means and variances of the respective methods.We have identified
this approach  to be examined to see if it can be used as a
foundation for a test perform a test on
means and variances individually.
\begin{equation}
D = (X_{1}-X_{2})
\end{equation}
\begin{equation}
M = (X_{1} + X_{2}) /2
\end{equation}
The Bradley Blackwood Procedure fits D on M as follows:\\
\begin{equation}
D = \beta_{0} + \beta_{1}M
\end{equation}
\begin{itemize}
\item The Bradley Blackwood test is a simultaneous test for bias and
precision. They propose a regression approach which fits D on M,
where D is the difference and average of a pair of results.
\item Both beta values, the intercept and slope, are derived from the respective means and
standard deviations of their respective data sets.
\item We determine if the respective means and variances are equal if
both beta values are simultaneously equal to zero. The Test is
conducted using an F test, calculated from the results of a
regression of D on M.
\item We have identified this approach  to be examined to see if it can
be used as a foundation for a test perform a test on means and
variances individually.
\item Russell et al have suggested this method be used in conjunction
with a paired t-test , with estimates of slope and intercept.
\end{itemize}
%subsection{t-test}

%%%%%%%%%%%%%%%%%%%%%%%%%%%%%%%%%%%%%%%%%%%%%%%%%%%%%%%%%%%%%%%%%%%%%%%%%
%%%%%%%  Blackwood Bradley Model         %%%%%%%%%%%%%%%%%%%%%%%%%%%%%%%%%
%%%%%%%%%%%%%%%%%%%%%%%%%%%%%%%%%%%%%%%%%%%%%%%%%%%%%%%%%%%%%%%%%%%%%%%%%
\subsection{Blackwood Bradley Model} 

Bradley and Blackwood have developed a regression based approach
assessing the agreement.




\newpage
\section{Bradley-Blackwood Test (Kevin Hayes Talk)}
%--------------------------------------------------------------------------------%
% KH - UW

This work considers the problem of testing $\mu_1$ = $\mu_2$ and $\sigma^2_1 = \sigma^2_2$ using a random sample from a bivariate normal distribution with parameters $(\mu_1, \mu_2, \sigma^2_1, \sigma^2_2, \rho)$. 

The new contribution is a decomposition of the Bradley-Blackwood test statistic (\textit{Bradley and Blackwood, 1989})for the simultaneous test of {$\mu_1$ = $\mu_2$; $\sigma^2_1 = \sigma^2_2$}  as a sum of two statistics. 

One is equivalent to the Pitman-Morgan (\textit{Pitman, 1939; Morgan, 1939}) test statistic 
for $\sigma^2_1 = \sigma^2_2$ and the other one is a new alternative to the standard paired-t test of $\mu_D = \mu_1 = \mu_2 = 0$. 

Surprisingly, the classic Student paired-t test makes no assumptions about the equality (or otherwise) of the 
variance parameters. 

The power functions for these tests are quite easy to derive, and show that when $\sigma^2_1 = \sigma^2_2$, 
the paired t-test has a slight advantage over the new alternative in terms of power, but when $\sigma^2_1 \neq \sigma^2_2$, the 
new test has substantially higher power than the paired-t test.

While Bradley and Blackwood provide a test on the joint hypothesis of equal means and equal variances their regression based approach does not separate these two issues.

The rejection of the joint hypothesis may be 
due to two groups with unequal means and unequal variances; unequal means and equal variances, or equal means and unequal variances. We propose an approach for resolving this (model selection) problem in a manner controlling the magnitudes of the relevant type I error probabilities.


\end{document}