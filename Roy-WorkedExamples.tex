
\section{Roy's examples}
Roy provides three case studies, using data sets well known in method comparison studies, to demonstrate how the methodology should be used.


%--------------------------------------------------------------------------Example 1a  ----  JSR Data %

The first case study is the Systolic blood pressure data, taken from \citet{BA99}.


%--------------------------------------------------------------------------Example 1b  ----  JSR Data %

To complete the study, the relevant values are provided for the $R \mbox{vs} S$ comparison also.

%--------------------------------------------------------------------------Example 2  ----  PEFR Data %

The second data set, a comparison of two peak expiratory flow rate measurements, is referenced by \citet{BA86}.


%--------------------------------------------------------------------------Example 3 Cardiac Ejection Fraction Data %
The last case study is also based on a data set from  \citet{BA99}. It contains the measurements of left ventricular cardiac eject fraction, measured by impedance cartography and radionuclide ventriculography, on twelve patients.
The number of replicated differs for each patient.

The bias is shown to be $0.7040$, with a p-value of $0.0204$. The MLEa of the between-method and within-method variance-covariance matrices of methods $RV$ and $IC$ are given by

\begin{equation}\hat{D}=\left(
\begin{array}{cc}
1.6323 & 1.1427 \\
1.1427 & 1.4498 \\
\end{array}
\right),
\end{equation}



\begin{equation}\hat{\Sigma}=\left(
\begin{array}{cc}
1.6323 & 1.1427 \\
1.1427 & 1.4498 \\
\end{array}
\right).
\end{equation}

\citet{roy} notes that these are the same estimate for variance as given by \citet{BA99}.


The repeatability coefficients are determined to be $0.9080$ for the RV method and $1.0293$ for the IC method.

From the estimated $\boldsymbol{\Omega_{i}}$ correlation matrix, the overall correlation coefficient is $0.7100$.
The overall correllation coefficients between two methods RV and IC are $0.9384$ and $0.9131$ respectively.

\citet{roy} concludes that is appropriate to switch between the two methods if needed.

%--------------------------------------------------------------------------Example 4 Coronary Artery Calcium data%

\citet{haber}

\citet{roy} recommends to not switch between the two method.

\section{LME}

Fitting model according to Roy

\newpage
\begin{verbatim}
Linear mixed-effects model fit by REML
Data: BA99
AIC      BIC    logLik
4319.707 4336.629 -2155.853

Random effects:
Formula: ~1 | subj
(Intercept) Residual
StdDev:    29.39085 12.44454

Fixed effects: ob.js ~ method
Value Std.Error  DF  t-value p-value
(Intercept) 127.40784  3.281757 424 38.82306       0

methodS     15.61961   1.102107 424 14.17250       0

Correlation:
(Intr)
methodS -0.168

Standardized Within-Group Residuals:
Min          Q1         Med          Q3         Max
-3.61292639 -0.42538402 -0.02467651  0.40166235  4.84280044

Number of Observations: 510 Number of Groups: 85

\end{verbatim}
%----------------------------------------------------------------------------------------%

\newpage
\section{Roy's examples}
Roy provides three case studies, using data sets well known in method comparison studies, to demonstrate how the methodology should be used.


%--------------------------------------------------------------------------Example 1a  ----  JSR Data %

The first case study is the Systolic blood pressure data, taken from \citet{BA99}.


%--------------------------------------------------------------------------Example 1b  ----  JSR Data %

To complete the study, the relevant values are provided for the $R \mbox{vs} S$ comparison also.

%--------------------------------------------------------------------------Example 2  ----  PEFR Data %

The second data set, a comparison of two peak expiratory flow rate measurements, is referenced by \citet{BA86}.


%--------------------------------------------------------------------------Example 3 Cardiac Ejection Fraction Data %
The last case study is also based on a data set from  \citet{BA99}. It contains the measurements of left ventricular cardiac eject fraction, measured by impedance cartography and radionuclide ventriculography, on twelve patients.
The number of replicated differs for each patient.

The bias is shown to be $0.7040$, with a p-value of $0.0204$. The MLEa of the between-method and within-method variance-covariance matrices of methods $RV$ and $IC$ are given by

\begin{equation}\hat{D}=\left(
\begin{array}{cc}
1.6323 & 1.1427 \\
1.1427 & 1.4498 \\
\end{array}
\right),
\end{equation}



\begin{equation}\hat{\Sigma}=\left(
\begin{array}{cc}
1.6323 & 1.1427 \\
1.1427 & 1.4498 \\
\end{array}
\right).
\end{equation}

\citet{roy} notes that these are the same estimate for variance as given by \citet{BA99}.


The repeatability coefficients are determined to be $0.9080$ for the RV method and $1.0293$ for the IC method.

From the estimated $\boldsymbol{\Omega_{i}}$ correlation matrix, the overall correlation coefficient is $0.7100$.
The overall correllation coefficients between two methods RV and IC are $0.9384$ and $0.9131$ respectively.

\citet{roy} concludes that is appropriate to switch between the two methods if needed.

%--------------------------------------------------------------------------Example 4 Coronary Artery Calcium data%

\citet{haber}

\citet{roy} recommends to not switch between the two method.


\section{LME}

Fitting model according to Roy

\newpage
\begin{verbatim}
Linear mixed-effects model fit by REML
Data: BA99
AIC      BIC    logLik
4319.707 4336.629 -2155.853

Random effects:
Formula: ~1 | subj
(Intercept) Residual
StdDev:    29.39085 12.44454

Fixed effects: ob.js ~ method
Value Std.Error  DF  t-value p-value
(Intercept) 127.40784  3.281757 424 38.82306       0

methodS     15.61961   1.102107 424 14.17250       0

Correlation:
(Intr)
methodS -0.168

Standardized Within-Group Residuals:
Min          Q1         Med          Q3         Max
-3.61292639 -0.42538402 -0.02467651  0.40166235  4.84280044

Number of Observations: 510 Number of Groups: 85

\end{verbatim}

%---------------------------------------------------------------------------------------------------%
\newpage

\section{IC/RV comparison}



For the the RV-IC comparison, $\hat{D}$ is given by


\begin{equation}
\hat{D}= \left[ \begin{array}{cc}
1.6323 & 1.1427  \\
1.1427 & 1.4498 \\
\end{array} \right]
\end{equation}

The estimate for the within-subject variance covariance matrix is
given by
\begin{equation}
\hat{\Sigma}= \left[ \begin{array}{cc}
0.1072 & 0.0372  \\
0.0372 & 0.1379  \\
\end{array}\right]
\end{equation}
The estimated overall variance covariance matrix for the the 'RV
vs IC' comparison is given by
\begin{equation}
Block \Omega_{i}= \left[ \begin{array}{cc}
1.7396 & 1.1799  \\
1.1799 & 1.5877  \\
\end{array} \right].
\end{equation}

The power of the
likelihood ratio test may depends on specific sample size and the
specific number of  replications, and the author proposes
simulation studies to examine this further.

\newpage
For the the RV-IC comparison, $\hat{D}$ is given by


\begin{equation}
\hat{D}= \left[ \begin{array}{cc}
1.6323 & 1.1427  \\
1.1427 & 1.4498 \\
\end{array} \right]
\end{equation}

The estimate for the within-subject variance covariance matrix is
given by
\begin{equation}
\hat{\Sigma}= \left[ \begin{array}{cc}
0.1072 & 0.0372  \\
0.0372 & 0.1379  \\
\end{array}\right]
\end{equation}
The estimated overall variance covariance matrix for the the 'RV
vs IC' comparison is given by
\begin{equation}
Block \Omega_{i}= \left[ \begin{array}{cc}
1.7396 & 1.1799  \\
1.1799 & 1.5877  \\
\end{array} \right].
\end{equation}

The power of the
likelihood ratio test may depends on specific sample size and the
specific number of  replications, and the author proposes
simulation studies to examine this further.


% \begin{equation}
% data here
% \end{equation}
%==================================================================%

For the the RV-IC comparison, $\hat{D}$ is given by


\begin{equation}
\hat{D}= \left[ \begin{array}{cc}
1.6323 & 1.1427  \\
1.1427 & 1.4498 \\
\end{array} \right]
\end{equation}

The estimate for the within-subject variance covariance matrix is
given by
\begin{equation}
\hat{\Sigma}= \left[ \begin{array}{cc}
0.1072 & 0.0372  \\
0.0372 & 0.1379  \\
\end{array}\right]
\end{equation}
The estimated overall variance covariance matrix for the the 'RV
vs IC' comparison is given by
\begin{equation}
Block \Omega_{i}= \left[ \begin{array}{cc}
1.7396 & 1.1799  \\
1.1799 & 1.5877  \\
\end{array} \right].
\end{equation}

The power of the
likelihood ratio test may depends on specific sample size and the
specific number of  replications, and the author proposes
simulation studies to examine this further.
