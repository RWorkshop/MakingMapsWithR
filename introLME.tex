
% introMCS - Grubbs Data

\documentclass[Chap4main.tex]{subfiles}

% Load any packages needed for this document
\begin{document}
\newpage
\section{Linear Mixed Effects Models}

% \subsection{What are LME Models?}

% \subsection{Laird-Ware Notation}

\subsection{Why use LMEs for Method Comparison?}
The LME model approach has seen increased use as a framework for method comparison studies in recent years (Lai $\&$ Shaio, Carstensen and Choudhary as examples). In part this is due to the increased profile of LME models, and furthermore the availability of capable software. Additionally LME based approaches may utilise the diagnostic and influence analysis techniques that have been developed in recent times.


Roy proposes an LME model with Kronecker product covariance structure in a doubly multivariate setup. Response for $i$th subject can be written as
\[ y_i = \beta_0 + \beta_1x_{i1} + \beta_2x_{i2} + b_{1i}z_{i1}  + b_{2i}z_{i2} + \epsilon_i \]
\begin{itemize}
\item $\beta_1$ and $\beta_2$ are fixed effects corresponding to both methods. ($\beta_0$ is the intercept.)
\item $b_{1i}$ and $b_{2i}$ are random effects corresponding to both methods.
\end{itemize}

Overall variability between the two methods ($\Omega$) is sum of between-subject ($D$) and within-subject variability ($\Sigma$),
\[
 \mbox{Block } \boldsymbol{\Omega}_i = \left[ \begin{array}{cc} d^2_1 & d_{12}\\ d_{12} & d^2_2\\ \end{array} \right]
+ \left[\begin{array}{cc} \sigma^2_1 & \sigma_{12}\\ \sigma_{12} & \sigma^2_2\\ \end{array}\right].
\]

The well-known ``Limits of Agreement", as developed by Bland and Altman (1986) are easily computable using the LME framework, proposed by Roy. While we will not be considering this analysis, a demonstration will be provided in the example.

Further to this, Roy(2009) demonstrates an suite of tests that can be used to determine how well two methods of measurement, in the presence of repeated measures, agree with each other.

\begin{itemize}\itemsep0.5cm
\item No Significant inter-method bias
\item No difference in the between-subject variabilities of the two methods
\item No difference in the within-subject variabilities of the two methods
\end{itemize}
\bibliography{DB-txfrbib}
\end{document}

