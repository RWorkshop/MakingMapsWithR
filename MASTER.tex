\documentclass[12pt, a4paper]{report}
\usepackage{epsfig}
\usepackage{subfigure}
%\usepackage{amscd}
\usepackage{amssymb}
\usepackage{amsbsy}
\usepackage{amsthm}
\usepackage{subfiles}
%\usepackage[dvips]{graphicx}
\usepackage{natbib}
\usepackage{subfiles}
\bibliographystyle{chicago}
\usepackage{vmargin}
% left top textwidth textheight headheight
% headsep footheight footskip
\setmargins{3.0cm}{2.5cm}{15.5 cm}{22cm}{0.5cm}{0cm}{1cm}{1cm}
\renewcommand{\baselinestretch}{1.5}
\pagenumbering{arabic}
\theoremstyle{plain}
\newtheorem{theorem}{Theorem}[section]
\newtheorem{corollary}[theorem]{Corollary}
\newtheorem{ill}[theorem]{Example}
\newtheorem{lemma}[theorem]{Lemma}
\newtheorem{proposition}[theorem]{Proposition}
\newtheorem{conjecture}[theorem]{Conjecture}
\newtheorem{axiom}{Axiom}
\theoremstyle{definition}
\newtheorem{definition}{Definition}[section]
\newtheorem{notation}{Notation}
\theoremstyle{remark}
\newtheorem{remark}{Remark}[section]
\newtheorem{example}{Example}[section]
\renewcommand{\thenotation}{}
\renewcommand{\thetable}{\thesection.\arabic{table}}
\renewcommand{\thefigure}{\thesection.\arabic{figure}}
\title{Research notes: linear mixed effects models}
\author{ } \date{ }

\begin{document}
	\author{Kevin O'Brien}
	\title{Print Ready Draft for Chaps 1 and 2}
	\date{\today}
	\maketitle
	
	\tableofcontents \setcounter{tocdepth}{2}
	
	\newpage

	\begin{itemize}
	\item \textbf{Working Title:} Linear Mixed Effects Model Diagnostics applied to Method Comparison Studies.\\
	 Fundamental Areas introduced in order
	\begin{itemize}
		\item[(1)] Method Comparison Studies
		\item[(2)] Linear Mixed Effects Models
		\item[(3)] Model Diagnostics for Lnear and LME models
	\end{itemize}
	\item \textbf{Chapter 1} : Introduction to Method Comparison Studies (10 pages approx)
    \item \textbf{Chapter 2} : Overview of existing methdologies	
    	\begin{itemize}
    		\item Grubb's Formulation
    		\item Formal Methods (e.g. Pitman-Morgan)
    		\item Bland Altman Methods
    		\item Regressions Models (incl Deming Regression and Bartko's Method)
    		\item LME models (Carstensen)
    		\item Roy's Tests for MCS
    		\item R implementation
	\end{itemize}
\end{itemize}
\subfile{IntroMCS.tex}
	\chapter{Review of MCS Methodologies}
	\section{Bland-Altman methodology}
	The issue of whether two measurement methods comparable to the 	extent that they can be used interchangeably with sufficient
	accuracy is encountered frequently in scientific research.
	Historically comparison of two methods of measurement was carried 	out by use of paired sample $t-$test, correlation coefficients or
	simple linear regression. Simple linear regression is unsuitable for method comparison studies because of the required assumption that one variable is measured without error. In comparing two methods, both methods are assume to have attendant random error.
	
	Statisticians Martin Bland and Douglas Altman recognized the inadequacies of these analyzes and
	articulated quite thoroughly the basis on which of which they are unsuitable for comparing two methods of measurement \citep*{BA83}. Furthermore they proposed their simple methodology specifically constructed for method comparison studies. They acknowledge the opportunity to apply other valid, but complex, methodologies, but argue that a simple approach is preferable, especially when the
	results must be `explained to non-statisticians'.
	
	Notwithstanding previous remarks about linear regression, the first step recommended, which the authors argue should be mandatory, is construction of a simple scatter plot of the data. The line of equality should also be shown, as it is necessary to give the correct interpretation of how both methods compare. In the case of good agreement, the observations would be distributed closely along the line of equality. A scatter plot of the Grubbs data is shown in Figure 1.1. Visual inspection confirms the previous conclusion that there is an inter-method bias present, i.e. Fotobalk device has a tendency to record a lower velocity.
	
	\begin{figure}[h!]
		\begin{center}
			\includegraphics[width=125mm]{GrubbsScatter.jpeg}
			\caption{Scatter plot For Fotobalk and Counter Methods.}\label{GrubbsScatter}
		\end{center}
	\end{figure}
	
	\citet{Dewitte} notes that scatter plots were very seldom presented in the Annals of Clinical Biochemistry. This apparently
	results from the fact that the `Instructions for Authors' dissuade the use of regression analysis, which conventionally is accompanied by a scatter plot.
	
	\newpage
	\subsection{Bland-Altman plots}
	
	In light of shortcomings associated with scatterplots, \citet*{BA83} recommend a further analysis of the data. Firstly
	case-wise differences of measurements of two methods $d_{i} = y_{1i}-y_{2i} \mbox{ for }i=1,2,\dots,n$ on the same subject
	should be calculated, and then the average of those measurements ($a_{i} = (y_{1i} + y_{2i})/2 \mbox{ for }i=1,2,\dots, n$).
	
	\citet{BA83} proposes a scatterplot of the case-wise averages and differences of two methods of measurement. This scatterplot has since become widely known as the Bland-Altman plot. \citet*{BA83} express the
	motivation for this plot thusly:
	\begin{quote}
		``From this type of plot it is much easier to assess the magnitude
		of disagreement (both error and bias), spot outliers, and see
		whether there is any trend, for example an increase in (difference) for high values. This way of plotting the data is a very powerful way of displaying the results of a method comparison study."
	\end{quote}
	
	The case wise-averages capture several aspects of the data, such as expressing the range over which the values were taken, and assessing whether the assumptions of constant variance holds.
	Case-wise averages also allow the case-wise differences to be presented on a two-dimensional plot, with better data visualization qualities than a one dimensional plot. \citet{BA86}
	cautions that it would be the difference against either measurement value instead of their average, as the difference relates to both value. This methodology has proved very popular, and the Bland-Altman plots is widely regarded as powerful graphical methodology for making a visual assessment of the data.
	
	The magnitude of the inter-method bias between the two methods is simply the average of the differences $\bar{d}$. This inter-method bias is represented with a line on the Bland-Altman plot. As the objective of the Bland-Altman plot is to advise on the agreement of two methods, it is the case-wise differences that are also particularly relevant. The variances around this bias is estimated by the standard deviation of these differences $S_{d}$.
	
	\subsection{Bland-Altman plots for the Grubbs data}
	
	In the case of the Grubbs data the inter-method bias is $-0.61$ metres per second, and is indicated by the dashed line on Figure 1.2. By inspection of the plot, it is also possible to compare the precision of each method. Noticeably the differences tend to increase as the averages increase.
	
	
	The Bland-Altman plot for comparing the `Fotobalk' and `Counter'
	methods, which shall henceforth be referred to as the `F vs C' comparison,  is depicted in Figure 1.2, using data from Table 1.3.
	The presence and magnitude of the inter-method bias is indicated
	by the dashed line.
	\newpage
	
	%Later it will be shown that case-wise differences are the sole
	%component of the next part of the methodology, the limits of
	%agreement.

	
	\begin{table}[h!]
		\renewcommand\arraystretch{0.7}%
		\begin{center}
			\begin{tabular}{|c||c|c||c|c|}
				\hline
				Round & Fotobalk  & Counter  & Differences  & Averages  \\
				&  [F] & [C] & [F-C] &  [(F+C)/2] \\
				\hline
				1 & 793.8 & 794.6 & -0.8 & 794.2 \\
				2 & 793.1 & 793.9 & -0.8 & 793.5 \\
				3 & 792.4 & 793.2 & -0.8 & 792.8 \\
				4 & 794.0 & 794.0 & 0.0 & 794.0 \\
				5 & 791.4 & 792.2 & -0.8 & 791.8 \\
				6 & 792.4 & 793.1 & -0.7 & 792.8 \\
				7 & 791.7 & 792.4 & -0.7 & 792.0 \\
				8 & 792.3 & 792.8 & -0.5 & 792.5 \\
				9 & 789.6 & 790.2 & -0.6 & 789.9 \\
				10 & 794.4 & 795.0 & -0.6 & 794.7 \\
				11 & 790.9 & 791.6 & -0.7 & 791.2 \\
				12 & 793.5 & 793.8 & -0.3 & 793.6 \\
				\hline
			\end{tabular}
			\caption{Fotobalk and Counter methods: differences and averages.}
		\end{center}
	\end{table}
	
	\begin{table}[h!]
		\renewcommand\arraystretch{0.7}%
		\begin{center}
			\begin{tabular}{|c||c|c||c|c|}
				\hline
				Round & Fotobalk  & Terma  & Differences  & Averages  \\
				&  [F] & [T] & [F-T] &  [(F+T)/2] \\
				\hline
				1 & 793.8 & 793.2 & 0.6 & 793.5 \\
				2 & 793.1 & 793.3 & -0.2 & 793.2 \\
				3 & 792.4 & 792.6 & -0.2 & 792.5 \\
				4 & 794.0 & 793.8 & 0.2 & 793.9 \\
				5 & 791.4 & 791.6 & -0.2 & 791.5 \\
				6 & 792.4& 791.6 & 0.8 & 792.0 \\
				7 & 791.7 & 791.6 & 0.1 & 791.6 \\
				8 & 792.3 & 792.4 & -0.1 & 792.3 \\
				9 & 789.6 & 788.5 & 1.1 & 789.0 \\
				10 & 794.4 & 794.7 & -0.3 & 794.5 \\
				11 & 790.9 & 791.3 & -0.4 & 791.1 \\
				12 & 793.5 & 793.5 & 0.0 & 793.5 \\
				
				\hline
			\end{tabular}
			\caption{Fotobalk and Terma methods: differences and averages.}
		\end{center}
	\end{table}
	
	\newpage
	
	\begin{figure}[h!]
		\begin{center}
			\includegraphics[width=120mm]{GrubbsBAplot-noLOA.jpeg}
			\caption{Bland-Altman plot For Fotobalk and Counter methods.}\label{GrubbsBA-noLOA}
		\end{center}
	\end{figure}
	
	
	
	In Figure 1.3 Bland-Altman plots for the `F vs C' and `F vs T'
	comparisons are shown, where `F vs T' refers to the comparison of
	the `Fotobalk' and `Terma' methods. Usage of the Bland-Altman plot
	can be demonstrate in the contrast between these comparisons. By inspection, there exists a larger inter-method bias in the `F vs C' comparison than in the `F vs T' comparison. Conversely there
	appears to be less precision in `F vs T' comparison, as indicated
	by the greater dispersion of covariates.
	
	\begin{figure}[h!]
		\begin{center}
			\includegraphics[height=90mm]{GrubbsDataTwoBAplots.jpeg}
			\caption{Bland-Altman plots for Grubbs' F vs C and F vs T comparisons.}\label{GrubbsDataTwoBAplots}
		\end{center}
	\end{figure}
	
	\newpage
	
	
	\subsection{Adverse features}
	
	Estimates for inter-method bias and variance of differences are only meaningful if there is uniform inter-bias and variability throughout the range of measurements. Fulfilment of these assumptions can be checked by visual inspection of the plot.The prototype Bland-Altman plots depicted in Figures 1.4, 1.5 and 1.6 are derived from simulated data, for the purpose of demonstrating how the plot would inform an analyst of features that would adversely affect use of the recommended methodology.
	
	Figure 1.4 demonstrates how the Bland-Altman plot would indicate
	increasing variance of differences over the measurement range.
	Fitted regression lines, for both the upper and lower half of the
	plot, has been added to indicate the trend. Figure 1.5 is an
	example of cases where the inter-method bias changes over the
	measurement range. This is known as proportional bias, and is
	defined by \citet{ludbrook97} as meaning that `one method gives values that are higher (or lower) than those from the other by an 	amount that is proportional to the level of the measured variable'. In both Figures 1.4 and 1.5, the assumptions necessary
	for further analysis using the limits of agreement are violated.
	
	Application of regression techniques to the Bland-Altman plot, and
	subsequent formal testing for the constant variability of
	differences is informative. The data set may be divided into two
	subsets, containing the observations wherein the difference values
	are less than and greater than the inter-method bias respectively.
	For both of these fits, hypothesis tests for the respective slopes
	can be performed. While both tests can be considered separately,
	multiple comparison procedures, such as the Benjamini-Hochberg
	\citep{BH} test, should be also be used.
	
	\begin{figure}[h!]
		\begin{center}
			\includegraphics[height=90mm]{BAFanEffect.jpeg}
			\caption{Bland-Altman plot demonstrating the increase of variance over the range.}\label{BAFanEffect}
		\end{center}
	\end{figure}
	
	\begin{figure}[h!]
		\begin{center}
			\includegraphics[height=90mm]{PropBias.jpeg}
			\caption{Bland-Altman plot indicating the presence of proportional bias.}\label{PropBias}
		\end{center}
	\end{figure}

	
	\newpage
	
	
	\subfile{LOA.tex}
	\subfile{BAtests.tex}
% \subfile{LOA.tex}
	
	
	
	
	
	\subsection{Replicate Measurements}
	
	Thus far, the formulation for comparison of two measurement methods is one where one measurement by each method is taken on	each subject. Should there be two or more measurements by each methods, these measurement are known as `replicate measurements'.
	\citet{BXC2008} recommends the use of replicate measurements, but acknowledges the additional computational complexity.
	
	\citet*{BA86} address this problem by offering two different approaches. The premise of the first approach is that replicate
	measurements can be treated as independent measurements. The second approach is based upon using the mean of the each group of
	replicates as a representative value of that group. Using either
	of these approaches will allow an analyst to estimate the inter
	method bias.
	
	%\subsubsection{Mean of Replicates Limits of Agreement}
	
	However, because of the removal of the effects of the replicate
	measurements error, this would cause the estimation of the
	standard deviation of the differences to be unduly small.
	\citet*{BA86} propose a correction for this.
	
	\citet{BXC2008} takes issue with the limits of agreement based on
	mean values of replicate measurements, in that they can only be interpreted as prediction
	limits for difference between means of repeated measurements by
	both methods, as opposed to the difference of all measurements.
	Incorrect conclusions would be caused by such a misinterpretation.
	\citet{BXC2008} demonstrates how the limits of agreement
	calculated using the mean of replicates are `much too narrow as
	prediction limits for differences between future single
	measurements'. This paper also comments that, while treating the
	replicate measurements as independent will cause a downward bias
	on the limits of agreement calculation, this method is preferable
	to the `mean of replicates' approach.
	
	
	
	\subsection{Identifiability}
	\citet{DunnSEME} highlights an important issue regarding using
	models such as these, the identifiability problem. This comes as a result of there being too many parameters to be estimated.
	Therefore assumptions about some parameters, or estimators used,
	must be made so that others can be estimated. For example in literature the variance
	ratio $\lambda=\frac{\sigma^{2}_{1}}{\sigma^{2}_{2}}$
	must often be assumed to be equal to $1$ \citep{linnet98}.\citet{DunnSEME} considers methodologies based on two methods with single measurements on each subject as inadequate for a serious
	study on the measurement characteristics of the methods. This is
	because there would not be enough data to allow for a meaningful
	analysis. There is, however, a contrary argument that in many
	practical settings it is very difficult to get replicate
	observations when the measurement method requires invasive medical
	procedure.
	

	\subfile{RegressionMethods.tex}
	% \subfile{Regression1.tex}
	\subfile{DemingRegression}

	

	%===================================================================================================================== %

	
	\subfile{AlternativeGraphicalApproaches.tex}
	\subfile{AgreementIndices.tex}
	\subfile{Mantha.tex}

	\subfile{TechAcceptModel.tex}


%	\subfile{DemingVarianceRatio}
%	\subfile{LeastProductRegression}
%	\subfile{RauchOutliers}
%\subfile{LMEnotes.tex}

%          \subfile{introLME.tex}
\bibliography{DB-txfrbib}
\end{document}


 %\subfile{ThesisOutline.tex}


%================================================%
 %\subfile{LMEmodelequation.tex}
 %\subfile{MLandREML.tex}
 %\subfile{LairdWare.tex}
 %\subfile{HendersonEquations}
 %\subfile{Regression}

 %\subfile{KenwardRogers}
 %\subfile{BXC}
 %\subfile{IntroToRoysTests}
 %\subfile{matrixtypes}
	
	%===================================================================================================================== %
	
