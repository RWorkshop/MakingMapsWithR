\documentclass[]{article}

%opening
\title{}
\author{}

\begin{document}

\subsubsection{Example 1: PEFR}
\begin{itemize}
\item	The mathematical indeterminacy in the one-measurement case is well known. Bland
	and Altman often avoid an explicit model in their Work so it is not clear if they
	are aware of this problem. In fact, they state (1983) that the “considerable extra
	complexity of such analysis will not be justified if a simple comparison is all that is
	required” and that this is especially true when the “results must be conveyed to and
	used by non-experts, e.g. clinicians.”
\item We disagree with this comment, and show in
	our Example 4 that summaries of results need not be complex. We are also preparing a paper for publication that we hope will make such modeling easier for practitioners to understand, perform, and summarize.
	
\item Using the structural equation model, What effect does this problem of indeterminacy have on making practical decisions when comparing two devices? lt can have a very important effect, as we illustrate with an example. Keep in mind that we
	are only considering the one-measurement case—if multiple measurements are made on each subject this indeterminacy disappears, although the BA plot alone (now of averages for each subject) still does not capture all the information needed.
\item Our example focuses on parameter estimation, not parameter uncertainty. The results we show hold true whether the data sets are based on N = 10 or N = 10, 000—
	parameter uncertainty is not the issue in parameter indeterminacy.
\end{itemize}
%================================================================================================================== %
\newpage
\begin{itemize}
	\item Here is our approach. To get estimates of the parameters, we will perform a
sensitivity analysis. We do this by pretending that one of the six quantities is known, and then varying this quantity over a range of values. Each fixed value of this quantity
allows us to find estimates of the remaining quantities.
 \item An examination of (4)-(5) reveals that the values of B0 and ax only appear in 
The oz parameter is not useful when comparing two methods. The /50 parameter,
while important (measuring one aspect of bias), is fairly straightforward to correct. For these reasons, we will simplify our analysis by ignoring this information. This
leaves us with four unknown parameters for which we have three estimates, as shown
in (5) or  We will use the latter form of the variance-covariance matrix, so (am, 51, R5,, R55) are the unknown parameters.
One of the strong disagreements We have with the BA method is that the BA
method cannot indicate which device is more precise. But we believe that this precision
question, and not a measure of agreement, is often a crucial measure by which a device
should be judged. For this reason we select a variety of R55 values for our sensitivity
analysis. 
\item The R55 parameter also has a range of possible values that is independent
of the data, making it a natural choice for sensitivity work.
The relationship of (0,, 51, RM) to (E, R55) is presented in Appendix A. For a set
of data, We can obtain the sample variance-covariance matrix S from the differeiices
and averages. We replace E with S in (6) and equate the terms to find estimates of
(0,, B1, R5,) as a function of R5,; for this S.
\item The example We use is from Altman (1995), p. 270, and details are provided in
Bland and Altman (1986). We selected this example because Bland and Altman have
used it repeatedly to explain their methods. 
\item The data set is based on N I 17 subjects.
PEFR (peak expiratory flow rate) measurements Were obtained for each subject, both
on a Wright meter (X) and a Mini meter  (In fact, several measurements were
made 011 each device for each subject, but Altman used only the first measurement
from each device as an example of the BA plot.) 
\item We will usually suppress the units,
liters/min, for brevity. The sample variance-covariance matrix of this set of data is
(variance of Y — X) S11 I 15027353, (variance of (Y + X) /2) S22 I 127861324,
and (covariance) S12 I 366.8015.
\item Based on this matrix, We generated the results in Figure 1. We have chosen to
display this graph over a Wide range of R55 I U5/0'5 values to emphasize that the
data themselves provide no evidence of this value. Without additional information,
such as that provided by making repeat measurements on each subject, any of the
values shown on this graph are equally likely (equal likelihoods) to have given rise to
the data observed.
\end{itemize}
\end{document}
For example, consider three R85 scenarios:
\begin{itemize}
\item[1.] Mini meter much less precise than the Wright meter, by a factor of 10. With
08/<75 I 10, We find [31 I 0.97, 6,; I 3.84, rig I 38.4.
\item[2.] Mini meter as precise as the Wright meter. With 0'5/0'5 I 1, we find 31 I 1.03,
6,; I 65 I 27.3.
\item[3.] Mini meter much more precise than the Wright meter, by a factor of 10. With
as/0,; I 0.1, we find 51 I 1.09, 6,; I 37.4, 65 I 3.74.
\end{itemize}
Bland (1995, p. 272) noted \textit{“the standard deviation of the differences is estimated
to be 38.8 litres/min ...”} He further adds that (italics ours) “[0/n the basis of these
data we would not conclude that the two devices are comparable or that the mini-meter
could reliably replace the Wright peak flow meter.” (He added that the Mini meter
had received considerable wear, but this is outside the scope of the data themselves.)
9



bu:
Est mates of 05
Figure 1:
5 ,
4 ,
3 V i '1.10
2 ' ~\\ \ _
\
101)  \ ’ 1.05
» j I3-\ _
\\
, \ i
\\
__ 2 ' ‘\\7ii _
0% i
. 55
<.- >0-mm»
> uwmw
_Q
10-zen 2 245610400 2 3456100110 2 345510100 2 345610200
Ratioq/<56
PEFR: Estimates of 0'5, 0,5 , and B1 as a function of R5,; = as 0'5
10
'0
o
' ' 0.95
' 0.90
Est mates of B



39.0
w
9°
w
Est mate of 0,,-,,
an
9°
o
37.5
l    
10-zoo 2 345610-1.00 2 345610000 2 3A581O100 2 345510100
Ratio 6,;/65
Figure 2: PEFR: Estimate of ad,” = ‘/0? + 03 as a function of 05/05.
We would instead conclude on the basis of these data that there is absolutely
no evidence for whether the two devices are comparable or whether the Mini meter
or the Wright meter is a more precise measuring device such estimates can not be
obtained from these data.
In the R5,; limits, the parameter fil (or its estimate, for a sample of data) is the
regression parameter of Y on X (if R5,; t? oo) or the the inverse of the regression
parameter of X on Y (if R85 —> O). In the case where R5,; = l (U? = 0?), B1 is
the orthogonal (principal component) regression parameter. For example, the slope
estimate found by regressing the Wright data on the Mini data is 0.97, which is the limit of Z31 as R55 —> oo, and the slope estimate found by regressing the Mini data on
the Wright data is 0.92, and 1/0.92 : 1,09, the limit of B, as R5,; a 0.
Bland’s estimated variance of the differences is (38.8)2, which is simply S11. Prom
(5), this estimate is biased upwards unless fll = 1. However, for this data set, the
bias is slight see Figure 2. The maximum occurs at 38.8 for the value of Rd; that
corresponds to B1 = 1. The relatively stability shown in this example corresponds to
comments made in Bland and Altman (1995).
Finally, we examine how 5*, varies as a function of R5,;—see Figure 3. This feature
ll



115 '
Est mate of ox
o
4//
t,    
10-2.00 2 3A5610-100 2 345510000 2 3456101.u0 2 3A56102uo
Ratio 6,;/65
Figure 3: PEFR: Estimate of U, as a function of R55 = U5/05‘
of the data exhibits relatively slight variation. Because this is a parameter associated
with the population being measured rather than the measurement devices themselves,
we will not examine this parameter any further.

To emphasize that any of the values associated with these figures are equally likely
to have given rise the observed data, We generated 100 pairs of simulated data from
each of the three R55 scenarios listed above‘ We also estimated the corresponding pm
and 50 to center the results appropriately. We then graphed the PEFR data and also
overlaid the PEFR data for each scenario. We selected N I 100 for each scenario to
smooth out some of the random variation.

See Figure 4. The three scenarios have very different values of the underlying
parameters, but result in bivariate data that are being sampled from the same distri-
bution. The BA plots, which we do not graph here, are simply a rotation and possible
re-scaling of these results.
\newpage
\subsection*{6 Information Contained in the Bland-Altman Plot}
A statistical analysis frequently consists of addressing two broad questions‘ First,
using a tentative mathematical model and assuming the model provides a reasonable
12



Figure 4: PEFR Data (solid squares) and Simulations (open circles) of Three Scenar-
IOS.
800
§4oo
200
o
800
600
400
200
soo I
, -I F
IF
PEFR Data
O 200 400 600 800
Wright
with Scenario 2
O
O
O O
Q)‘O
0%
mag
°9
d9
Q 
O 200 400 600 800
sou i
600 '
$5
O
o ‘?>
400' W O
O, O
r nwoo
0 200 400 600 B00
with Scenario 1
O _
0
with Scenario 3
oooD8>
E
7 0%,;
7 < 
- gtggqa
0
0 200 400 600 800



representation of the data, What estimates (and uncertainties) are associated with the
parameters of the model? Such estimates, by providing a summary of the data if the
model is reasonable, can then be used to make summaries and help aid in decisions.
Second, does the tentative model in fact provide a reasonable representation of the
data? If so, good; if not, decisions need to be made whether to use the model anyway,
to use another model, or to abandon the idea of modeling. A simple example is in
linear regression, in which a tentative model is used and the data are fit to the model
assuming the model is reasonable; and then a residual analysis is performed to see if
the model is reasonable.

This article has emphasized the first aspect of this analysis, for which the BA plot
does not provide enough information. However, the BA plot is valuable in addressing
whether the model is reasonable, For example, the plot can be useful in detecting
outliers, in detecting higher variability at higher readings, and in detecting non-
linear relationships. Bland and Altman use such checks as part of their method.
Note however, that checks of a constant difference (fil I 1) can be misleading, as We
illustrate in Example 2.

We have emphasized the one-measurement case, because this appears to be how
the BA method is most frequently used. We strongly recommend that no measurement-
comparison study ever be conducted unless it includes multiple measurements per
subject on each machine (unless this is physically impossible). If so, then all six pa-
rameters may be estimated under the structural model. In fact, because the minimal
suflicient statistics are of dimension seven, this structural model can be tested against
the extended structural model. See Example 4.

In such a case, the BA plot should be based on the average readings for each
subject on each machine. Without explicitly incorporating information on the esti-
mates of 0'5 and <75, the plot still suffers the same indeterminacy and its use should
be restricted to graphical examination of the assumptions behind the model.
7 Example 2. Blood Pressure

Bland and Altman are aware that problems can exist in interpreting graphs in which
measurement variation exists. For example, in their 1995 paper, they criticized the
method that has sometimes been advocated of plotting the difference against one of
the measurements, say X, often the “gold standard”—see their paper for references.
They correctly point out how this can be misleading, and instead recommended the
Bland-Altman approach.

Plotting the difference against the X measurements is misleading, because
Y - X
l X l N MvN(,i”, 2”),
14



where (using the structural model again)
I! »30+(51_1)/11]
= 7
1» 1 ,1, < >
E” = U? [(51 _ U2 + Rgz (R26 +1) 51* 1 — Rim]
” Sym 1 + Rgz
and the resulting correlation is
PX)/9'Y*¢7X /3i’1’R§I
/7Y—X,X = = ~
\/vi + J? - Qflxyvxvy \/[(51 -1)? + R§,, (R5, +1)l (1 + Rgx)
The middle term, involving the observable (X, Y) data, appears in the Bland and
Altman (1995) paper. Even if two measurement devices are identical (flo = 0, B1 = 1,
R5,; : 1), a spurious correlation appears in the data because
R61
PY-X,X : i
,/2 (1 + R3,)
Bland and Altman (1995) also noted the correlation from their plot. It is (first
line appears in their paper)
U2 _ U2
/'Y—X,(X+Y)/2 I   (3)
\/ (”§<+"§) -4/’§<y"§<@’§
:  T 1 +  (R56 7 1) (9)
\/[(51 -1)2 + Rgw (Rg, +1)] [(51 +1)2 + Rgw (rag, +1)]
They state “[t]his is zero if the variances are equal, and will be small unless there
is a marked di1'lerence in the variability between subjects for the two methods" They
precede this by stating “[i]f the study includes a Wide range of measurements, and
unless the two methods of measurement have very poor agreement, we expect 03¢ and
0%, to be similar and p to be fairly large, at least 0.7."
The set of data they examined was a random sub-sample of 200 blood pressure
readings from a larger study—the sample of size 200 was used to avoid clutter in their
graphs. For those data, the estimate of /Jy_X,(X+y)/2 is U17, with a 95% confidence
interval that excludes 0. They note, in line with (9) that this correlation may be
non-zero if either there is a trend in the relationship (our fil > 1, for example) or if
“one method has considerably more measurement error than the other...” and go on
to note that this can be estimated only by making repeated measurements.
15



2 L 1.4
101° 1 7 *~—\
<.= > uumww
//
E»
- Ni. u ~
2 \ '
\
\
6:
Est mates of <1,
Q
o= > U'\O\~l@ O
2 S
Est mates of [5
V \ _ ._
._ e \
\
\\ 6
I \ » * 1.0
2 6‘ ([3,=1 ref ||ne) \\
\\\\ -
10'“;
10-200 2 345610-100 2 345610000 2 3A55101.O0 2 3A561O2DD
Ratio 6‘/0§
Figure 5:
From the information supplied in that article, a reasonable approximation to the
sample variance-covariance matrix of the differences and averages may be retrieved,
yielding S11 = 213.16, S12 = 58.00, S22 = 546.71. This was used to generate Figure 5.
These figures are generally in line with their cornrrients, but serve again to indicate the
indeterminacy in the one-measurement case. For example, the changing bias in the
sample could range as high as 33% per unit ($1 : 1133) to —9% per unit  I 0.91).
We believe that the sample correlation of 0.17 and its statistical significance would
likely be taken by most users to indicate that fil > 1. However, the entire range of
values is once again equally likely to have generated these data, including [31 : 1,
6,; = 711,65 = 12.8.
8 Example 3: PEFR Revisited
We will use this example to outline the approach we recommend in the multiple-
measurement case. We plan to explain this approach elsewhere in a style designed for the non-statistician. \Ve again use the PEFR data from Bland and Altman (1986). There were N = 17 subjects, each of whom was measured twice on each of the Wright
16



Wright Meter \ Mini Meter \
I XXXX Z XXXXXXXXXXXX
: XXXX : X
1 XXXX : XXXX
1 XXXX : X
I xx
: XXXX : XXXXX
unoaqmuv
\.0oo\|o\u|
>< ><
I XXX
: XXXXX : xxx
: XXX : XXXX
..|>cor\>|-0
>4 X
X
u>o:r\>»-0
Figure 6: PEFR: Stem-and-Leaf Plots of Last Digits of Wright and Mini Meters. and Mini meters. As those authors noted, those data were collected to illustrate a
statistical method, and we will use it with this in mind as well—the small sample size would normally be too small for decision making.
Like Bland and Altman, we begin by a graphical examination of the data. However, those authors used only the first measurement of each method to illustrate the
comparison of methods, and used the second measurement only for studying repeata-
bility. Although it is clear to us why they may have done so, it is an inefficient use of
data and, we believe, does not set a good example for data analysis. We will examine
all the data simultaneously.
Before any means or differences are calculated, it is important to look at the
raw data themselves. This cal1 provide unexpected insights. Here, we discovered a
difference between the two measuring devices that, to our knowledge, has not been
noted by Bland and Altman in their publications. The existence of round-off error
of a measuring device can often be considered by examining only the last digit—e.g.,
516 is reduced to a 6. See Figure 6.
The actual readings had a range of over 400 for each meter, so the last digit
is expected to be randomly distributed. This is observed for the Wright Meter.
However, the Mini Meter readings suffer from rounding this is most obvious by the
large number of ()’s, but also reveals itself in the peaks at what we shall call 2.5, 5,
and 7.5. The smaller three peaks are of little consequence here, but the high peak at
O suggests that some readings may have been misread by as much as 5 units. We do
not examine this issue further in the analysis.
Our next plots continue to examine the features of each machine. For each ma-
chine, we recommend a plot of standard deviation vs mean for each subject. However,
for the two-measurement case, a plot of difference vs mean can be used instead—Bland
and Altman (1986) made these same, natural, suggestions. There are two main pur-
poses for such a plot: to see if the standard deviation increases with the mean, and
if so perform an alternate analysis (such as a log transformation of the raw data);
17



0 Wright Differences
50' A MiniDifferences O
O A
A
0.
A
O
-so - O
A
-100 —
\ I ‘ \ \ \
-2 -1 0 1 2
Normal z-scores
%====================================================================================================== %
Figure 7: PEFR: Normal Quantile-Quantile Plot of Wright and Mini Differences.
and to look for outliers. Such graphs, not shown here, reveal a very large outlier for
the Mini meter, which Bland and Altman (1986) noted. However, there appear to be
problems with the Wright meter as well, which the authors did not note. Because the
standard deviation does not increase with the mean, we show both sets of differences
with normal quantile-quantile plots. See Figure 7.
%====================================================================================================== %
The outlying value of —96 stands out for the Mini Meter, but there appears to be
at least one outlier for the Wright data as well. Using the method of Grubbs (1969)
and an algorithm supplied by NIST ( http://www.itl.nist.gov/div898/handbook/eda/
section3/eda35h.htm) based on G I H1%LX(D1' — D) /$13, where the {D,,i : 1, ..., 17}
are the differences using one of the devices and SD is the sample standard deviation
of the the {Di}, we find G = 3.23 for Mini, which is significant at P = 0.0006.
This is the only such outlier in the Mini data, and we will set it aside (subject #7).
However, the V\/right data has G = 2.71 and P = 0.03. If this point is set aside and
we re-run the test, we have G = 2.64 and P = 0.04, so there is some evidence of
two outliers. However, this evidence of outliers had P > 0.01, and so we decided to
keep these points. This is in line with Grubbs’ (1969), who wrote “it is generally
recommended that a low significance level, such as 1%, be used...”. (The NIST site
uses G above, but the Grubbs statistic used only the “one-sided” version of this test.
18



I
600  O 100 O
O6
is’ @ 0° O
?@ Q
0
-—  o
Ave
->-
O
O
e M—W
O
O 
§e
O 1
9
ll
rage
U1
O
O
no
M'n
D'ffere
0
o
0
300 I.
' /o -100
200  O
200 300 400 500 600 200 300 400 500 600
Wnght Average Average M&W
Figure 8: PEFR: Average Plot, and Difference—Averag'e Plot.
Model # | Model
4 Same as 3 except equation (2) is used
3 See equation (1): og I 0
2 Same as 3 but <12 : 0§: same measurement variance
2' | Same as 3 but [30 : 0, 51 : 1: same readings on average
1 | Same as 3 but 0'? = 0;? and [30 = O, [31 = 1: identical devices
Table 1: Hierarchy of Models
For this reason, the on levels and P-values cited there need to be multiplied by 2, and
we have done so here.)
Next, we graphically examine how the two devices compare, using the means
for each subject and device. The main reasons for this plot are twofold: to look
for outliers (a different type than before) and to look for a non-linear relationship.
(Increases in variability with the means would normally be detected in the earlier
plots.) See Figure 8, where the outlier from subject 7 has been removed.
Neither plot suggests any smooth departures from linearity, but the plot on the
right suggests that two additional subjects (numbers 2 and 15) were measured differ-
ently on the two machines.
%============================================================================================================================================================= %
\newpage
The standard statistical approach to formally examine such data is based on
a series of structural models. The models we examine are shown in Table 1 and correspond to the numbered listing at the beginning of Section 2. The models form a hierarchy with the exception of 2 and 2’. To examine the
models, we start with the largest and search for the smallest model consistent with the
19



data. We use maximum likelihood to fit the models. The deviance (—2log-likelihood)
for Model 4 is 519.30, while for Model 3 it is 532.64. The difference of 13.34 is highly
significant (P I 0.0003) when using the standard X? reference distribution, although
the sample sizes are too small here to justify using this reference distribution with
high precision. The outliers in the right-hand plot of Figure 8 and the dramatic
increase in the estimate of measurement standard deviations in moving from Model 4 (65 I 44, 65 I 16,65 I 11) to Model 3 (65 I 0,6,; I 25, 65 I 19) also indicate that
Model 3 does not fit these data well.
One can only theorize why these two subjects were measured so differently for
these data. One would normally want to re-measure them to see if there are consistent differences in the devices or if these were anomalies. For purposes of this
example, we will assume the latter and set these two subjects aside as well. The deviances for Model 4 (442.99) and Model 3 (447.50) differ by 4.51, which yields
P I 0.033 with the X? reference distribution. Because likelihood methods generally
reject hypotheses at a higher than nominal rate for small sample sizes, there is not
strong evidence to reject Model 3 for Model 4. The increase in the estimate of mea-
surement standard deviations in moving from Model 4 (65 I 21, 65 I 17, 66 I 12) to
Model 3 (65 I 0, 65 I 22,65 I 14) was also less, and there was also no evidence of
other outliers in the right»hand plot of Figure 8.


The deviances for Model 2’ (449.6), Model 2 (448.52) , and Model 1 (451.17, for
which 6,; I 65 I 19) show that Model 1 is the simplest model consistent with the
data for the N I 14 subjects. Based on these results, it is reasonable to summarize
the analysis as follows:
1. The Mini meter measurement on one subject was a clear outlier. (Subjectis
data removed from analysis.)
2. Measurements for two subjects were inconsistent across devices. (May indicate
fundamental differences in devices for certain subjects or at certain times. In
this analysis, we set these subjects’ data aside.)
3. For the remaining N I 14 subjects, the data suggest the devices may function
identically, with a estimated measurement standard deviation of 19 liters/min.

%================================================= %
This means that the repeatability is estimated to be 54 liters/min.
The last number is based on the British Standard Institute’s (BSI’s) definition of
repeatability (1979), the value in which 95\% of differences in reading will lie.
Our analysis treats the two devices in a symmetric way and finds no evidence of
a difference. This contradicts the analysis of Bland (1995, p. 272) who performs an
analysis only on the first reading, and estimates a repeatability value of 78 liters / min.
He states that “on the basis of these data we would not conclude that the two methods
are comparable or that the mini-meter could reliably replace the Wright peak flow
meter.” In fact, after excluding the subject with the obvious outlier from the analysis,
20



the estimates of Within-device measurement error based solely on the duplicate set
of readings were 16 for the Wright meter and 12 for the Mini meter, less (but not
statistically so) than the Wright meter.
9 Example 4. Use of Regression
Bland and Altman (2003) stated that an appropriate use of regression exists for
the case in which a new method has different units from the old (but are, presum-
ably, linearly related). A regression of Y (old) on X (new), when sampling from
a bivariate normal distribution, estimates the regression line E [Y \X] = E [Y] +
pXy (0')//(TX) (X — E [X]), with Var (Y \X) : 0%, (1 — /1%”). Using the structural
equation model, this becomes
xi vi
flxy = 1? (19)
,/ ((7; + fig) ([11:72 + 172)
_ '31
51472
Ell/lXl I flO+fl1#x+ (X7#x>
5
= /3n+51H1+fi(X—H1)
5%‘
v - Y X : 2 be
m( I ) UY ( (oi +0§) (3%; +0?)
: 5%, 1 _  Z
(1 ‘L Rip) (51+ R§1Rg6)
As a special case, first consider when the two devices are in fact equivalent, which
is the same as comparing the device to itself. In this case
U2
Ell/lXl = H1+fi(X—H1)
Unless there is no measurement error in X , this produces a slope less than 1, as is
well known, and thus gives the regression to the mean phenomenon that Bland and
Altman have repeatedly criticized, including in the (2003) paper itself. (It is true
that this estimate of the next measurement made on the same device, for example,
is a better estimate than the reasonable (but naive) estimate X itself, under the
21



random-ac model that is used here. See Fuller (1987, p. 75) for additional comments.
However, we believe most researchers would be more interested in what we shall call
the unadjusted corrected value, which here would be X , and which in general would
be ,Z3O+E1X, Where the estimates do and K31 would be obtained through the structural-
equation model approach, based on the data from the multiple-measurement case.)
ln the general case, it is clear from (10) that the unadjusted corrected value for
Y will be correct only when X I ,uX (in practice, if the observed X is very close
to the observed mean of the X ’s), but otherwise suffers from the same regression-to-
the-inean phenomenon. As before, in the one-measurement-case, the estimate of fll
cannot be obtained.
Bland and Altman (2003) also recommend that 95\% prediction-interval limits be
used as a substitute for their standard 95\% limits of agreement. It is diflicult for us
to understand why they would make such a recommendation, because the two limits
have little in common, as we now show. The 95\% prediction-interval limit is meant to
capture one future value With a 95\% chance, Where the 95\% takes into account both
the estimate of variability of a future value Y, given X, and the estimated variability
of the regression coeflicients involved. These estimated limits are a strongly affected
by the sample size of the data used in the regression. The 95\% limits of agreement
take into account the the estimate of the variability of Y X, but do not take into
account any estimated variability of the estimates. Such estimated limits are not very
dependent on the sample size. Suppose We have N pairs of values, and call the 11”’
pair (X,;,Y,), with resulting least square equation of YX = 5/0 + '11 (X — X), where
X is the average of the X,-’s in the sample, and (yo/$1) are the usual least square
estimates. Let the estimated residual standard be denoted by sy,X Then the 95% P.I.
at a new value X is
1“/X i tN_2,@_g75\/VJEO) + (X _ X)2 vU(?1)+ 8€|X
~ 1 1
I YX i tN—2,O.9755Y\X W + (X * X)2   +1=
i/ _ X
where 5% = 2 (Xi — X)2 / (N — 1) is the sample variance of the X/s, and the de-
pendence of the width on the sample size is clear. Now, these actual values are based
on a particular sample. However, we can examine the properties of the width of this
interval by first replacing these estimates with the parameters they are estimating.
This produces a width of
22



2
5? "‘ KW 1 e  X
1 2 1
—+(X—n,) at
\/N (N—1Jv%(1+R§m)
a very complex function of the parameters. Consider only the simple case in which
fll = 1, as Bland and Altman did. The width reduces to
1
1 + R§@R?@\/1 "  X
I I 5
1 2 1
\/F+<X*"”) (N71)o-2v(1+R(%m) +1‘
For a sample of size N, the largest value of X in the data, the Nth order statis-
tic, can be approximated by M, + ZN/(N+1)0'X. From this, the smallest and largest
estimated widths are
/ 1
2tN_2?(]>g750'3~, 1 +  1 —   >< 
1 2 Imax 1
W + ZN/<»v+1>g + =
Where [max : 1 for the maximum width and Imax : O for the minimum width. The
95% width for limits of agreement when [31 : 1, on the other hand, is simply an
estimate of
2ZQ_Q75‘/ fig + 0'3.
We now consider three special cases of (11). First assume that 0'5 = 0, but that
rt, and as are fixed. Then (11) reduces to
1 I
2i1v,2,0.97s0q I — + Z2 i +17
N N/(N+1)(N_1)
a standard regression result. Second, assume that 06 = 0, but that J, and U5 are
fixed. Then (11) reduces to
23



1
2151vI2,0.975¢T6 I >< (12)
1 + R31
1 I
(1 — +Ziv/<iv+1>Ljx + 1~
N (N 1)
Third, assume that as I <75. We get
1
2tN—2,0.9759'e( /1 + W >< (13)
II:
/ 1 2 rm,
W J“ ZN/(N+1)m ‘L 1'
To examine these, consider a variety of ways in which graphs of Y vs. X might
appear. The unitless measure of this association is /2Xy, and We consider /2Xy values
of (0.5, 0.7, 0.9,0.95) for our comparisons. We select (05, 05) values for three cases
to be (10, 0), (O, 10), and (10, 10), respectively, to create the graphs—this choice of
scale has no effect on the relative comparisons. The flxy values for each scenario let
us determine the corresponding ax value based on (10) with fll I 1.
Rather than use these unusual intervals, the user could consider Y :1: z9_975sy|X.
This leads to the following estimated interval widths, which we call regression zmldths.
For the 0,; I 0 case, the width estimates 2z9_975a6, which is identical to the agree-
ment Width, and independent of pxy. For the 0'5 I O case, the width estimate
2z0_975a,;, / 1/ (1 + Ric) I 2z@_975a,;pXY, which will be slightly smaller than the 2z0_975o5
agreement width, and especially so for smaller values of pxy. (The P.I. approach
makes this slightly larger for smaller samples sizes, but not because of any good theo-
retical reasons.) For the 05 I 06 case. the width estimate 2\/iZ[)_975O'51/1 — Rgm/2 (1 + R6 ) I
2\/52097505, / (1 + p§(y) /2p§fl,., which is again slightly less than the 2\/52097505 dif»
ference width.
In fact, we can do better than this—from the data, We can estimate pxy and this
2 2
in turn lets us find an upper bound for the width. Because /)Xy < q / (1 + pxy) /2/)XY
for 0 < pxy < 1, this upper bound is Y i 20975 (5;/(X /pxy). (Note: it appears that
this upper bound would always based on the 05 I 0 case, but but we have not
proven this.) Call this Width the regression max width. From this, we compare the
BA difference widths (which Bland and Altman are trying to estimate) to the BA
prediction interval Widths (minimum and maximum), the regression Width, and the
regression max width. See Figures 9 to 11. From these, it is clear that the regression
24



l rho_ l rho: 0-95 _
'70
.. _____.___ _ 40
_ mo: 0-5 l _ 11-7 1
BA Pl min/max
i Regr & BA Diff
70 ’ Regr max
40 ' "
H-i “l“\ ~~r—
aaqo 2 :4 4 5 57810e910 z a 4 5 578100
N
Figure 9: Interval Widths, Four Methods, vs p and N, when 05 = O.
and regression max widths estimate reasonable ranges for the BA difference Widths,
while the BA prediction interval widths do not.
10 Recommendations
The use of the BA methodology is certainly to be preferred to the other methods that
Bland and Altman have criticized. However, We are very concerned on the emphasis
they place on the one-measurement cases in their examples. This, we believe, inad-
vertently suggests to researchers that such a methodology is acceptable. I11 Altmanls
(1995) generally excellent textbook, for example, he stated “For simplicity, I shall
use only one measurement by each method here. We could make use of the duplicate
data by using the average of each pair first, but this introduces an extra stage in the
calculation. Bland and Altman (1986) give details.”
It is true that Bland and Altman (1986) give formulas for the multiple-measurement
case, near the end of that article. Ironically though, even when they have two readings
available per subje<:t/ device, they still only use first to illustrate the BA plot this
may be simpler to explain, perhaps, but it is not an appropriate way to analyze these
data. We believe in simple analysis, but not at the expense of properly using the data
25



I "M2 \ m<>= 0-95 _
BA F'l min/max _ 20
i BA Diff, Regv max
Regr
— mm 0-5 \ — 01 T
40'
20* ‘ 7 ' ” 7‘
‘_'_v ‘ ‘ ““-'|~\ \ \ -
sew 2 3 A 5 6781(]g91|_) 2 a 4 5 678199
N
Figure 10: Interval Widths, Four Methods, vs ,0 and N, when 0'5 =
26



\ mo=_ \ mo: 0-95 _
'90
'50
— rho: 0.5 \ 0.1 I
BA Pl min/max
Z BA om
Regr, Regr max
90'
so — _ _ _ Q: Ii  ’ ’ ’ ’ ’ ’ ’ ’ 4:: 1': 
H-\ ‘  ‘ 
3210 2 3 45578109910 2 3 45512100
N
Figure 11: Interval Widths, Four Methods, vs p and N, When 0'5 = 0,;
27



for modeling.
Our recommendations are simple.
1. If physically possible, repeated measurements on each device for each subject
should be required.
2. The extended structural equation model should be used for analysis. Plots,
including sample standard deviations versus means (or, in the case of two-
measurement studies, second measurement versus first measurement) for each
device, Y vs. X, and the BA plot, should be used along with formal tests to
check for model adequacy—this includes outliers, nonlinear relationships, and
non-constancy of variance.
3. If the model is adequate, provide estimates and confidence intervals of the pa-
rameters and other related quantities. The latter may include estimates that
are related to local standards, such as those of BS1.
We will illustrate the use of extended structural equation models, including model
checking, in a separate article. It is our intent to do this is such a way to allow these
methods to become easily accessible to researchers.
Appendix A- (<T1.51.R§1) I f (E. R65)
Based on (6), we want to find (01,, [31, R5,”) = f (E, R55). We will show it here in terms
of population parameters. For a sample of data, simply replace E with its estimate
S.
Define the ratios R1 = (2212 + E11) / (4222 — E11) ,Rg = (2212 — E11) / (4222 — E11)
where E,-j is the (i,j)m element of E. Next, define I2 I R3; — 1 —2R2R§5 — 2R1. Then,
a lengthy but straightforward calculation shows that
5, : <—b+,/b2+4R§6>/2
Rt» = —m<m-1—2R1)/R56
"Z = (4E22—311)/(451)
From this, we can calculate related quantities of interest. For example, erg =
2 2 2 _ 2 2 2
UzR5w and Us i (7a:R6wRe6'
With the information on fil, we can also calculate /it : 2 (/12 — pl) / 2 and then
50 = H1 — (/51 *1)/Jr
References
Altman, D. G. and Bland, J. l\/I. (1983), “Measurement in medicine: the analysis
of method comparison studies,” The Statistician, 32, 307—317.
28



Bland, J. M. (1995), An Introduction to Medical Statistics, Oxford Medical
Publications, New York.
Bland, J. M., and Altman, D. G. (1986), “Statistical methods for assessing
agreement between two methods of clinical measurement," Lancet i, 307-310.
Bland, J. M.. and Altman, D. G. (1995), “Comparing methods of measurement:
why plotting difference against standard method is misleading,” Lancet, 346,
1085—1087.
Bland J. M., and Altman D. G. (1999) “Measuring agreement in method com-
parison studies.” Statistical Methods in Medical Research 8, 135—160.
Bland J. M., and Altman D. G. (2003), “Applying the right statistics: analysis
of rneasurenient studies,” Ultrasound Obstet Gynecol, 22, 85 93.
Bollen, K. A. (1989), Structural Equations with Latent Variables, Wiley, New
York.
British Standards Institution (1979), Precision of Test Methods, Part 1: Guide
for the Determination of Repeatability and Reproducibility for a Standard Test
Method. BS 5497, Part 1. London.
Fuller, W. (1987), Measurement Error Models, Wiley, New York.
Grubbs, F. (1969), “Procedures for Detecting Outlying Observations in Sam-
ples,” Technometrics, 11, 1—21.
Lindley, D. V. (1947), “Regression Lines and the Linear Functional Relation-
ship,” Journal of the Royal Statistical Society Supplement, 9, 218 244.
Mandel, J. (1984), “Fitting Straight Lines When Both Variables are Subject to
Error," Journal of Quality Technology, 16, 1-14.
Maloney, C. J., and Rastogi, S. C. (1970), “Significance Test for Grubbs”s Es-
timators,” Biometrics 26, 671-676.
Morgan, W. A. (1939), “A test of significance of the differences between two
variances in a sample from a normal bivariate population,” Biometrika, 13—19.
Pitman, E. J. G. (1939), “A note on normal correlation,” Biometrika, 31, 9—12.
Reiersol, O. (1950), “Identifiability of a Linear Relation between Variables
Which Are Subject to Error,” Econometrica, 18, 375—389.
Spiegelman, C. (1979), “On Estimating the Slope of a Straight Line when Both
Variables are Subject to Error,” The Annals of Statistics, 7, 201-206.
29



\end{document}

