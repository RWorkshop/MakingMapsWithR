\documentclass[MASTER.tex]{subfiles}
\begin{document}
\subsection{Prevalence of the Bland-Altman plot}
	\citet*{BA86}, which further develops the Bland-Altman methodology,
	was found to be the sixth most cited paper of all time by the
	\citet{BAcite}. \cite{Dewitte} describes the rate at which
	prevalence of the Bland-Altman plot has developed in scientific
	literature. \citet{Dewitte} reviewed the use of Bland-Altman plots
	by examining all articles in the journal `Clinical Chemistry'
	between 1995 and 2001. This study concluded that use of the
	Bland–Altman plot increased over the years, from 8\% in 1995 to
	14\% in 1996, and 31–36\% in 2002.
	
	The Bland-Altman Plot has since become expected, and
	often obligatory, approach for presenting method comparison
	studies in many scientific journals \citep{hollis}. Furthermore
	\citet{BritHypSoc} recommend its use in papers pertaining to
	method comparison studies for the journal of the British
	Hypertension Society.
	
\section{Bland Altman Plots In Literature}
\citet{mantha} contains a study the use of Bland Altman plots of
44 articles in several named journals over a two year period. 42
articles used Bland Altman's limits of agreement, wit the other
two used correlation and regression analyses. \citet{mantha}
remarks that 3 papers, from 42 mention predefined maximum width
for limits of agreement which would not impair medical care.

The conclusion of \citet{mantha} is that there are several
inadequacies and inconsistencies in the reporting of results ,and
that more standardization in the use of Bland Altman plots is
required. The authors recommend the prior determination of limits
of agreement before the study is carried out. This contention is
endorsed by \citet{lin}, which makes a similar recommendation for
the sample size, noting that\emph{sample sizes required either was
not mentioned or no rationale for its choice was given}.

\begin{quote}
In order to avoid the appearance of "data dredging", both the
sample size and the (limits of agreement) should be specified and
justified before the actual conduct of the trial. \citep{lin}
\end{quote}

\citet{Dewitte} remarks that the limits of agreement should be
compared to a clinically acceptable difference in measurements.
%%%%%%%%%%%%%%%%%%%%%%%%%%%%%%%%%%%%%%%%%%%%%%%%%%%%%%%%%%%%%%%%%%%%%%%%%

\end{document}