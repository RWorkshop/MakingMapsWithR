
In this section, we will reconsider the conversion problem, where by the methods of measurements are denominated in different units.
Conversion problems arise when the comparison is between two 
approximate methods of measurement each of which measures the quantity in different units.

This situation can arise when the methods in question proceed by measuring different proxies for the underlying 
quantity of interest. (lewis 1991)

For the single measurement case, the author can not foresee any scope for insights that are not already offered by using a structural relation model, as proposed by lewis et 1991, or error-in-variables regression. 
In the case of orthonormal regression, it is not reasonable to assume that both methods have equal measurement variance, when they are denominated in different units.
The analyst may attempt to mitigate the problem by scaling the variance of one method, but even still problems remain.
Similarly for Deming regression, no further insights on how to properly estimate the variance ratio can be offered.

For the case of conversion problem with replicate measurements, a framework that incorporates the ideas offered by Roy (2009) can be proposed. Estimates for between-subject and within-subject variances may be sought.
However Roy's tests on variability are no longer applicable, as one would not expect the method to have similar estimates. An estimate for the scaling factor $\beta$ may be sought, where $Y_i \approx \beta X$.


\[ X_i = \tau_i + \delta_i \]
\[ Y_i = \alpha + \beta X \tau_i + \epsilon_i\]


We will simulate a data set based in lewis conversion problems, provide three replicates values for both measurements. To acheive this we add ``jitter noise" to three copies of each original measurement.
