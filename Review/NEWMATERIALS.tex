The  question being answered is not always clear, but is usually epxressed as an attempt to quantify the agreement
between two methods (Bland and Altman 1995)

Some lack of agreement between different methods of measurement is inevitable. What matters is the amount by which they
disagree. we want to know by how much the new method is likely to differ from the old, so that it is not enough to cause
problems in the mathematical interpretation we can preplace the old method by the new, or even use the two interchangably.


It often happens that the same physical and chemical property can be measured in different ways. For example, one can determine
For example, one can determine sodium in serum by flame atomic emission spectroscopy or by isotops dilution mass spectroscopy. The question arises as to whcih methd is better (Mandel 1991)

In areas of inter-laboratory quality control, method comparisons, assay validations and individual bio-equivalence, etc, the agree between observations and target (reference) value is
of interest (lin 2002)

The purpose of comparing two methods of measurement of a continuous biological variable is to uncover systematic differences, not to point to
similarities. (ludbrook 1997)

In the pharmaceutical industry, measurement methods that measure the quantity of prdocuts are regulated. The FDA (U.S. Food and
Drug Administration) requires that the manufacturer show equivalency prior to approving the new or alternatice method in quality control (Tan \& Inglewicz ,1999)
