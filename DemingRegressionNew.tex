\documentclass[Chap2main.tex]{subfiles}

% Load any packages needed for this document

\begin{document}
%-----------------------------------------%

\section{Regression Techniques for MCS}

The \textbf{\textit{mcr}} packages provides a set of regression techniques to quantify the relation between two measurement methods.

In particular, it address regression problems with errors in both variables, but without repeated measurements.
The \textbf{\textit{mcr}} package follows the CLSI EP09-A3 recommendations for analytical
method comparison and estimation of bias using patient samples.


\textit{Methods featured in the \textbf{mcr} package}

\begin{itemize}
\item Deming Regression
\item Weighted Deming Regression
\item Passing-Bablock Regression
\end{itemize}

The \textit{creatinine} gives the blood and serum preoperative creatinine measurements in 110 heart surgery patients.

\begin{framed}
\begin{verbatim}
library("mcr")
data("creatinine", package="mcr")
tail(creatinine)


fit.lr <- mcreg(as.matrix(creatinine), method.reg="LinReg", na.rm=TRUE)
fit.wlr <- mcreg(as.matrix(creatinine), method.reg="WLinReg", na.rm=TRUE)
compareFit( fit.lr, fit.wlr )
\end{verbatim}
\end{framed}


\subsection{Bootstap Techniques}
Use of Bootstap Techniques to obtain Confidence Interval estimates

%----------------------------------------%
\newpage
\section{Bayesian Approaches}

%----------------------------------------%
\newpage
\section{References}
Carpenter, J., Bithell, J. (2000) Bootstrap confidence intervals: when, which, what? A practical
guide for medical statisticians. Stat Med, 19 (9), 1141–1164.

\end{document}


\ewpage
\begin{itemize}
\item Model I regression
\item Model II regression
\end{itemize}
Model I regression [Criterion v Test]
[Cornbleet Gochman 1979] define this analysis as the case in which the independent variable, X, is measured without error, with y as the dependent variable.
 
In method comparison studies, the X variable is a precisely measured reference method. In the [Cornbleet Gochman 1979] paper. It is argued that criterion may be regarded as the correct value. Other papers dispute this.
 
 
Model II regression [Test V Test]
In this type of analysis,both of the measurement methods are test methods, with both expected to be subject to error. Deming regression is an approach to model II regression.

\subsection{Deming Regression}

 \begin{itemize}

\item The 95\% confidence interval for the Intercept can be used to test the hypothesis that A=0. This hypothesis is accepted if the confidence interval for A contains the value 0. 
\item If the hypothesis is rejected, then it is concluded that A is significant different from 0 and both methods differ at least by a constant amount. 

\item The 95\% confidence interval for the Slope can be used to test the hypothesis that B=1. This hypothesis is accepted if the confidence interval for B contains the value 1. 

\item If the hypothesis is rejected, then it is concluded that B is significant different from 1 and there is at least a proportional difference between the two methods. Cochrane Cornbleet
\item The authors make the distinction between model I and model II regression types.
\item Model II regression is the appropriate type when the predictor variable “x” is measured with imprecision.
Cornbleet and Cochrane remark that clinical laboratory measurements usually increase in absolute imprecision when larger values are measured. % - [**]
 \end{itemize}

\subsubsection*{Guidelines}
Always plot the data. Suspected outliers may be identified from the scatter plot.
$S_{ex}$  represents the precision of a single x measurement near the mean value of X
\[\lambda = frac{S^2_{ex}}{S^2_{ey}}\]

\end{document}

