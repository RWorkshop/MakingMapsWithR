\documentclass[Chap2cmain.tex]{subfiles}
\begin{document}


%------------------------------------------------------------------------------%
\subsection{Relevance of Repeatability} 
Repeatability of two method limit the amount of agreement that is possible.\\
If one method has poor repeatability, the agreement is bound to be poor. If both methods have poor repeatability, agreement is
even worse.


%----------------------------------------------------------------------------------------%
\section{Repeatability}
\subsection{Repeatability and gold standards}
Currently the phrase `gold standard' describes the most accurate method of measurement available. No other criteria are set out. Further to \citet{dunnSEME}, various gold standards have a varying levels of repeatability. Dunn cites the example of the sphygmomanometer, which is prone to measurement error. Consequently it can be said that a measurement method can be the `gold standard', yet have poor repeatability. Some authors, such as [cite] and [cite] have recognized this problem. Hence, if the most accurate method is considered to have poor repeatability, it is referred to as a 'bronze standard'.  Again, no formal definition of a 'bronze standard' exists.

The coefficient of repeatability may provide the basis of formulation a formal definition of a `gold standard'. For example, by determining the ratio of $CR$ to the sample mean $\bar{X}$. Further to [Lin], it is preferable to have a sample size specified in advance. A gold standard may be defined as the method with the lowest value of $\lambda = CR /\bar{X}$ with $\lambda < 0.1\%$. Similarly, a silver standard may be defined as the method with the lowest value of $\lambda $ with $0.1\% \leq \lambda < 1\%$. Such thresholds are solely for expository purposes.


%----------------------------------------------------------------------------------------%

\chapter{Ceofficient of Repeatability}
%Definition of Repeatability


\section{Add Ins}
importance of repeatability
'cursiously replicate measurements are rarely made in method comparison studies, so that an important aspect of comparability is 
often overlooked.

lack of repaeatability can interfere with the comparsion of two methods because if one methods has poor repeatability, in the sense that there is
considerable variation in repeated measurements on the same subject, the agreement between two methods is bound to be porr.



%------------------------------------------------------------------------------------------%
\section{Bland and Altman}
\begin{itemize}
	\item Two readings by the same method will be within $1.96
	\sqrt{2} \sigma_w $ or $2.77 \sigma_w $ for 95\% of subjects. Thisvalue is called the repeatability coefficient.
	
	\item For observer J using the sphygmomanometer $ \sigma_w = \sqrt{37.408} = 6.116$ and so the repeatability coefficient is
	$2:77 \times 6.116 = 16:95$ mmHg.
	
	\item For the machine S,$ \sigma_w = \sqrt{83.141} = 9.118$ and the repeatability coefficient is $2:77 \times 9.118 = 25.27$ mmHg.
	
	\item Thus, the repeatability of the machine is 50\% greater than that of the observer.
\end{itemize}
%-------------------------------------------------------------------%
\section{Carstensen}
\begin{itemize}
	\item The limits of agreement are not always the only issue of
	interest — the assessment of method specific repeatability and
	reproducibility are of interest in their own right.
	
	\item Repeatability can only be assessed when replicate
	measurements by each method are available.
	
	\item The repeatability coefficient for a method is defined as the
	upper limits of a prediction interval for the absolute difference
	between two measurements by the same method on the same item under
	identical circumstances.
	
	\item If the standard deviation of a measurement is $\sigma$ the
	repeatability coefficient is $2\times\sqrt{2} \sigma = 2.83\times
	\sigma \approx 2.8 \sigma$.
	
	
	\item The repeatability of measurement methods is calculated
	differently under the two models \item Under the model assuming
	exchangeable replicates (1), the repeatability is based only on
	the residual standard deviation, i.e. $2.8\sigma_m$
	
	
	\item Under the model for linked replicates (2) there are two
	possibilities depending on the circumstances.
	
	\item If the variation between replicates within item can be
	considered a part of the repeatability it will be $2.8 \sqrt{
		\omega^2 + \sigma^2_m}$.
	
	\item However, if replicates are taken under substantially
	different circumstances, the variance component $\omega^2$ may be
	considered irrelevant in the repeatability and one would therefore
	base the repeatability on the measurement errors alone, i.e. use
	$2.8 \sigma_m$.
	\end{itemize}
	
	%------------------------------------------------------------------------------%
	\subsection{Repeatability}
	
	
	
	\citet{BA99} strongly recommends the simultaneous estimation of
	repeatability and agreement be collecting replicated data.
	\citet{ARoy2009} notes the lack of convenience in such
	calculations.
	
	
	If one method has poor repeatability in the sense of considerable
	variability, then agreement between two methods is bound to be
	poor \citep{ARoy2009}.
	
	It is important to report repeatability when assessing
	measurement, because it measures the purest form of random error
	not influenced by other factors \citep{Barnhart}.
	
\section{Reproducibility}
 
It is advisable to be able to distinguish between Repeatability and a similar concept ‘Reproducibility’. Reproducibility is
 
\subsection{2 The Coefficient of Repeatability}
Since for the repeated measurements the same method is used, the mean difference should be zero.
 
Therefore the Coefficient of Repeatability (CR) can be calculated as 1.96 (or 2) times the standard deviations of the differences between the two measurements (d2 and d1):
WRONG

\newpage
\section{Repeatability}

The quality of repeatability is the ability of a measurement method to give consistent results for a particular subject. That is to say that a measurement will agree with prior and subsequent measurements of the same subject.

Repeatability is defined by the \citet{IUPAC} as `the closeness of agreement between independent results obtained with the same method on identical test material, under the same conditions (same
operator, same apparatus, same laboratory and after short intervals of time)'  and is determined by taking multiple measurements on a series of subjects.

Repeatability is important in the context of method comparison because the repeatability of two methods influence the amount of agreement which is possible between those methods. If one method have poor repeatability, then agreement with that method and another will necessarily be poor also.
\citet{barnhart} and \citet{roy} highlight the importance of reporting repeatability in method comparison, because it measures the purest random error not influenced by any external factors. Statistical procedures on within-subject variances of two methods are equivalent to tests on their respective repeatability coefficients. A formal test is introduced by \citet{roy}, which will discussed in due course.



%------------------------------------------------------------------------------%
\section{Repeatability}
A measurement method can be said to have a good level of repeatability if there is consistency in repeated measurements on
the same subject using that method. Conversely, a method has poor
repeatability if there is considerable variation in repeated measurements.


This is relevant to method comparison studies because the 'repeatabilities' of the two methods of measurement affects the
level of agreement of those methods.Poor repeatability in one method would result in poor agreement. More so if there is poor
repeatability in both methods.

The British standards Insitute[1979] define a coefficient of repeatability  as \emph{the value below which the difference
between two single test results..may be expected to lie within a specified probability.}In the absence of
other indications, the probability is 95\%.

\subsection{Repeatability}
Repeatability (or \textit{test-retest reliability})  describes the variation in measurements taken by a single method of measurement on the same item and under the same conditions. 
A less-than-perfect test-retest reliability causes test-retest variability. Such variability can be caused by, for example, intra-individual variability and intra-observer variability. 
A measurement may be said to be repeatable when this variation is smaller than some agreed limit.
Test-retest variability is practically used, for example, in medical monitoring of conditions. In these situations, there is often a predetermined "critical difference", and for differences in monitored values that are smaller than this critical difference, the possibility of pre-test variability as a sole cause of the difference may be considered in addition to, for examples, changes in diseases or treatments.

According to the Guidelines for Evaluating and Expressing the Uncertainty of NIST Measurement Results, the following conditions need to be fulfilled in the establishment of repeatability:
\begin{itemize}
	\item	the same measurement procedure
	\item	the same observer
	\item	the same measuring instrument, used under the same conditions
	\item	the same location
	\item	repetition over a short period of time.
\end{itemize}
\newpage
%--------------------------------------------------------------------%
\subsection{Bland and Altman 1999}
As noted by Bland and Altman 1999, the repeatability of two methods of measurement can  potentially limit
Repeatability (using Bland-Altman plot)
The Bland-Altman plot may also be used to assess a method’s repeatability by comparing repeated measurements using one single measurement method on a sample of items.
The plot can then also be used to check whether the variability or precision of a method is related to the size of the characteristic being measured.
Since for the repeated measurements the same method is used, the mean difference should be zero.
Therefore the Coefficient of Repeatability (CR) can be calculated as 1.96 (often rounded to 2) times the standard deviation of the case-wise differences.
%--------------------------------------------------------------------%

%--------------------------------------------------------------------%
\subsection{Notes from BXC Book (chapter 9)}
The assessment of method-specific repeatability and reproducibility is of interest in its own right.
Repeatability and reproducibility can only be assessed when replicate measurements by each method are available.
If replicate measurements by a method are available, it is simple to estimate the measurement error for a method, using a model with fixed effects for item, then taking the residual standard deviation as measurement error standard deviation.
However, if replicates are linked, this may produce an estimate that biased upwards.
The repeatability coefficient (or simply repeatability) for a method is defined as the upper limit of a
prediction interval for the absolute difference between two measurements by the same method on the same
item under identical circumstances (see above conditions)

\[y_{mir}  = \alpha_{m} + \beta_m( \mu_i + a_{ir} + c_{mi}) + e_{mir}\]

The variation between measurements under identical circumstances.


\bibliography{DB-txfrbib}
\end{document}
