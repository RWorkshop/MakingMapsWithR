\documentclass[12pt, a4paper]{article}
\usepackage{epsfig}
\usepackage{subfigure}
%\usepackage{amscd}
\usepackage{amssymb}
\usepackage{amsbsy}
\usepackage{amsthm}
%\usepackage[dvips]{graphicx}
\usepackage{natbib}
\bibliographystyle{chicago}
\usepackage{vmargin}
% left top textwidth textheight headheight
% headsep footheight footskip
\setmargins{3.0cm}{2.5cm}{15.5 cm}{22cm}{0.5cm}{0cm}{1cm}{1cm}
\renewcommand{\baselinestretch}{1.5}
\pagenumbering{arabic}
\theoremstyle{plain}
\newtheorem{theorem}{Theorem}[section]
\newtheorem{corollary}[theorem]{Corollary}
\newtheorem{ill}[theorem]{Example}
\newtheorem{lemma}[theorem]{Lemma}
\newtheorem{proposition}[theorem]{Proposition}
\newtheorem{conjecture}[theorem]{Conjecture}
\newtheorem{axiom}{Axiom}
\theoremstyle{definition}
\newtheorem{definition}{Definition}[section]
\newtheorem{notation}{Notation}
\theoremstyle{remark}
\newtheorem{remark}{Remark}[section]
\newtheorem{example}{Example}[section]
\renewcommand{\thenotation}{}
\renewcommand{\thetable}{\thesection.\arabic{table}}
\renewcommand{\thefigure}{\thesection.\arabic{figure}}
\title{Research notes: linear mixed effects models}
\author{ } \date{ }

\begin{document}
\author{Kevin O'Brien}
\title{LME Estimation and Algorithms}

\tableofcontents \setcounter{tocdepth}{2}

%------------------------------------------------%
\newpage

\section{Algorithms}
Maximum likelihood estimation is a method of obtaining estimates
of unknown parameters by optimizing a likelihood function. The ML
parameter estimates are the values of the argument that maximise
the likelihood function, i.e. the estimates that make the observed
values of the dependent variable most likely, given the
distributional assumptions

The most common iterative algorithms used for the optimization
problem in the context of LMEs are the EM algoritm, fisher scoring
algorithm and NR algorithm, which [cite:West] commends as the
preferred method.

A mixed model is an extension of the general linear models that
can specify additional random effects terms.

Parameter of the mixed model can be estimated using either ML or
REML, while the AIC and the BIC can be used as measures of
"goodness of fit" for particular models, where smaller values are
considered preferable.

%--------------------------------------------------------------------%

\newpage

\subsection{ML v REML}
(\textbf{\emph{Wikipedia}})The restricted (or residual, or reduced) maximum likelihood (REML) approach is a particular form of maximum likelihood estimation which does not base estimates on a maximum likelihood fit of all the information, but instead uses a likelihood function calculated from a transformed set of data, so that nuisance parameters have no effect.

In contrast to the earlier maximum likelihood estimation, REML can produce unbiased estimates of variance and covariance parameters.

%-----------------------------------------------------------------------------------------%
\newpage
\section{ML procedures for LME}

The maximum likelihood procedure of Hartley and Rao yields
simultaneous estimates for both the fixed effects and the random
effect, by maximising the likelihood of $\boldsymbol{y}$ with
respect to each element of $\boldsymbol{\beta}$ and
$\boldsymbol{b}$.

%-----------------------------------------------------------------------------------------%
\newpage
\section{Estimation of random effects}

Estimation of random effects for LME models in the NLME package is accomplished through use
of both EM (Expectation-Maximization) algorithms and Newton-Raphson algorithms.
\begin{itemize}
\item EM iterations bring estimates of the parameters into the region of the optimum very quickly, but
convergence to the optimum is slow when near the optimum.
\item Newton-Raphson iterations are computationally intensive and can be unstable when far from the
optimum. However, close to the optimum they converge quickly.
\item The LME function implements a hybrid approach, using 25 EM iterations to quickly get near the
optimum, then switching to Newton-Raphson iterations to quickly converge to the optimum. \item If
convergence problems occur, the ``control� argument in LME can be used to change the way the
model arrives at the optimum.
\end{itemize}


%------------------------------------------------%
\newpage
\addcontentsline{toc}{section}{Bibliography}
\bibliography{2012bib}
\end{document} 