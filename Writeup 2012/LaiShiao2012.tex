\documentclass[12pt, a4paper]{article}
\usepackage{natbib}
\usepackage{vmargin}
\usepackage{epsfig}
\usepackage{subfigure}
%\usepackage{amscd}
\usepackage{amssymb}
\usepackage{amsbsy}
\usepackage{amsthm}
%\usepackage[dvips]{graphicx}
\bibliographystyle{chicago}
\renewcommand{\baselinestretch}{1.2}

% left top textwidth textheight headheight % headsep footheight footskip
\setmargins{3.0cm}{2.5cm}{15.5 cm}{23.5cm}{0.5cm}{0cm}{1cm}{1cm}

\pagenumbering{arabic}

\begin{document}
\author{Kevin O'Brien}
\title{Transfer Report - LME models in MCS}
\date{\today}
\maketitle
\tableofcontents \setcounter{tocdepth}{2}

\section{LaiShiao}


\citet{LaiShiao} advocates the use of LME models to study method comparison problems. The authors analyse a data set typical of method comparison studies using SAS software, with particular use of the \emph{`Proc Mixed'} package. The stated goal of this study is to determine which factor from a specified group of factors is the key contributor to the difference in the two methods.

The study relates to oxygen saturation, the most investigated variable in clinical nursing studies \citep{LaiShiao}. The two method compared are functional saturation (SO2, percent functional oxy-hemoglobin) and fractional saturation (HbO2, percent fractional oxy-hemoglobin), which is considered to be the `gold standard' method of measurement.

\citet{LaiShiao} establishes an LME model for analysing the differences $D_{ijtl}$, where $D_{ijtl}$ is the differences of the measurements (i.e = $SO2_{ijtl}$ - $HbO2_{ijtl}$) for the ith donor at the $j$th level of foetal haemoglobin percent (Fhbperct) and the $t$th repeated measurement by the $l$th practitioner of the experiment.


(\citet{BXC2004} also advocates the use of LME models in comparing methods, but with a different emphasis.)





\addcontentsline{toc}{section}{Bibliography}

\bibliography{2012bib}
\end{document}
