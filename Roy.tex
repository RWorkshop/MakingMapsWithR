\section{Introduction}

\citet{roy} uses an approach based on linear mixed effects (LME) models for the purpose of comparing the agreement between two methods of measurement, where replicate measurements on items (often individuals) by both methods are available. She provides three tests of hypothesis appropriate for evaluating the agreement between the two methods of measurement under this sampling scheme. These tests consider null hypotheses that assume: absence of inter-method bias; equality of between-subject variabilities of the two methods; equality of within-subject variabilities of the two methods. By inter-method bias we mean that a systematic difference exists between observations recorded by the two methods. Differences in between-subject variabilities of the two methods arise when one method is yielding average response levels for individuals than are more variable than the average response levels for the same sample of individuals taken by the other method.  Differences in within-subject variabilities of the two methods arise when one method is yielding responses for an individual than are more variable than the responses for this same individual taken by the other method. The two methods of measurement can be considered to agree, and subsequently can be used interchangeably, if all three null hypotheses are true.

\bigskip

Let $y_{mir} $ denote the $r$th replicate measurement on the $i$th item by the $m$th method, where $m=1,2,$ $i=1,\ldots,N,$ and $r = 1,\ldots,n_i.$ When the design is balanced and there is no ambiguity we can set $n_i=n.$ The LME model underpinning Roy's approach can be written
\begin{equation}\label{Roy-model}
y_{mir} = \beta_{0} + \beta_{m} + b_{mi} + \epsilon_{mir}.
\end{equation}
Here $\beta_0$ and $\beta_m$ are fixed-effect terms representing, respectively, a model intercept and an overall effect for method $m.$ The model can be reparameterized by gathering the $\beta$ terms together into (fixed effect) intercept terms $\alpha_m=\beta_0+\beta_m.$ The $b_{1i}$ and $b_{2i}$ terms are correlated random effect parameters having $\mathrm{E}(b_{mi})=0$ with $\mathrm{Var}(b_{mi})=g^2_m$ and $\mathrm{Cov}(b_{1i}, b_{2 i})=g_{12}.$ The random error term for each response is denoted $\epsilon_{mir}$ having $\mathrm{E}(\epsilon_{mir})=0$, $\mathrm{Var}(\epsilon_{mir})=\sigma^2_m$, $\mathrm{Cov}(\epsilon_{1ir}, \epsilon_{2 ir})=\sigma_{12}$, $\mathrm{Cov}(\epsilon_{mir}, \epsilon_{mir^\prime})= 0$ and $\mathrm{Cov}(\epsilon_{1ir}, \epsilon_{2 ir^\prime})= 0.$ Two methods of measurement are in complete agreement if the null hypotheses $\mathrm{H}_1\colon \alpha_1 = \alpha_2$ and $\mathrm{H}_2\colon \sigma^2_1 = \sigma^2_2 $ and $\mathrm{H}_3\colon g^2_1= g^2_2$ hold simultaneously. \citet{roy} uses a Bonferroni correction to control the familywise error rate for tests of $\{\mathrm{H}_1, \mathrm{H}_2, \mathrm{H}_3\}$ and account for difficulties arising due to multiple testing. Roy also integrates $\mathrm{H}_2$ and $\mathrm{H}_3$ into a single testable hypothesis $\mathrm{H}_4\colon \omega^2_1=\omega^2_2,$ where $\omega^2_m = \sigma^2_m + g^2_m$ represent the overall variability of method $m.$  Disagreement in overall variability may be caused by different between-item variabilities, by different within-item variabilities, or by both.  If the exact cause of disagreement between the two methods is not of interest, then the overall variability test $H_4$ is an alternative to testing $H_2$ and $H_3$ separately.

\bigskip

\citet{BXC2008} also use a LME model for the purpose of comparing two methods of measurement where replicate measurements are available on each item. Their interest lies in generalizing the popular limits-of-agreement (LOA) methodology advocated by \citet{BA86} to take proper cognizance of the replicate measurements. \citet{BXC2008} demonstrate statistical flaws with two approaches proposed by \citet{BA99} for the purpose of calculating the variance of the inter-method bias when replicate measurements are available. Instead, \citet{BXC2008} use a fitted mixed effects model to obtain appropriate estimates for the variance of the inter-method bias.  As their interest mainly lies in extending the Bland-Altman methodology, other formal tests are not considered.

\bigskip

\citet{BXC2008} develop their model from a standard two-way analysis of variance model, reformulated for the case of replicate measurements, with random effects terms specified as appropriate. 
Their model describing $y_{mir} $, again the $r$th replicate measurement on the $i$th item by the $m$th method ($m=1,2,$ $i=1,\ldots,N,$ and $r = 1,\ldots,n$), can be written as
\begin{equation}\label{BXC-model}
y_{mir}  = \alpha_{m} + \mu_{i} + a_{ir} + c_{mi} + \epsilon_{mir}.
\end{equation}
The fixed effects $\alpha_{m}$ and $\mu_{i}$  represent the intercept for method $m$ and the `true value' for item $i$ respectively. The random-effect terms comprise an item-by-replicate interaction term $a_{ir} \sim \mathcal{N}(0,\varsigma^{2})$, a method-by-item interaction term $c_{mi} \sim \mathcal{N}(0,\tau^{2}_{m}),$ and model error terms $\varepsilon \sim \mathcal{N}(0,\varphi^{2}_{m}).$ All random-effect terms are assumed to be independent.
For the case when replicate measurements are assumed to be exchangeable for item $i$, $a_{ir}$ can be removed.

%%---Comparative Complexity
There is a substantial difference in the number of fixed parameters used by the respective models. For the model in (\ref{Roy-model}) requires two fixed effect parameters, i.e. the means of the two methods, for any number of items $N$. In contrast, the model described by (\ref{BXC-model}) requires $N+2$ fixed effects for $N$ items. The inclusion of fixed effects to account for the `true value' of each item greatly increases the level of model complexity.

%%---Estimability of Tau
When only two methods are compared, \citet{BXC2008} notes that separate estimates of $\tau^2_m$ can not be obtained due to the model over-specification. To overcome this, the assumption of equality, i.e. $\tau^2_1 = \tau^2_2$, is required.




\newpage
\section{MAY 2012 : Research Notes}
Roy (2009) proposes a suite of hypothesis tests for assessing the agreement of two methods of measurement, when replicate measurements are obtained for each item, using a LME approach. (An item would commonly be a patient).  Two methods of measurement can be said to be in agreement if there is no significant difference between in three key respects. Firstly, there is no inter-method bias between the two methods, i.e. there is no persistent tendency for one method to give higher values than the other.
Secondly, both methods of measurement have the same  within-subject variability. In such a case the variance of the replicate measurements would consistent for both methods.
Lastly, the methods have equal between-subject variability.  Put simply, for the mean measurements for each case, the variances of the mean measurements from both methods are equal.
\subsection{Testing for Inter-method Bias}
Firstly, a practitioner would investigate whether a significant inter-method bias is present between the methods. This bias is specified as a fixed effect in the LME model.  For a practitioner who has a reasonable level of competency in \texttt{R} and undergraduate statistics (in particular simple linear regression model) this is a straight-forward procedure.
\subsection{Reference Model (Ref.Fit)}
Conventionally LME models can be tested using Likelihood Ratio Tests, wherein a reference model is compared to a nested model.
\begin{framed}
\begin{verbatim}
> Ref.Fit = lme(y ~ meth-1, data = dat,   #Symm , Symm#
+     random = list(item=pdSymm(~ meth-1)), 
+     weights=varIdent(form=~1|meth),
+     correlation = corSymm(form=~1 | item/repl), 
+     method="ML")
\end{verbatim}
\end{framed}


Roy(2009) presents two nested models that specify the condition of equality as required, with a third nested model for an additional test. There three formulations share the same structure, and can be specified by making slight alterations of the code for the Reference Model.

\subsection{Nested Model (Between-Item Variability)}
\begin{framed}
\begin{verbatim}
> NMB.fit  = lme(y ~ meth-1, data = dat,   #CS , Symm#
+     random = list(item=pdCompSymm(~ meth-1)),
+     correlation = corSymm(form=~1 | item/repl), 
+     method="ML")
\end{verbatim}
\end{framed}

\newpage

\section{Introduction}

\citet{roy} uses an approach based on linear mixed effects (LME) models for the purpose of comparing the agreement between two methods of measurement, where replicate measurements on items (ofttimes individuals) by both methods are available. She provides three tests of hypothesis appropriate for evaluating the agreement between the two methods of measurement under this sampling scheme. These tests consider null hypotheses that assume: absence of inter-method bias; equality of between-subject variabilities of the two methods; equality of within-subject variabilities of the two methods. By inter-method bias we mean that a systematic difference exists between observations recorded by the two methods. Differences in between-subject variabilities of the two methods arise when one method is yielding average response levels for individuals than are more variable than the average response levels for the same sample of individuals taken by the other method.  Differences in within-subject variabilities of the two methods arise when one method is yielding responses for an individual than are more variable than the responses for this same individual taken by the other method. The two methods of measurement can be considered to agree, and subsequently can be used interchangeably, if all three null hypotheses are true.

\bigskip

%------------------------------------------------------------------------------------------------%
\section{Roy's Hypotheses Tests}

In order to express Roy's LME model in matrix notation we gather all $2n_i$ observations specific to item $i$ into a single vector  $\boldsymbol{y}_{i} = (y_{1i1},y_{2i1},y_{1i2},\ldots,y_{mir},\ldots,y_{1in_{i}},y_{2in_{i}})^\prime.$ The LME model can be written
\[
\boldsymbol{y_{i}} = \boldsymbol{X_{i}\beta} + \boldsymbol{Z_{i}b_{i}} + \boldsymbol{\epsilon_{i}},
\]
where $\boldsymbol{\beta}=(\beta_0,\beta_1,\beta_2)^\prime$ is a vector of fixed effects, and $\boldsymbol{X}_i$ is a corresponding $2n_i\times 3$ design matrix for the fixed effects. The random effects are expressed in the vector $\boldsymbol{b}=(b_1,b_2)^\prime$, with $\boldsymbol{Z}_i$ the corresponding $2n_i\times 2$ design matrix. The vector $\boldsymbol{\epsilon}_i$ is a $2n_i\times 1$ vector of residual terms. Random effects and residuals are assumed to be independent of each other.

The random effects are assumed to be distributed as $\boldsymbol{b}_i \sim \mathcal{N}_2(0,\boldsymbol{G})$. The between-item variance covariance matrix $\boldsymbol{G}$ is constructed as follows:
\[ \boldsymbol{G} =\left(
            \begin{array}{cc}
              g^2_1  & g_{12} \\
              g_{12} & g^2_2 \\
            \end{array}
          \right) \]

% This is probably a good place to discuss how R_i can  be interpreted as a Kroneckor product

The matrix of random errors $\boldsymbol{\epsilon}_i$ is distributed as $\mathcal{N}_2(0,\boldsymbol{R}_i)$.
\citet{hamlett} shows that the variance covariance matrix for the residuals(i.e. the within-item sources of variation between both methods) can be expressed as the Kroneckor product of an $n_i \times n_i$ identity matrix and the partial within-item variance covariance matrix $\boldsymbol{\Sigma}$, i.e. $\boldsymbol{R}_{i} = \boldsymbol{I}_{n_{i}} \otimes \boldsymbol{\Sigma}$.
\[
\boldsymbol{\Sigma} = \left( \begin{array}{cc}
  \sigma^2_{1} & \sigma_{12} \\
  \sigma_{12} & \sigma^2_{2} \\
\end{array}\right),
\]
where $\sigma^2_{1}$ and $\sigma^2_{2}$ are the within-subject variances of the respective methods, and $\sigma_{12}$ is the within-item covariance between the two methods. The within-item variance covariance matrix $\boldsymbol{\Sigma}$ is assumed to be the same for all replications.Computational analysis of linear mixed effects models allow for the explicit analysis of both $\boldsymbol{G}$ and $\boldsymbol{R_i}$.

For expository purposes consider the case where each item provides three replicate measurements by each method. In matrix form the model has the structure
\[
\boldsymbol{y}_{i} =
%---Design Matrix X ----%
\left(\begin{array}{ccc}
 1 & 1 & 0 \\ 1 & 0 & 1 \\ 1 & 1 & 0 \\
 1 & 0 & 1 \\ 1 & 1 & 0 \\ 1 & 0 & 1 \\
\end{array}\right)
%---FE Matrix----%
\left(\begin{array}{c}
 \beta_0 \\ \beta_1 \\ \beta_2 \\
\end{array}\right)
+
%---Design Matrix Z----%
\left(\begin{array}{cc}
1 & 0 \\0 & 1 \\1 & 0 \\0 & 1 \\0 & 1 \\\end{array}
\right)
%---RE Matrix----%
\left(\begin{array}{c}
b_{1i} \\   b_{2i} \\
\end{array}\right)
+
%------Errors Vector---%
\left( \begin{array}{c}
\epsilon_{1i1} \\\epsilon_{2i1} \\\epsilon_{1i2} \\ \epsilon_{2i2} \\\epsilon_{1i3} \\\epsilon_{2i3} \\
\end{array}\right).
\]
The between item variance covariance $\boldsymbol{G}$ is as before, while the within item variance covariance is given as
%------Specification of within item VC matrix R---%
\[
\boldsymbol{R}_i = \left(
\begin{array}{cccccc}
  \sigma^2_{1} & \sigma_{12} & 0 & 0 & 0 & 0 \\
  \sigma_{12} & \sigma^2_{2} & 0 & 0 & 0 & 0 \\
  0 & 0 & \sigma^2_{1} & \sigma_{12} & 0 & 0 \\
  0 & 0 & \sigma_{12} & \sigma^2_{2} & 0 & 0 \\
  0 & 0 & 0 & 0 & \sigma^2_{1} & \sigma_{12} \\
  0 & 0 & 0 & 0 & \sigma_{12} & \sigma^2_{2} \\
\end{array} \right)
\]

The overall variability between the two methods is the sum of between-item variability
$\boldsymbol{G}$ and partial within-item variability $\boldsymbol{\Sigma}$. \citet{roy} denotes the overall variability as ${\mbox{Block - }\boldsymbol \Omega_{i}}$. The overall variation for methods $1$ and $2$ are given by

%------Overall variability in terms of G and R ----%
\begin{equation}
\left(\begin{array}{cc}
              \omega^2_1  & \omega_{12} \\
              \omega_{12} & \omega^2_2 \\
       \end{array}  \right)
 =
\left(\begin{array}{cc}
              g^2_1  & g_{12} \\
              g_{12} & g^2_2 \\
\end{array} \right)
+
\left( \begin{array}{cc}
              \sigma^2_1  & \sigma_{12} \\
              \sigma_{12} & \sigma^2_2 \\
\end{array}\right)
\end{equation}
%------------------------------------------------------------------------%
\newpage
\subsection{Roy's hypothesis tests for variability}
% Three hypothesis tests follow from this equation.
Lack of agreement can arise if there is a disagreement in overall variabilities. This lack of agreement may be due to differing between-item variabilities, differing within-item variabilities, or both. The formulation presented above usefully facilitates a series of significance tests that assess if and where such differences arise. \citet{roy} allows for a formal test of each. These tests are comprised of a formal test for the equality of between-item variances,
\begin{eqnarray*}
\operatorname{H_0} : g^2_1 = g^2_2 \\
\operatorname{H_1} : g^2_1 \neq g^2_2
\end{eqnarray*}
a formal test for the equality of within-item variances,
\begin{eqnarray*}
\operatorname{H_0} : \sigma^2_1 = \sigma^2_2 \\
\operatorname{H_1} : \sigma^2_1 \neq \sigma^2_2
\end{eqnarray*}
and finally, a formal test for the equality of overall variances.
\begin{eqnarray*}
\operatorname{H_0} : \omega^2_1 = \omega^2_2 \\
\operatorname{H_1} : \omega^2_1 \neq \omega^2_2
\end{eqnarray*}

These tests are complemented by the ability to consider the inter-method bias and the overall correlation coefficient.
Two methods can be considered to be in agreement if criteria based upon these methodologies are met. Additionally Roy makes reference to the overall correlation coefficient of the two methods, which is determinable from variance estimates.

\subsection{Computation of limits of agreement under Roy's model}
The limits of agreement \citep{BA86} are ubiquitous in method comparison studies.
The computation thereof require that the variance of the difference of measurements. This variance is easily computable from the  variance estimates in the ${\mbox{Block - }\boldsymbol \Omega_{i}}$ matrix, i.e.
\[
% Check this
\operatorname{Var}(y_1 - y_2) = \sqrt{ \omega^2_1 + \omega^2_2 - 2\omega_{12}}.
\]




%With regards to the specification of the variance terms, Carstensen  remarks that using their approach is common, %remarking that \emph{ The only slightly non-standard (meaning ``not often used") feature is the differing residual %variances between methods }\citep{bxc2010}.

\newpage
\section{Roy's Hypotheses Tests}

In order to express Roy's LME model in matrix notation we gather all $2n_i$ observations specific to item $i$ into a single vector  $\boldsymbol{y}_{i} = (y_{1i1},y_{2i1},y_{1i2},\ldots,y_{mir},\ldots,y_{1in_{i}},y_{2in_{i}})^\prime.$ The LME model can be written
\[
\boldsymbol{y_{i}} = \boldsymbol{X_{i}\beta} + \boldsymbol{Z_{i}b_{i}} + \boldsymbol{\epsilon_{i}},
\]
where $\boldsymbol{\beta}=(\beta_0,\beta_1,\beta_2)^\prime$ is a vector of fixed effects, and $\boldsymbol{X}_i$ is a corresponding $2n_i\times 3$ design matrix for the fixed effects. The random effects are expressed in the vector $\boldsymbol{b}=(b_1,b_2)^\prime$, with $\boldsymbol{Z}_i$ the corresponding $2n_i\times 2$ design matrix. The vector $\boldsymbol{\epsilon}_i$ is a $2n_i\times 1$ vector of residual terms. Random effects and residuals are assumed to be independent of each other.

The random effects are assumed to be distributed as $\boldsymbol{b}_i \sim \mathcal{N}_2(0,\boldsymbol{G})$. The between-item variance covariance matrix $\boldsymbol{G}$ is constructed as follows:
\[ \boldsymbol{G} =\left(
            \begin{array}{cc}
              g^2_1  & g_{12} \\
              g_{12} & g^2_2 \\
            \end{array}
          \right) \]

The matrix of random errors $\boldsymbol{\epsilon}_i$ is distributed as $\mathcal{N}_2(0,\boldsymbol{R}_i)$.
\citet{hamlett} shows that the variance covariance matrix for the residuals(i.e. the within-item sources of variation between both methods) can be expressed as the Kroneckor product of an $n_i \times n_i$ identity matrix and the partial within-item variance covariance matrix $\boldsymbol{\Sigma}$, i.e. $\boldsymbol{R}_{i} = \boldsymbol{I}_{n_{i}} \otimes \boldsymbol{\Sigma}$.
\[
\boldsymbol{\Sigma} = \left( \begin{array}{cc}
  \sigma^2_{1} & \sigma_{12} \\
  \sigma_{12} & \sigma^2_{2} \\
\end{array}\right),
\]
where $\sigma^2_{1}$ and $\sigma^2_{2}$ are the within-subject variances of the respective methods, and $\sigma_{12}$ is the within-item covariance between the two methods. The within-item variance covariance matrix $\boldsymbol{\Sigma}$ is assumed to be the same for all replications.Computational analysis of linear mixed effects models allow for the explicit analysis of both $\boldsymbol{G}$ and $\boldsymbol{R_i}$.

For expository purposes consider the case where each item provides three replicate measurements by each method. In matrix form the model has the structure
\[
\boldsymbol{y}_{i} =
%---Design Matrix X ----%
\left(\begin{array}{ccc}
 1 & 1 & 0 \\ 1 & 0 & 1 \\ 1 & 1 & 0 \\
 1 & 0 & 1 \\ 1 & 1 & 0 \\ 1 & 0 & 1 \\
\end{array}\right)
%---FE Matrix----%
\left(\begin{array}{c}
 \beta_0 \\ \beta_1 \\ \beta_2 \\
\end{array}\right)
+
%---Design Matrix Z----%
\left(\begin{array}{cc}
1 & 0 \\0 & 1 \\1 & 0 \\0 & 1 \\0 & 1 \\\end{array}
\right)
%---RE Matrix----%
\left(\begin{array}{c}
b_{1i} \\   b_{2i} \\
\end{array}\right)
+
%------Errors Vector---%
\left( \begin{array}{c}
\epsilon_{1i1} \\\epsilon_{2i1} \\\epsilon_{1i2} \\ \epsilon_{2i2} \\\epsilon_{1i3} \\\epsilon_{2i3} \\
\end{array}\right).
\]
The between item variance covariance $\boldsymbol{G}$ is as before, while the within item variance covariance is given as
%------Specification of within item VC matrix R---%
\[
\boldsymbol{R}_i = \left(
\begin{array}{cccccc}
  \sigma^2_{1} & \sigma_{12} & 0 & 0 & 0 & 0 \\
  \sigma_{12} & \sigma^2_{2} & 0 & 0 & 0 & 0 \\
  0 & 0 & \sigma^2_{1} & \sigma_{12} & 0 & 0 \\
  0 & 0 & \sigma_{12} & \sigma^2_{2} & 0 & 0 \\
  0 & 0 & 0 & 0 & \sigma^2_{1} & \sigma_{12} \\
  0 & 0 & 0 & 0 & \sigma_{12} & \sigma^2_{2} \\
\end{array} \right)
\]

The overall variability between the two methods is the sum of between-item variability
$\boldsymbol{G}$ and partial within-item variability $\boldsymbol{\Sigma}$. \citet{roy} denotes the overall variability as ${\mbox{Block - }\boldsymbol \Omega_{i}}$. The overall variation for methods $1$ and $2$ are given by

%------Overall variability in terms of G and R ----%
\begin{equation}
\left(\begin{array}{cc}
              \omega^2_1  & \omega_{12} \\
              \omega_{12} & \omega^2_2 \\
       \end{array}  \right)
 =
\left(\begin{array}{cc}
              g^2_1  & g_{12} \\
              g_{12} & g^2_2 \\
\end{array} \right)
+
\left( \begin{array}{cc}
              \sigma^2_1  & \sigma_{12} \\
              \sigma_{12} & \sigma^2_2 \\
\end{array}\right)
\end{equation}
%------------------------------------------------------------------------%
\newpage
\subsection{Roy's hypothesis tests for variability}
% Three hypothesis tests follow from this equation.
Lack of agreement can arise if there is a disagreement in overall variabilities. This lack of agreement may be due to differing between-item variabilities, differing within-item variabilities, or both. The formulation presented above usefully facilitates a series of significance tests that assess if and where such differences arise. \citet{roy} allows for a formal test of each. These tests are comprised of a formal test for the equality of between-item variances,
\begin{eqnarray*}
\operatorname{H_2} : g^2_1 = g^2_2 \\
\operatorname{K_2} : g^2_1 \neq g^2_2
\end{eqnarray*}
and a formal test for the equality of within-item variances.
\begin{eqnarray*}
\operatorname{H_3} : \sigma^2_1 = \sigma^2_2 \\
\operatorname{K_3} : \sigma^2_1 \neq \sigma^2_2
\end{eqnarray*}
A formal test for the equality of overall variances is also presented.
\begin{eqnarray*}
\operatorname{H_4} : \omega^2_1 = \omega^2_2 \\
\operatorname{K_4} : \omega^2_1 \neq \omega^2_2
\end{eqnarray*}

These tests are complemented by the ability to consider the inter-method bias and the overall correlation coefficient.
Two methods can be considered to be in agreement if criteria based upon these methodologies are met. Additionally Roy makes reference to the overall correlation coefficient of the two methods, which is determinable from variance estimates.
\newpage


%----------------------------------------------------------------------------------------%
\newpage
%\chapter{Limits of Agreement}

%\section{Modelling Agreement with LME Models}


\section{Introduction}

\citet{roy} uses an approach based on linear mixed effects (LME) models for the purpose of comparing the agreement between two methods of measurement, where replicate measurements on items, typically individuals, by both methods are available. She provides three tests of hypothesis appropriate for evaluating the agreement between the two methods of measurement under this sampling scheme. These tests consider null hypotheses that assume: absence of inter-method bias; equality of between-subject variabilities of the two methods; equality of within-subject variabilities of the two methods. By inter-method bias we mean that a systematic difference exists between observations recorded by the two methods. Differences in between-subject variabilities of the two methods arise when one method is yielding average response levels for individuals than are more variable than the average response levels for the same sample of individuals taken by the other method.  Differences in within-subject variabilities of the two methods arise when one method is yielding responses for an individual than are more variable than the responses for this same individual taken by the other method. The two methods of measurement can be considered to agree, and subsequently can be used interchangeably, if all three null hypotheses are true.


\bigskip

Let $y_{mir} $ be the $r$th replicate measurement on the $i$th item by the $m$th method, where $m=1,2,$ $i=1,\ldots,N,$ and $r = 1,\ldots,n_i.$ When the design is balanced and there is no ambiguity we can set $n_i=n.$ The LME model can be written
\begin{equation}
y_{mir} = \beta_{0} + \beta_{m} + b_{mi} + \epsilon_{mir}.
\end{equation}
Here $\beta_0$ and $\beta_m$ are fixed-effect terms representing, respectively, a model intercept and an overall effect for method $m.$ The $b_{1i}$ and $b_{2i}$ terms represent random effect parameters corresponding to the two methods, having $\mathrm{E}(b_{mi})=0$ with $\mathrm{Var}(b_{mi})=g^2_m$ and $\mathrm{Cov}(b_{mi}, b_{m^\prime i})=g_{12}.$ The random error term for each response is denoted $\epsilon_{mir}$ having $\mathrm{E}(\epsilon_{mir})=0$, $\mathrm{Var}(\epsilon_{mir})=\sigma^2_m$, $\mathrm{Cov}(b_{mir}, b_{m^\prime ir})=\sigma_{12}$, $\mathrm{Cov}(\epsilon_{mir}, \epsilon_{mir^\prime})= 0$ and $\mathrm{Cov}(\epsilon_{mir}, \epsilon_{m^\prime ir^\prime})= 0.$
When two methods of measurement are in agreement, there is no significant differences between $\beta_1$ and $\beta_2,$ $g^2_1 $ and $ g^2_2$, and $\sigma^2_1 $ and $ \sigma^2_2$.
\bigskip

% Complete paragraph by specifying variances and covariances for epsilons.
% I thing that these are your sigmas?
% Also, state equality of the parameters in this model when each of the three hypotheses above are true.



\section{Roy's Hypotheses Tests}

In order to express Roy's LME model in matrix notation we gather all $2n_i$ observations specific to item $i$ into a single vector  $\boldsymbol{y}_{i} = (y_{1i1},y_{2i1},y_{1i2},\ldots,y_{mir},\ldots,y_{1in_{i}},y_{2in_{i}})^\prime.$ The LME model can be written
\[
\boldsymbol{y_{i}} = \boldsymbol{X_{i}\beta} + \boldsymbol{Z_{i}b_{i}} + \boldsymbol{\epsilon_{i}},
\]
where $\boldsymbol{\beta}=(\beta_0,\beta_1,\beta_2)^\prime$ is a vector of fixed effects, and $\boldsymbol{X}_i$ is a corresponding $2n_i\times 3$ design matrix for the fixed effects. The random effects are expressed in the vector $\boldsymbol{b}=(b_1,b_2)^\prime$, with $\boldsymbol{Z}_i$ the corresponding $2n_i\times 2$ design matrix. The vector $\boldsymbol{\epsilon}_i$ is a $2n_i\times 1$ vector of residual terms.

The distribution of the random effects is described as $\boldsymbol{b}_i \sim N(0,\boldsymbol{G})$. Similarly  random errors are distributed as $\boldsymbol{\epsilon}_i \sim N(0,\boldsymbol{R}_i)$. The random effects and residuals are assumed to be independent. Both covariance matrices can be written as follows;
% Assumptions made on the structures of $\boldsymbol{G}$ and $\boldsymbol{R}_i$ will be discussed in due course.


\[ \boldsymbol{G} =\left(
\begin{array}{cc}
g^2_1  & g_{12} \\
g_{12} & g^2_2 \\
\end{array}
\right) \]
and


\[ \boldsymbol{R}_i =\left(
\begin{array}{cccccccc}
\sigma^2_1  & \sigma_{12} & 0 & 0 & \ldots & \ldots & 0 & 0 \\
\sigma_{12} & \sigma^2_2  & 0 & 0  & \ldots & \ldots & 0 & 0\\

0 & 0 &\sigma^2_1  & \sigma_{12} & \ldots & \ldots& 0 &  0 \\
0 & 0 &\sigma_{12} & \sigma^2_2  & \ldots & \ldots & 0 & 0 \\
\vdots & \vdots &\vdots & \vdots & \ddots & \ddots& \vdots & \vdots \\

0 & 0 &0 & 0 & \ldots & \ldots&\sigma^2_1  & \sigma_{12} \\
0 & 0 &0 & 0 & \ldots & \ldots &\sigma_{12} & \sigma^2_2 \\
\end{array}
\right). \]



\bigskip
The above terms can be used to express the  variance covariance matrix $\boldsymbol{\Omega}_i$ for the responses on item $i$ ,
\[
\boldsymbol{\Omega}_i = \boldsymbol{Z}_i \boldsymbol{G} \boldsymbol{Z}_i^{\prime} + \boldsymbol{R}_i.
\]

\bigskip

For expository purposes consider the case where each item provides three replicates by each method. Then in matrix notation the model has the structure
\[
\boldsymbol{y}_{i} =
\left(
\begin{array}{c}
y_{1i1} \\
y_{2i1} \\
y_{1i2} \\
y_{2i2} \\
y_{1i3} \\
y_{2i3} \\
\end{array}
\right) = 
\left(
\begin{array}{ccc}
1 & 1 & 0 \\
1 & 0 & 1 \\
1 & 1 & 0 \\
1 & 0 & 1 \\
1 & 1 & 0 \\
1 & 0 & 1 \\
\end{array}
\right)
\left(
\begin{array}{c}
\beta_0 \\ \beta_1 \\ \beta_2 \\
\end{array}
\right)
+
\left(
\begin{array}{cc}
1 & 0 \\
0 & 1 \\
1 & 0 \\
0 & 1 \\
1 & 0 \\
0 & 1 \\
\end{array}
\right)\left(
\begin{array}{c}
b_{1i} \\   b_{2i} \\
\end{array}
\right)
+
\left(
\begin{array}{c}
\epsilon_{1i1} \\
\epsilon_{2i1} \\
\epsilon_{1i2} \\
\epsilon_{2i2} \\
\epsilon_{1i3} \\
\epsilon_{2i3} \\
\end{array}
\right).
\]

The partial within-item variance covariance matrix of two methods at any replicate is denoted $\boldsymbol{\Sigma}$, where $\sigma^2_{1}$ and $\sigma^2_{2}$ are the within-subject variances of both methods, and $\sigma_{12}$ is the within-item covariance between the two methods. The within-item variance covariance matrix $\boldsymbol{\Sigma}$ is assumed to be the same for all replications.

\[
\boldsymbol{\Sigma} = \left( \begin{array}{cc}
\sigma^2_{1} & \sigma_{12} \\
\sigma_{12} & \sigma^2_{2} \\
\end{array}\right).
\]

\citet{hamlett} shows that $\boldsymbol{R}_{i}$  can be expressed as $\boldsymbol{I}_{n_{i}} \otimes \boldsymbol{\Sigma}$. The covariance matrix has the same structure for all items, except for dimension, which depends on the number of replicates. The $2 \times 2$ block diagonal Block-$\boldsymbol{\Omega}_{i}$ represents the covariance matrix between two methods, and is the sum of $\boldsymbol{G}$ and $\boldsymbol{\Sigma}$.

\[ \textrm{Block-}\boldsymbol{\Omega}_{i}  = \left(\begin{array}{cc}
\omega^2_1  & \omega_{12} \\
\omega_{12} & \omega^2_2 \\
\end{array}  \right)
=  \left(
\begin{array}{cc}
g^2_1  & g_{12} \\
g_{12} & g^2_2 \\
\end{array} \right)+
\left(
\begin{array}{cc}
\sigma^2_1  & \sigma_{12} \\
\sigma_{12} & \sigma^2_2 \\
\end{array}\right)
\]

The variance of case-wise difference in measurements can be determined from Block-$\boldsymbol{\Omega}_{i}$. Hence limits of agreement can be computed.

\newpage
\subsection{Remarks}
Lack of agreement can arise if there is a disagreement in overall variabilities. This may be due to due to the disagreement in either between-item
variabilities or within-item variabilities, or both. \citet{roy} allows for a formal test of each.

\subsection{Hypothesis Testing}
The formulation presented above usefully facilitates a series of
significance tests that advise as to how well the two methods
agree. These tests are as follows:
\begin{itemize}
	\item A formal test for the equality of between-item variances,
	\item A formal test for the equality of within-item variances,
	\item A formal test for the equality of overall variances.
\end{itemize}
These tests are complemented by the ability to consider the inter-method bias and the overall correlation coefficient. Two methods can be considered to be in agreement if criteria based upon these methodologies are met. Additionally Roy makes reference to the overall correlation coefficient of the two methods, which is determinable from variance estimates.

\newpage
\section{Carstensen's Limits of agreement}
\citet{bxc2008} presents a methodology to compute the limits of
agreement based on LME models. Importantly, Carstensen's underlying model differs from Roy's model in some key respects, and therefore a prior discussion of Carstensen's model is required.

\subsection{Carstensen's Model}

\citet{BXC2004} presents a model to describe the relationship between a value of measurement and its
real value. The non-replicate case is considered first, as it is the context of the Bland Altman plots. This model assumes that inter-method bias is the only difference between the two methods.

A measurement $y_{mi}$ by method $m$ on individual $i$ is formulated as follows;
\begin{equation}
y_{mi}  = \alpha_{m} + \mu_{i} + e_{mi} \qquad  e_{mi} \sim
\mathcal{N}(0,\sigma^{2}_{m})
\end{equation}
The differences are expressed as $d_{i} = y_{1i} - y_{2i}$. For the replicate case, an interaction term $c$ is added to the model, with an associated variance component. All the random effects are assumed independent, and that all replicate measurements are assumed to be exchangeable within each method.

\begin{equation}
y_{mir}  = \alpha_{m} + \mu_{i} + c_{mi} + e_{mir}, \qquad  e_{mi}
\sim \mathcal{N}(0,\sigma^{2}_{m}), \quad c_{mi} \sim \mathcal{N}(0,\tau^{2}_{m}).
\end{equation}
%----

Of particular importance is terms of the model, a true value for item $i$ ($\mu_{i}$).  The fixed effect of Roy's model comprise of an intercept term and fixed effect terms for both methods, with no reference to the true value of any individual item. A distinction can be made between the two models: Roy's model is a standard LME model, whereas Carstensen's model is a more complex additive model.


\subsection{Assumptions on Variability}

Aside from the fixed effects, another important difference is that Carstensen's model requires that particular assumptions be applied, specifically that the off-diagonal elements of the between-item
and within-item variability matrices are zero. By extension the
overall variability off diagonal elements are also zero.

Also, implementation requires that the between-item variances are
estimated as the same value: $g^2_1 = g^2_2 = g^2$. Necessarily
Carstensen's method does not allow for a formal test of the
between-item variability.

\[\left(\begin{array}{cc}
\omega^1_2  & 0 \\
0 & \omega^2_2 \\
\end{array}  \right)
=  \left(
\begin{array}{cc}
g^2  & 0 \\
0 & g^2 \\
\end{array} \right)+
\left(
\begin{array}{cc}
\sigma^2_1  & 0 \\
0 & \sigma^2_2 \\
\end{array}\right)
\]

In cases where the off-diagonal terms in the overall variability
matrix are close to zero, the limits of agreement due to
\citet{bxc2008} are very similar to the limits of agreement that
follow from the general model.

\newpage


\section{Note 1: Coefficient of Repeatability}
The coefficient of repeatability is a measure of how well a
measurement method agrees with itself over replicate measurements
\citep{BA99}. Once the within-item variability is known, the
computation of the coefficients of repeatability for both methods
is straightforward.


\section{Note 2: Model terms}
It is important to note the following characteristics of this model.
\begin{itemize}
	\item Let the number of replicate measurements on each item $i$ for both methods be $n_i$, hence $2 \times n_i$ responses. However, it is assumed that there may be a different number of replicates made for different items. Let the maximum number of replicates be $p$. An item will have up to $2p$ measurements, i.e. $\max(n_{i}) = 2p$.
	
	% \item $\boldsymbol{y}_i$ is the $2n_i \times 1$ response vector for measurements on the $i-$th item.
	% \item $\boldsymbol{X}_i$ is the $2n_i \times  3$ model matrix for the fixed effects for observations on item $i$.
	% \item $\boldsymbol{\beta}$ is the $3 \times  1$ vector of fixed-effect coefficients, one for the true value for item $i$, and one effect each for both methods.
	
	\item Later on $\boldsymbol{X}_i$ will be reduced to a $2 \times 1$ matrix, to allow estimation of terms. This is due to a shortage of rank. The fixed effects vector can be modified accordingly.
	\item $\boldsymbol{Z}_i$ is the $2n_i \times  2$ model matrix for the random effects for measurement methods on item $i$.
	\item $\boldsymbol{b}_i$ is the $2 \times  1$ vector of random-effect coefficients on item $i$, one for each method.
	%\item $\boldsymbol{\epsilon}$  is the $2n_i \times  1$ vector of residuals for measurements on item $i$.
	%\item $\boldsymbol{G}$ is the $2 \times  2$ covariance matrix for the random effects.
	%\item $\boldsymbol{R}_i$ is the $2n_i \times  2n_i$ covariance matrix for the residuals on item $i$.
	\item The expected value is given as $\mbox{E}(\boldsymbol{y}_i) = \boldsymbol{X}_i\boldsymbol{\beta}.$ \citep{hamlett}
	\item The variance of the response vector is given by $\mbox{Var}(\boldsymbol{y}_i)  = \boldsymbol{Z}_i \boldsymbol{G} \boldsymbol{Z}_i^{\prime} + \boldsymbol{R}_i$ \citep{hamlett}.
\end{itemize}
\bibliography{DB-txfrbib}
\end{document}
